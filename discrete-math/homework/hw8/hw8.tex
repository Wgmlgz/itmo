\documentclass{article}
\usepackage{import}
\import{../../../lib/latex/}{wgmlgz}

\begin{document}

\itmo[
       variant=111,
       labn=8,
       worktype=Домашняя работа,
       discipline=Дискретная математика,
       group=P3115,
       student=Владимир Мацюк,
       teacher=Поляков Владимир Иванович,
       logo=../../../lib/img/itmo.png
]

\newcommand{\car}{\multicolumn{1}{c@{\hspace*{\tabcolsep}\makebox[0pt]{\curvearrowleft}}}{}}
\newcommand{\rcar}{\multicolumn{1}{c@{\hspace*{\tabcolsep}\makebox[0pt]{\curvearrowright}}}{}}
\newcommand{\ncar}{\multicolumn{1}{c@{\hspace*{\tabcolsep}\makebox[0pt]{}}}{}}
\newcommand{\SPACE}{\multicolumn{12}{c}{}}
\newcommand{\INT}{\multicolumn{5}{c}{\MM{Интерпретации}}}
\newcommand{\PLUS}{\multirow{2}{*}{+}}
\newcommand{\MINUS}{\multirow{2}{*}{-}}
\newcommand{\SIGN}{\multicolumn{2}{c}{\MM{Знаковая}}}
\newcommand{\USIGN}{\multicolumn{2}{c}{\MM{Беззнаковая}}}

\section{Числа}
$$
       \begin{array}{|c|c|}
              \hline
              A & 4,7 \nl
              B & 0,053 \nl
       \end{array}
$$
\section{Задание}

\begin{enumerate}
       \item Заданные числа А (делимое) и В (делитель) представить в форматах Ф1 и Ф2 с укороченной мантиссой (8 двоичных разрядов). Метод округления выбирается произвольно.
       \\ Примечание: общее число разрядов в формате – 16.
       \item Выполнить операцию деления операндов в формате Ф1.
       \item В случае положительного результата «пробного» вычитания сохранить младшую тетраду.
       \item Выполнить операцию деления операндов в формате Ф2.
       \item Результаты представить в форматах операндов, перевести в десятичную систему счисления и проверить их правильность.
       \item Определить абсолютную и относительную погрешности результатов и обосновать их причину.
       Варианты задания приведены в табл. 8 Приложения 1.
       
             
\end{enumerate}
\section{Решение}
\begin{enumerate}
       \item Деление в формате Ф1:

$$ A = 4.7 = (4.B30)_{16} = (0.4B)_{16} * 16^1 $$
$$ B = 0.053 = (0.0D910)_{16} = (0.D9)_{16} * 16^{-1} $$
$$ Xc = Xa - Xb + d $$
$$ Xc = 65 - 63 + 64 = 66 $$
$$ Pc = 2 $$
$$\begin{array}{c} \\
    \\ \begin{array}{|c|c|c|c|} \hline \textup{N\ шага}              & \textup{Действие}                                                                                       & \textup{Делимое}                                                           & \textup{Частное}                                                \\ \hline
             \begin{array}{c}$$$$ \\ $$$$ \\ $$0$$\end{array} & \begin{array}{c}$$R0$$ \\ $$[-M_b]доп$$ \\ $$Ma$$\end{array}            & \begin{array}{c}$$101110010$$ \\ $$100100111$$ \\ $$001001011$$\end{array} & \begin{array}{c}$$00000000$$ \\ $$$$ \\ $$00000000$$\end{array} \\ \hline
             \begin{array}{c}$$$$ \\ $$$$ \\ $$1$$\end{array} & \begin{array}{c}$$R1$$ \\ $$[M_b]пр$$ \\ $$\leftarrow R0$$\end{array}  & \begin{array}{c}$$110111101$$ \\ $$011011001$$ \\ $$011100100$$\end{array} & \begin{array}{c}$$00000000$$ \\ $$$$ \\ $$00000000$$\end{array} \\ \hline
             \begin{array}{c}$$$$ \\ $$$$ \\ $$2$$\end{array} & \begin{array}{c}$$R2$$ \\ $$[M_b]пр$$ \\ $$\leftarrow R1$$\end{array}  & \begin{array}{c}$$001010011$$ \\ $$011011001$$ \\ $$101111010$$\end{array} & \begin{array}{c}$$00000001$$ \\ $$$$ \\ $$00000000$$\end{array} \\ \hline
             \begin{array}{c}$$$$ \\ $$$$ \\ $$3$$\end{array} & \begin{array}{c}$$R3$$ \\ $$[-M_b]доп$$ \\ $$\leftarrow R2$$\end{array} & \begin{array}{c}$$111001101$$ \\ $$100100111$$ \\ $$010100110$$\end{array} & \begin{array}{c}$$00000010$$ \\ $$$$ \\ $$00000010$$\end{array} \\ \hline
             \begin{array}{c}$$$$ \\ $$$$ \\ $$4$$\end{array} & \begin{array}{c}$$R4$$ \\ $$[M_b]пр$$ \\ $$\leftarrow R3$$\end{array}  & \begin{array}{c}$$001110011$$ \\ $$011011001$$ \\ $$110011010$$\end{array} & \begin{array}{c}$$00000101$$ \\ $$$$ \\ $$00000100$$\end{array} \\ \hline
             \begin{array}{c}$$$$ \\ $$$$ \\ $$5$$\end{array} & \begin{array}{c}$$R5$$ \\ $$[-M_b]доп$$ \\ $$\leftarrow R4$$\end{array} & \begin{array}{c}$$000001101$$ \\ $$100100111$$ \\ $$011100110$$\end{array} & \begin{array}{c}$$00001011$$ \\ $$$$ \\ $$00001010$$\end{array} \\ \hline
             \begin{array}{c}$$$$ \\ $$$$ \\ $$6$$\end{array} & \begin{array}{c}$$R6$$ \\ $$[-M_b]доп$$ \\ $$\leftarrow R5$$\end{array} & \begin{array}{c}$$101000001$$ \\ $$100100111$$ \\ $$000011010$$\end{array} & \begin{array}{c}$$00010110$$ \\ $$$$ \\ $$00010110$$\end{array} \\ \hline
             \begin{array}{c}$$$$ \\ $$$$ \\ $$7$$\end{array} & \begin{array}{c}$$R7$$ \\ $$[M_b]пр$$ \\ $$\leftarrow R6$$\end{array}  & \begin{array}{c}$$101011011$$ \\ $$011011001$$ \\ $$010000010$$\end{array} & \begin{array}{c}$$00101100$$ \\ $$$$ \\ $$00101100$$\end{array} \\ \hline
             \begin{array}{c}$$$$ \\ $$$$ \\ $$8$$\end{array} & \begin{array}{c}$$R8$$ \\ $$[M_b]пр$$ \\ $$\leftarrow R7$$\end{array}  & \begin{array}{c}$$110001111$$ \\ $$011011001$$ \\ $$010110110$$\end{array} & \begin{array}{c}$$01011000$$ \\ $$$$ \\ $$01011000$$\end{array} \\ \hline
    \end{array} \\
    \\
    \\\end{array}$$

$$ C^* = (0.58)_{16} * 16^2 = (58)_{16} = 88 $$
$$ \Delta C = C^T - C^* = 88.679 - 88 = 0.679 $$
$$ δC = |\Delta C/C^T| * 100\% = |0.679/88.679| * 100\% = 0.77\% $$
Погрешность вызвана неточным представлением операндов

\item Деление в формате Ф2:

$$ A = 4.7 = (100.101100110)_2 = (0.10010110)_2 * 2^3 $$
$$ B = 0.053 = (0.0000110110010001)_2 = (0.11011001)_2 * 2^{-4} $$
$$ Xc = Xa - Xb + d $$
$$ Xc = 131 - 124 + 128 = 135 $$
$$ Pc = 7 $$

  $$\begin{array}{c}(A>0,\ B>0) \\ 
\left.[A]_{пр.}=0111000;\ [B]_{пр.}=1011011\right. \\ 
 \\ \begin{array}{|c|c|c|c|p{5cm}|} \hline N\ \textup{шага} & \textup{Операнды\ и\ действия} & \begin{array}{c}\textup{СЧП}\ (\textup{старшие}\\\textup{разряды})\end{array} & \begin{array}{c}\textup{Множитель}\ и\\\textup{СЧП}\ (\textup{младшие}\\\textup{разряды})\end{array} & \textup{Пояснения} \\ \hline 
0 & \textup{СЧП} & 0\ 0\ 0\ 0\ 0\ 0\ 0 & 1\ 0\ 1\ 1\ 0\ 1\ \underline{1} & \textup{Обнуление\ старших\ разрядов\ СЧП} \\ \hline 
1 & \begin{array}{c} [A]_{\textup{доп.}}\\ \textup{СЧП}\\\textup{СЧП}\rightarrow\end{array} & \begin{array}{c} 0\ 1\ 1\ 1\ 0\ 0\ 0 \\ \hline 1\ 0\ 0\ 1\ 0\ 0\ 0 \\ 1\ 1\ 0\ 0\ 1\ 0\ 0 \end{array} & \begin{array}{c}  \\ |\ 1\ 0\ 1\ 1\ 0\ 1\ 1 \\ 0\ |\ 1\ 0\ 1\ 1\ 0\ \underline{1} \end{array} & \textup{Вычитание\ множимого\ из\ СЧП.;\ Сдвиг \ СЧП\ и\ множителя\ вправо} \\ \hline 
2 & \textup{СЧП} \rightarrow & 1\ 1\ 1\ 0\ 0\ 1\ 0 & 0\ 0\ |\ 1\ 0\ 1\ 1\ \underline{0} & \textup{При\ сдвиге\ младший\ разряд\ не\ изменился;\ Сдвиг\ СЧП\ и\ множителя\ вправо} \\ \hline 
3 & \begin{array}{c} [A]_{\textup{пр.}}\\ \textup{СЧП}\\\textup{СЧП}\rightarrow\end{array} & \begin{array}{c} 1\ 0\ 0\ 1\ 0\ 0\ 0 \\ \hline 0\ 1\ 0\ 1\ 0\ 1\ 0 \\ 0\ 0\ 1\ 0\ 1\ 0\ 1 \end{array} & \begin{array}{c}  \\ 0\ 0\ |\ 1\ 0\ 1\ 1\ 0 \\ 0\ 0\ 0\ |\ 1\ 0\ 1\ \underline{1} \end{array} & \textup{Cложение\ СЧП\ с\ множимым.;\ Сдвиг \ СЧП\ и\ множителя\ вправо} \\ \hline 
4 & \begin{array}{c} [A]_{\textup{доп.}}\\ \textup{СЧП}\\\textup{СЧП}\rightarrow\end{array} & \begin{array}{c} 0\ 1\ 1\ 1\ 0\ 0\ 0 \\ \hline 1\ 0\ 1\ 1\ 1\ 0\ 1 \\ 1\ 1\ 0\ 1\ 1\ 1\ 0 \end{array} & \begin{array}{c}  \\ 0\ 0\ 0\ |\ 1\ 0\ 1\ 1 \\ 1\ 0\ 0\ 0\ |\ 1\ 0\ \underline{1} \end{array} & \textup{Вычитание\ множимого\ из\ СЧП.;\ Сдвиг \ СЧП\ и\ множителя\ вправо} \\ \hline 
5 & \textup{СЧП} \rightarrow & 1\ 1\ 1\ 0\ 1\ 1\ 1 & 0\ 1\ 0\ 0\ 0\ |\ 1\ \underline{0} & \textup{При\ сдвиге\ младший\ разряд\ не\ изменился;\ Сдвиг\ СЧП\ и\ множителя\ вправо} \\ \hline 
6 & \begin{array}{c} [A]_{\textup{пр.}}\\ \textup{СЧП}\\\textup{СЧП}\rightarrow\end{array} & \begin{array}{c} 1\ 0\ 0\ 1\ 0\ 0\ 0 \\ \hline 0\ 1\ 0\ 1\ 1\ 1\ 1 \\ 0\ 0\ 1\ 0\ 1\ 1\ 1 \end{array} & \begin{array}{c}  \\ 0\ 1\ 0\ 0\ 0\ |\ 1\ 0 \\ 1\ 0\ 1\ 0\ 0\ 0\ |\ \underline{1} \end{array} & \textup{Cложение\ СЧП\ с\ множимым.;\ Сдвиг \ СЧП\ и\ множителя\ вправо} \\ \hline 
7 & \begin{array}{c} [A]_{\textup{доп.}}\\ \textup{СЧП}\\\textup{СЧП}\rightarrow\end{array} & \begin{array}{c} 0\ 1\ 1\ 1\ 0\ 0\ 0 \\ \hline 1\ 0\ 0\ 1\ 1\ 1\ 1 \\ 0\ 1\ 0\ 0\ 1\ 1\ 1 \end{array} & \begin{array}{c}  \\ 1\ 0\ 1\ 0\ 0\ 0\ |\ 1 \\ 1\ 1\ 0\ 1\ 0\ 0\ \underline{0}\ | \end{array} & \textup{Вычитание\ множимого\ из\ СЧП.;\ Сдвиг \ СЧП\ и\ множителя\ вправо} \\ \hline 
 \end{array} \\
 \\ 
 \\  \left[C\right]_{\textup{пр.}}=01001111101000_2 = 5096_{10}\end{array}$$
  $$\begin{array}{c}(A<0,\ B>0) \\ 
\left.[-A]_{доп.}=1001000;\ [B]_{пр.}=1011011\right. \\ 
 \\ \begin{array}{|c|c|c|c|p{5cm}|} \hline N\ \textup{шага} & \textup{Операнды\ и\ действия} & \begin{array}{c}\textup{СЧП}\ (\textup{старшие}\\\textup{разряды})\end{array} & \begin{array}{c}\textup{Множитель}\ и\\\textup{СЧП}\ (\textup{младшие}\\\textup{разряды})\end{array} & \textup{Пояснения} \\ \hline 
0 & \textup{СЧП} & 0\ 0\ 0\ 0\ 0\ 0\ 0 & 1\ 0\ 1\ 1\ 0\ 1\ \underline{1} & \textup{Обнуление\ старших\ разрядов\ СЧП} \\ \hline 
1 & \begin{array}{c} [-A]_{\textup{пр.}}\\ \textup{СЧП}\\\textup{СЧП}\rightarrow\end{array} & \begin{array}{c} 0\ 1\ 1\ 1\ 0\ 0\ 0 \\ \hline 0\ 1\ 1\ 1\ 0\ 0\ 0 \\ 0\ 0\ 1\ 1\ 1\ 0\ 0 \end{array} & \begin{array}{c}  \\ |\ 1\ 0\ 1\ 1\ 0\ 1\ 1 \\ 0\ |\ 1\ 0\ 1\ 1\ 0\ \underline{1} \end{array} & \textup{Cложение\ СЧП\ с\ множимым.;\ Сдвиг \ СЧП\ и\ множителя\ вправо} \\ \hline 
2 & \textup{СЧП} \rightarrow & 0\ 0\ 0\ 1\ 1\ 1\ 0 & 0\ 0\ |\ 1\ 0\ 1\ 1\ \underline{0} & \textup{При\ сдвиге\ младший\ разряд\ не\ изменился;\ Сдвиг\ СЧП\ и\ множителя\ вправо} \\ \hline 
3 & \begin{array}{c} [-A]_{\textup{доп.}}\\ \textup{СЧП}\\\textup{СЧП}\rightarrow\end{array} & \begin{array}{c} 1\ 0\ 0\ 1\ 0\ 0\ 0 \\ \hline 1\ 0\ 1\ 0\ 1\ 1\ 0 \\ 1\ 1\ 0\ 1\ 0\ 1\ 1 \end{array} & \begin{array}{c}  \\ 0\ 0\ |\ 1\ 0\ 1\ 1\ 0 \\ 0\ 0\ 0\ |\ 1\ 0\ 1\ \underline{1} \end{array} & \textup{Вычитание\ множимого\ из\ СЧП.;\ Сдвиг \ СЧП\ и\ множителя\ вправо} \\ \hline 
4 & \begin{array}{c} [-A]_{\textup{пр.}}\\ \textup{СЧП}\\\textup{СЧП}\rightarrow\end{array} & \begin{array}{c} 0\ 1\ 1\ 1\ 0\ 0\ 0 \\ \hline 0\ 1\ 0\ 0\ 0\ 1\ 1 \\ 0\ 0\ 1\ 0\ 0\ 0\ 1 \end{array} & \begin{array}{c}  \\ 0\ 0\ 0\ |\ 1\ 0\ 1\ 1 \\ 1\ 0\ 0\ 0\ |\ 1\ 0\ \underline{1} \end{array} & \textup{Cложение\ СЧП\ с\ множимым.;\ Сдвиг \ СЧП\ и\ множителя\ вправо} \\ \hline 
5 & \textup{СЧП} \rightarrow & 0\ 0\ 0\ 1\ 0\ 0\ 0 & 1\ 1\ 0\ 0\ 0\ |\ 1\ \underline{0} & \textup{При\ сдвиге\ младший\ разряд\ не\ изменился;\ Сдвиг\ СЧП\ и\ множителя\ вправо} \\ \hline 
6 & \begin{array}{c} [-A]_{\textup{доп.}}\\ \textup{СЧП}\\\textup{СЧП}\rightarrow\end{array} & \begin{array}{c} 1\ 0\ 0\ 1\ 0\ 0\ 0 \\ \hline 1\ 0\ 1\ 0\ 0\ 0\ 0 \\ 1\ 1\ 0\ 1\ 0\ 0\ 0 \end{array} & \begin{array}{c}  \\ 1\ 1\ 0\ 0\ 0\ |\ 1\ 0 \\ 0\ 1\ 1\ 0\ 0\ 0\ |\ \underline{1} \end{array} & \textup{Вычитание\ множимого\ из\ СЧП.;\ Сдвиг \ СЧП\ и\ множителя\ вправо} \\ \hline 
7 & \begin{array}{c} [-A]_{\textup{пр.}}\\ \textup{СЧП}\\\textup{СЧП}\rightarrow\end{array} & \begin{array}{c} 0\ 1\ 1\ 1\ 0\ 0\ 0 \\ \hline 0\ 1\ 1\ 0\ 0\ 0\ 0 \\ 1\ 0\ 1\ 1\ 0\ 0\ 0 \end{array} & \begin{array}{c}  \\ 0\ 1\ 1\ 0\ 0\ 0\ |\ 1 \\ 0\ 0\ 1\ 1\ 0\ 0\ \underline{0}\ | \end{array} & \textup{Cложение\ СЧП\ с\ множимым.;\ Сдвиг \ СЧП\ и\ множителя\ вправо} \\ \hline 
 \end{array} \\
 \\ 
 \\  \left[C\right]_{\textup{доп}.}=10110000011000_2  \\
  \left[C\right]_{\textup{пр}.}= -01001111101000_2 = -5096_{10}\end{array}$$
  $$\begin{array}{c}(A>0,\ B<0) \\ 
\left.[A]_{пр.}=0111000;\ [-B]_{доп.}=0100101\right. \\ 
 \\ \begin{array}{|c|c|c|c|p{5cm}|} \hline N\ \textup{шага} & \textup{Операнды\ и\ действия} & \begin{array}{c}\textup{СЧП}\ (\textup{старшие}\\\textup{разряды})\end{array} & \begin{array}{c}\textup{Множитель}\ и\\\textup{СЧП}\ (\textup{младшие}\\\textup{разряды})\end{array} & \textup{Пояснения} \\ \hline 
0 & \textup{СЧП} & 0\ 0\ 0\ 0\ 0\ 0\ 0 & 0\ 1\ 0\ 0\ 1\ 0\ \underline{1} & \textup{Обнуление\ старших\ разрядов\ СЧП} \\ \hline 
1 & \begin{array}{c} [A]_{\textup{доп.}}\\ \textup{СЧП}\\\textup{СЧП}\rightarrow\end{array} & \begin{array}{c} 0\ 1\ 1\ 1\ 0\ 0\ 0 \\ \hline 1\ 0\ 0\ 1\ 0\ 0\ 0 \\ 1\ 1\ 0\ 0\ 1\ 0\ 0 \end{array} & \begin{array}{c}  \\ |\ 0\ 1\ 0\ 0\ 1\ 0\ 1 \\ 0\ |\ 0\ 1\ 0\ 0\ 1\ \underline{0} \end{array} & \textup{Вычитание\ множимого\ из\ СЧП.;\ Сдвиг \ СЧП\ и\ множителя\ вправо} \\ \hline 
2 & \begin{array}{c} [A]_{\textup{пр.}}\\ \textup{СЧП}\\\textup{СЧП}\rightarrow\end{array} & \begin{array}{c} 1\ 0\ 0\ 1\ 0\ 0\ 0 \\ \hline 0\ 0\ 1\ 1\ 1\ 0\ 0 \\ 0\ 0\ 0\ 1\ 1\ 1\ 0 \end{array} & \begin{array}{c}  \\ 0\ |\ 0\ 1\ 0\ 0\ 1\ 0 \\ 0\ 0\ |\ 0\ 1\ 0\ 0\ \underline{1} \end{array} & \textup{Cложение\ СЧП\ с\ множимым.;\ Сдвиг \ СЧП\ и\ множителя\ вправо} \\ \hline 
3 & \begin{array}{c} [A]_{\textup{доп.}}\\ \textup{СЧП}\\\textup{СЧП}\rightarrow\end{array} & \begin{array}{c} 0\ 1\ 1\ 1\ 0\ 0\ 0 \\ \hline 1\ 0\ 1\ 0\ 1\ 1\ 0 \\ 1\ 1\ 0\ 1\ 0\ 1\ 1 \end{array} & \begin{array}{c}  \\ 0\ 0\ |\ 0\ 1\ 0\ 0\ 1 \\ 0\ 0\ 0\ |\ 0\ 1\ 0\ \underline{0} \end{array} & \textup{Вычитание\ множимого\ из\ СЧП.;\ Сдвиг \ СЧП\ и\ множителя\ вправо} \\ \hline 
4 & \begin{array}{c} [A]_{\textup{пр.}}\\ \textup{СЧП}\\\textup{СЧП}\rightarrow\end{array} & \begin{array}{c} 1\ 0\ 0\ 1\ 0\ 0\ 0 \\ \hline 0\ 1\ 0\ 0\ 0\ 1\ 1 \\ 0\ 0\ 1\ 0\ 0\ 0\ 1 \end{array} & \begin{array}{c}  \\ 0\ 0\ 0\ |\ 0\ 1\ 0\ 0 \\ 1\ 0\ 0\ 0\ |\ 0\ 1\ \underline{0} \end{array} & \textup{Cложение\ СЧП\ с\ множимым.;\ Сдвиг \ СЧП\ и\ множителя\ вправо} \\ \hline 
5 & \textup{СЧП} \rightarrow & 0\ 0\ 0\ 1\ 0\ 0\ 0 & 1\ 1\ 0\ 0\ 0\ |\ 0\ \underline{1} & \textup{При\ сдвиге\ младший\ разряд\ не\ изменился;\ Сдвиг\ СЧП\ и\ множителя\ вправо} \\ \hline 
6 & \begin{array}{c} [A]_{\textup{доп.}}\\ \textup{СЧП}\\\textup{СЧП}\rightarrow\end{array} & \begin{array}{c} 0\ 1\ 1\ 1\ 0\ 0\ 0 \\ \hline 1\ 0\ 1\ 0\ 0\ 0\ 0 \\ 1\ 1\ 0\ 1\ 0\ 0\ 0 \end{array} & \begin{array}{c}  \\ 1\ 1\ 0\ 0\ 0\ |\ 0\ 1 \\ 0\ 1\ 1\ 0\ 0\ 0\ |\ \underline{0} \end{array} & \textup{Вычитание\ множимого\ из\ СЧП.;\ Сдвиг \ СЧП\ и\ множителя\ вправо} \\ \hline 
7 & \begin{array}{c} [A]_{\textup{пр.}}\\ \textup{СЧП}\\\textup{СЧП}\rightarrow\end{array} & \begin{array}{c} 1\ 0\ 0\ 1\ 0\ 0\ 0 \\ \hline 0\ 1\ 1\ 0\ 0\ 0\ 0 \\ 1\ 0\ 1\ 1\ 0\ 0\ 0 \end{array} & \begin{array}{c}  \\ 0\ 1\ 1\ 0\ 0\ 0\ |\ 0 \\ 0\ 0\ 1\ 1\ 0\ 0\ \underline{0}\ | \end{array} & \textup{Cложение\ СЧП\ с\ множимым.;\ Сдвиг \ СЧП\ и\ множителя\ вправо} \\ \hline 
 \end{array} \\
 \\ 
 \\  \left[C\right]_{\textup{доп}.}=10110000011000_2  \\
  \left[C\right]_{\textup{пр}.}= -01001111101000_2 = -5096_{10}\end{array}$$
  $$\begin{array}{c}(A<0,\ B<0) \\ 
\left.[-A]_{доп.}=1001000;\ [-B]_{доп.}=0100101\right. \\ 
 \\ \begin{array}{|c|c|c|c|p{5cm}|} \hline N\ \textup{шага} & \textup{Операнды\ и\ действия} & \begin{array}{c}\textup{СЧП}\ (\textup{старшие}\\\textup{разряды})\end{array} & \begin{array}{c}\textup{Множитель}\ и\\\textup{СЧП}\ (\textup{младшие}\\\textup{разряды})\end{array} & \textup{Пояснения} \\ \hline 
0 & \textup{СЧП} & 0\ 0\ 0\ 0\ 0\ 0\ 0 & 0\ 1\ 0\ 0\ 1\ 0\ \underline{1} & \textup{Обнуление\ старших\ разрядов\ СЧП} \\ \hline 
1 & \begin{array}{c} [-A]_{\textup{пр.}}\\ \textup{СЧП}\\\textup{СЧП}\rightarrow\end{array} & \begin{array}{c} 0\ 1\ 1\ 1\ 0\ 0\ 0 \\ \hline 0\ 1\ 1\ 1\ 0\ 0\ 0 \\ 0\ 0\ 1\ 1\ 1\ 0\ 0 \end{array} & \begin{array}{c}  \\ |\ 0\ 1\ 0\ 0\ 1\ 0\ 1 \\ 0\ |\ 0\ 1\ 0\ 0\ 1\ \underline{0} \end{array} & \textup{Cложение\ СЧП\ с\ множимым.;\ Сдвиг \ СЧП\ и\ множителя\ вправо} \\ \hline 
2 & \begin{array}{c} [-A]_{\textup{доп.}}\\ \textup{СЧП}\\\textup{СЧП}\rightarrow\end{array} & \begin{array}{c} 1\ 0\ 0\ 1\ 0\ 0\ 0 \\ \hline 1\ 1\ 0\ 0\ 1\ 0\ 0 \\ 1\ 1\ 1\ 0\ 0\ 1\ 0 \end{array} & \begin{array}{c}  \\ 0\ |\ 0\ 1\ 0\ 0\ 1\ 0 \\ 0\ 0\ |\ 0\ 1\ 0\ 0\ \underline{1} \end{array} & \textup{Вычитание\ множимого\ из\ СЧП.;\ Сдвиг \ СЧП\ и\ множителя\ вправо} \\ \hline 
3 & \begin{array}{c} [-A]_{\textup{пр.}}\\ \textup{СЧП}\\\textup{СЧП}\rightarrow\end{array} & \begin{array}{c} 0\ 1\ 1\ 1\ 0\ 0\ 0 \\ \hline 0\ 1\ 0\ 1\ 0\ 1\ 0 \\ 0\ 0\ 1\ 0\ 1\ 0\ 1 \end{array} & \begin{array}{c}  \\ 0\ 0\ |\ 0\ 1\ 0\ 0\ 1 \\ 0\ 0\ 0\ |\ 0\ 1\ 0\ \underline{0} \end{array} & \textup{Cложение\ СЧП\ с\ множимым.;\ Сдвиг \ СЧП\ и\ множителя\ вправо} \\ \hline 
4 & \begin{array}{c} [-A]_{\textup{доп.}}\\ \textup{СЧП}\\\textup{СЧП}\rightarrow\end{array} & \begin{array}{c} 1\ 0\ 0\ 1\ 0\ 0\ 0 \\ \hline 1\ 0\ 1\ 1\ 1\ 0\ 1 \\ 1\ 1\ 0\ 1\ 1\ 1\ 0 \end{array} & \begin{array}{c}  \\ 0\ 0\ 0\ |\ 0\ 1\ 0\ 0 \\ 1\ 0\ 0\ 0\ |\ 0\ 1\ \underline{0} \end{array} & \textup{Вычитание\ множимого\ из\ СЧП.;\ Сдвиг \ СЧП\ и\ множителя\ вправо} \\ \hline 
5 & \textup{СЧП} \rightarrow & 1\ 1\ 1\ 0\ 1\ 1\ 1 & 0\ 1\ 0\ 0\ 0\ |\ 0\ \underline{1} & \textup{При\ сдвиге\ младший\ разряд\ не\ изменился;\ Сдвиг\ СЧП\ и\ множителя\ вправо} \\ \hline 
6 & \begin{array}{c} [-A]_{\textup{пр.}}\\ \textup{СЧП}\\\textup{СЧП}\rightarrow\end{array} & \begin{array}{c} 0\ 1\ 1\ 1\ 0\ 0\ 0 \\ \hline 0\ 1\ 0\ 1\ 1\ 1\ 1 \\ 0\ 0\ 1\ 0\ 1\ 1\ 1 \end{array} & \begin{array}{c}  \\ 0\ 1\ 0\ 0\ 0\ |\ 0\ 1 \\ 1\ 0\ 1\ 0\ 0\ 0\ |\ \underline{0} \end{array} & \textup{Cложение\ СЧП\ с\ множимым.;\ Сдвиг \ СЧП\ и\ множителя\ вправо} \\ \hline 
7 & \begin{array}{c} [-A]_{\textup{доп.}}\\ \textup{СЧП}\\\textup{СЧП}\rightarrow\end{array} & \begin{array}{c} 1\ 0\ 0\ 1\ 0\ 0\ 0 \\ \hline 1\ 0\ 0\ 1\ 1\ 1\ 1 \\ 0\ 1\ 0\ 0\ 1\ 1\ 1 \end{array} & \begin{array}{c}  \\ 1\ 0\ 1\ 0\ 0\ 0\ |\ 0 \\ 1\ 1\ 0\ 1\ 0\ 0\ \underline{0}\ | \end{array} & \textup{Вычитание\ множимого\ из\ СЧП.;\ Сдвиг \ СЧП\ и\ множителя\ вправо} \\ \hline 
 \end{array} \\
 \\ 
 \\  \left[C\right]_{\textup{пр.}}=01001111101000_2 = 5096_{10}\end{array}$$


$$ C^* = (0.10110000)_2 * 2^7 = (1011000.0)_2 = 88 $$
$$ \Delta C = C^T - C^* = 88.679 - 88 = 0.679 $$
$$ δC = |\Delta C/C^T| * 100\% = |0.679/88.679| * 100\% = 0.77\% $$
Погрешность вызвана неточным представлением операндов

\end{enumerate}
\end{document}
