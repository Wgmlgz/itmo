\documentclass{article}
\usepackage{import}
\import{../../../lib/latex/}{wgmlgz}

\begin{document}

\itmo[
       variant=109,
       labn=7,
       worktype=Домашняя работа,
       discipline=Дискретная математика,
       group=P3115,
       student=Владимир Мацюк,
       teacher=Поляков Владимир Иванович,
       logo=../../../lib/img/itmo.png
]

\newcommand{\car}{\multicolumn{1}{c@{\hspace*{\tabcolsep}\makebox[0pt]{\curvearrowleft}}}{}}
\newcommand{\rcar}{\multicolumn{1}{c@{\hspace*{\tabcolsep}\makebox[0pt]{\curvearrowright}}}{}}
\newcommand{\ncar}{\multicolumn{1}{c@{\hspace*{\tabcolsep}\makebox[0pt]{}}}{}}
\newcommand{\SPACE}{\multicolumn{12}{c}{}}
\newcommand{\INT}{\multicolumn{5}{c}{\MM{Интерпретации}}}
\newcommand{\PLUS}{\multirow{2}{*}{+}}
\newcommand{\MINUS}{\multirow{2}{*}{-}}
\newcommand{\SIGN}{\multicolumn{2}{c}{\MM{Знаковая}}}
\newcommand{\USIGN}{\multicolumn{2}{c}{\MM{Беззнаковая}}}

\section{Числа}
$$
       \begin{array}{|c|c|}
              \hline
              A & 8,9 \nl
              B & 0,071 \nl
       \end{array}
$$
\section{Задание}

\begin{enumerate}
       \item Заданные числа А (множимое) и В (множитель) представить в фор-матах Ф1 и Ф2 с укороченной мантиссой (12 двоичных разрядов). Метод округления выбирается произвольно.
             Примечание: общее число разрядов в формате – 20.
       \item Выполнить операцию умножения операндов в формате Ф1, исполь-зуя метод ускоренного умножения мантисс на два разряда множите-ля.
       \item Выполнить операцию умножения операндов в формате Ф2, исполь-зуя метод ускоренного умножения мантисс на четыре разряда мно-жителя.
       \item Результаты представить в форматах операндов, перевести в десятич-ную систему счисления и проверить их правильность.
       \item Определить абсолютную и относительную погрешности результатов и обосновать их причину.
       \item Сравнить погрешности результатов аналогичных операций для форматов Ф1 и Ф2 и объяснить причины их сходства или различия.
             Варианты задания приведены в табл. 7 Приложения 1.
             
\end{enumerate}
\section{Решение}
\begin{enumerate}
       \item Формат Ф1 (число разрядов мантисы = 12):
             $$ A = 8,9_{10} = \textmd{8.E6}_{16} = \textmd{0.8E6}_{16} * 16^1 $$
             $$
                    \begin{array}{|c|c|c|}
                           \hline        
                           0 & 1000001 & 100011100110 \nl
                    \end{array}
             $$
             $$ A = 0,071_{10} = \textmd{0.123}_{16} = \textmd{0.123}_{16} * 16^0 $$
             $$
                    \begin{array}{|c|c|c|}
                           \hline        
                           0 & 1000000 & 000100100011 \nl
                    \end{array}
             $$
             $$ SignC = SignA \oplus SignB = 0 $$
             $$
                    \begin{array}{rcr}
                           X_A         = & \PLUS  & 1000001         \\
                           X_B         = &        & 1000000 \nl
                           X_A + X_B   = & \MINUS & 10000001        \\
                           d           = &        & 1000000     \nl
                           X_C         = &        & 0000001
                    \end{array}
             $$
             $$ P_C = 1 $$
             $$\begin{array}{c} \\ 
 \\ \begin{array}{|c|c|c|c|} \hline \textup{N\ шага} & \textup{Действие} & \textup{Делимое} & \textup{Частное} \\ \hline 
\begin{array}{c}$$0$$ \\ $$$$ \\ $$$$\end{array} & \begin{array}{c}$$Ma$$ \\ $$[-M_b]доп$$ \\ $$R0$$\end{array} & \begin{array}{c}$$010010110$$ \\ $$100100111$$ \\ $$110111101$$\end{array} & \begin{array}{c}$$00000000$$ \\ $$$$ \\ $$00000000$$\end{array} \\ \hline 
\begin{array}{c}$$1$$ \\ $$$$ \\ $$$$\end{array} & \begin{array}{c}$$\leftarrow R0$$ \\ $$[M_b]пр$$ \\ $$R1$$\end{array} & \begin{array}{c}$$101111010$$ \\ $$011011001$$ \\ $$001010011$$\end{array} & \begin{array}{c}$$00000000$$ \\ $$$$ \\ $$00000001$$\end{array} \\ \hline 
\begin{array}{c}$$2$$ \\ $$$$ \\ $$$$\end{array} & \begin{array}{c}$$\leftarrow R1$$ \\ $$[-M_b]доп$$ \\ $$R2$$\end{array} & \begin{array}{c}$$010100110$$ \\ $$100100111$$ \\ $$111001101$$\end{array} & \begin{array}{c}$$00000010$$ \\ $$$$ \\ $$00000010$$\end{array} \\ \hline 
\begin{array}{c}$$3$$ \\ $$$$ \\ $$$$\end{array} & \begin{array}{c}$$\leftarrow R2$$ \\ $$[M_b]пр$$ \\ $$R3$$\end{array} & \begin{array}{c}$$110011010$$ \\ $$011011001$$ \\ $$001110011$$\end{array} & \begin{array}{c}$$00000100$$ \\ $$$$ \\ $$00000101$$\end{array} \\ \hline 
\begin{array}{c}$$4$$ \\ $$$$ \\ $$$$\end{array} & \begin{array}{c}$$\leftarrow R3$$ \\ $$[-M_b]доп$$ \\ $$R4$$\end{array} & \begin{array}{c}$$011100110$$ \\ $$100100111$$ \\ $$000001101$$\end{array} & \begin{array}{c}$$00001010$$ \\ $$$$ \\ $$00001011$$\end{array} \\ \hline 
\begin{array}{c}$$5$$ \\ $$$$ \\ $$$$\end{array} & \begin{array}{c}$$\leftarrow R4$$ \\ $$[-M_b]доп$$ \\ $$R5$$\end{array} & \begin{array}{c}$$000011010$$ \\ $$100100111$$ \\ $$101000001$$\end{array} & \begin{array}{c}$$00010110$$ \\ $$$$ \\ $$00010110$$\end{array} \\ \hline 
\begin{array}{c}$$6$$ \\ $$$$ \\ $$$$\end{array} & \begin{array}{c}$$\leftarrow R5$$ \\ $$[M_b]пр$$ \\ $$R6$$\end{array} & \begin{array}{c}$$010000010$$ \\ $$011011001$$ \\ $$101011011$$\end{array} & \begin{array}{c}$$00101100$$ \\ $$$$ \\ $$00101100$$\end{array} \\ \hline 
\begin{array}{c}$$7$$ \\ $$$$ \\ $$$$\end{array} & \begin{array}{c}$$\leftarrow R6$$ \\ $$[M_b]пр$$ \\ $$R7$$\end{array} & \begin{array}{c}$$010110110$$ \\ $$011011001$$ \\ $$110001111$$\end{array} & \begin{array}{c}$$01011000$$ \\ $$$$ \\ $$01011000$$\end{array} \\ \hline 
\begin{array}{c}$$8$$ \\ $$$$ \\ $$$$\end{array} & \begin{array}{c}$$\leftarrow R7$$ \\ $$[M_b]пр$$ \\ $$R8$$\end{array} & \begin{array}{c}$$100011110$$ \\ $$011011001$$ \\ $$111110111$$\end{array} & \begin{array}{c}$$10110000$$ \\ $$$$ \\ $$10110000$$\end{array} \\ \hline 
 \end{array} \\
 \\ 
 \\ \end{array}$$
             $$
                    \begin{array}{c}
                           C^* = 0.A1_{16} * 16^1 = A.1_{16}  = 0.6289\\
                           \Delta C = C_T - C^* = 0.6319 - 0.6289 = 0.003 \\
                           \delta C = \left|\frac{\Delta C}{C_T}\right| \cdot 100\% = \left|\frac{0.003}{0.6319}\right| \cdot 100\% = 0.4 \\
                    \end{array}
             $$
       \item Формат Ф2
              $$ A = 8,9_{10} = 1000.1110011_{2} = 0.100011100110_{2} * 2^4 $$
              $$
                     \begin{array}{|c|c|c|}
                            \hline        
                            0 & 10000100 & 00011100110 \nl
                     \end{array}
              $$
              $$ A = 0,071_{10} = 0.0001001000101101_{2} = 0.100100010110_{2} * 2^{-3} $$
              $$
                     \begin{array}{|c|c|c|}
                            \hline        
                            0 & 10000000 & 00100010110 \nl
                     \end{array}
              $$
              $$ SignC = SignA \oplus SignB = 0 $$
              $$ P_C = 7 $$
              
  $$\begin{array}{c}\begin{array}{|c|c|c|c|} \hline \begin{array}{c}\textup{Оперaнды}\end{array} & \begin{array}{c}\textup{СЧП\ (стaршие\ рaзряды)}\end{array} & \begin{array}{c}\textup{\ В/СЧП\ (млaдшие\ рaзряды)}\end{array} & \begin{array}{c}\textup{Признaк\ коррекции}\end{array} \\ \hline 
\textup{СЧП} & 00000000000000000 & 100100010110 & 0 \\ \hline 
\begin{array}{c} \left[4A \right]_{\textup{}} \\  \left[+2A\right]_{\textup{}} \\ \textup{СЧП} \\ \textup{СЧП\ \rightarrow\ 4}\end{array} & \begin{array}{c}00010001110011000 \\ 00001000111001100 \\ 00011010101100100 \\ 00000001101010110\end{array} & \begin{array}{c} \\  \\ 100100010110 \\ 000000000011\end{array} & 0 \\ \hline 
\begin{array}{c} \left[0A \right]_{\textup{}} \\  \left[+1A\right]_{\textup{}} \\ \textup{СЧП} \\ \textup{СЧП\ \rightarrow\ 4}\end{array} & \begin{array}{c}00000000000000000 \\ 00000100011100110 \\ 00000110000111100 \\ 00000000011000011\end{array} & \begin{array}{c} \\  \\ 010010010001 \\ 000000000000\end{array} & 0 \\ \hline 
\begin{array}{c} \left[8A \right]_{\textup{}} \\  \left[+A \right]_{\textup{}} \\ \textup{СЧП} \\ \textup{СЧП\ \rightarrow\ 4}\end{array} & \begin{array}{c}00100011100110000 \\ 00000100011100110 \\ 00101000011011001 \\ 00000010100001101\end{array} & \begin{array}{c} \\  \\ 110001001001 \\ 000000000101\end{array} & 0 \\ \hline 
 \end{array} \\
 \\ 
 \\ \end{array}$$

              $$
                     \begin{array}{c}
                            C^* = 0.10100001101_{2}  = 0.6289\\
                            \Delta C = C_T - C^* = 0.6319 - 0.63134765625 = 0.00055 \\
                            \delta C = \left|\frac{\Delta C}{C_T}\right| \cdot 100\% = \left|\frac{0.00055}{0.6319}\right| \cdot 100\% = 0.087 \\
                     \end{array}
              $$
\end{enumerate}
\end{document}
