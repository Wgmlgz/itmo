\documentclass{article}
\usepackage{import}
\import{../../../lib/latex/}{wgmlgz}

\begin{document}

\itmo[
  variant=111,
  labn=3,
  worktype=Домашняя работа,
  discipline=Дискретная математика,
  group=P3115,
  student=Владимир Мацюк,
  teacher=Поляков Владимир Иванович,
  logo=../../../lib/img/itmo.png
]

\newcommand{\car}{\multicolumn{1}{c@{\hspace*{\tabcolsep}\makebox[0pt]{\curvearrowleft}}}{}}
\newcommand{\rcar}{\multicolumn{1}{c@{\hspace*{\tabcolsep}\makebox[0pt]{\curvearrowright}}}{}}
\newcommand{\ncar}{\multicolumn{1}{c@{\hspace*{\tabcolsep}\makebox[0pt]{}}}{}}
\newcommand{\SPACE}{\multicolumn{12}{c}{}}
\newcommand{\INT}{\multicolumn{5}{c}{\MM{Интерпретации}}}
\newcommand{\PLUS}{\multirow{2}{*}{+}}
\newcommand{\MINUS}{\multirow{2}{*}{-}}
\newcommand{\SIGN}{\multicolumn{2}{c}{\MM{Знаковая}}}
\newcommand{\USIGN}{\multicolumn{2}{c}{\MM{Беззнаковая}}}

\section{Числа}
$$
  \begin{array}{|c|c|}
    \hline
    A & 73 \nl
    B & 48 \nl
  \end{array}
$$
\section{Задание}
\begin{enumerate}
  \item Для заданных чисел А и В выполнить операцию знакового вычитания со всеми комбинациями знаков операндов. Для каждого примера:
        \begin{enumerate}
          \item
          \item проставить межразрядные заёмы, возникающие при вычитании;
          \item дать знаковую интерпретацию операндов и результатов. При получении отрицательного результата предварительно преобразовать его из дополнительного кода в прямой;
          \item дать беззнаковую интерпретацию операндов и результатов, при получении неверного результата пояснить причину его возникновения;
          \item показать значения арифметических флагов.
        \end{enumerate}

        $$ A = 73_{10} = 1001001_2,\ B = 48_{10} = 00110000_2 $$

        $$ A > 0,\ B > 0 $$
        $$\begin{array}{ccc|cccccccccccccc}
            \SPACE & \INT                                                                                         \\
                   &              &   & \rcar & \rcar &   &   &   &   &   &  & \SIGN  &    & \USIGN               \\
            \MINUS & A_{\MM{пр.}} & 0 & 1     & 0     & 0 & 1 & 0 & 0 & 1 &  & \MINUS & 73 &        & \MINUS & 73 \\
                   & B_{\MM{пр.}} & 0 & 0     & 1     & 1 & 0 & 0 & 0 & 0 &  &        & 48 &        &        & 48 \\ \cline{2-2} \cline{5-10} \cline{12-13} \cline{15-16}
                   & C_{\MM{пр.}} & 0 & 0     & 0     & 1 & 1 & 0 & 0 & 1 &  &        & 25 &        &        & 25 \\
          \end{array}
        $$
        $$ CF=0,\ ZF=0,\ PF=0,\ AF=0,\ SF=0,\ OF=0,\ $$

        $$ A < 0,\ B > 0 $$
        $$\begin{array}{ccc|cccccccccccccc}
            \SPACE & \INT                                                                                    \\
                   &              &   &   &   &   &   &   &   &   &  & \SIGN  &      & \USIGN                \\
            \MINUS & A_{\MM{пр.}} & 1 & 0 & 1 & 1 & 0 & 1 & 1 & 1 &  & \MINUS & -73  &        & \MINUS & 183 \\
                   & B_{\MM{пр.}} & 0 & 0 & 1 & 1 & 0 & 0 & 0 & 0 &  &        & 48   &        &        & 48  \\ \cline{2-2} \cline{5-10} \cline{12-13} \cline{15-16}
                   & C_{\MM{пр.}} & 1 & 0 & 0 & 0 & 0 & 1 & 1 & 1 &  &        &      &        &        & 135 \\ \cline{2-2} \cline{5-10} \cline{12-13} \cline{15-16}
                   & C_{\MM{об.}} & 0 & 1 & 1 & 1 & 1 & 0 & 0 & 1 &  &        & -121 &        &        &     \\
          \end{array}
        $$
        $$ CF=0,\ ZF=0,\ PF=1,\ AF=0,\ SF=1,\ OF=0,\ $$

        $$ A > 0,\ B < 0 $$
        $$\begin{array}{ccc|cccccccccccccc}
            \SPACE & \INT                                                                                           \\
                   &              &   & \rcar & \rcar &   &   &   &   &   &  & \SIGN  &     & \USIGN                \\
            \MINUS & A_{\MM{пр.}} & 0 & 1     & 0     & 0 & 1 & 0 & 0 & 1 &  & \MINUS & 73  &        & \MINUS & 73  \\
                   & B_{\MM{пр.}} & 1 & 1     & 0     & 1 & 0 & 0 & 0 & 0 &  &        & -48 &        &        & 208 \\ \cline{2-2} \cline{5-10} \cline{12-13} \cline{15-16}
                   & C_{\MM{пр.}} & 0 & 1     & 1     & 1 & 1 & 0 & 0 & 1 &  &        & 121 &        &        & 121 \\
          \end{array}
        $$
        $$ CF=1,\ ZF=0,\ PF=0,\ AF=0,\ SF=0,\ OF=0,\ $$
        $$ A < 0,\ B < 0 $$
        $$\begin{array}{ccc|cccccccccccccc}
            \SPACE & \INT                                                                                       \\
                   &              & \rcar &   &   &   &   &   &   &   &  & \SIGN  &     & \USIGN                \\
            \MINUS & A_{\MM{пр.}} & 1     & 0 & 1 & 1 & 0 & 1 & 1 & 1 &  & \MINUS & -73 &        & \MINUS & 183 \\
                   & B_{\MM{пр.}} & 1     & 1 & 0 & 1 & 0 & 0 & 0 & 0 &  &        & -48 &        &        & 208 \\ \cline{2-2} \cline{5-10} \cline{12-13} \cline{15-16}
                   & C_{\MM{пр.}} & 1     & 1 & 1 & 0 & 0 & 1 & 1 & 1 &  &        &     &        &        & 231 \\ \cline{2-2} \cline{5-10} \cline{12-13} \cline{15-16}
                   & C_{\MM{об.}} & 0     & 0 & 0 & 1 & 1 & 0 & 0 & 1 &  &        & -25 &        &        &     \\
          \end{array}
        $$
        $$ CF=0,\ ZF=0,\ PF=1,\ AF=0,\ SF=1,\ OF=0,\ $$
        Для беззнаковой и знаковой интерпретации результат неверен вследствие возникающего заёма из разряда за пределами формата
  \item Cохранив значение первого операнда А, выбрать такое значение В, чтобы в операции вычитания с разными знаками имел место особый случай переполнения формата. Выполнить два примера, иллюстрирующие эти случаи, для каждого из них проделать пункты a, b, c, d.
        $$ A + B > 128,  \Rightarrow   128 – A  < B < 127 $$
        $$ A = 73,\ B = 64 $$


        $$\begin{array}{ccc|cccccccccccccc}
            \SPACE & \INT                                                                                            \\
                   &              & \rcar &   &   & \rcar &   &   &   &   &  & \SIGN  &      & \USIGN                \\
            \MINUS & A_{\MM{пр.}} & 1     & 0 & 1 & 1     & 0 & 1 & 1 & 1 &  & \MINUS & -73  &        & \MINUS & 183 \\
                   & B_{\MM{пр.}} & 0     & 1 & 0 & 0     & 0 & 0 & 0 & 0 &  &        & 64   &        &        & 64  \\ \cline{2-2} \cline{5-10} \cline{12-13} \cline{15-16}
                   & C_{\MM{пр.}} & 0     & 1 & 1 & 1     & 0 & 1 & 1 & 1 &  &        & -137 &        &        & 119 \\
          \end{array}
        $$
        $$ CF=0,\ ZF=0,\ PF=1,\ AF=0,\ SF=0,\ OF=1,\ $$
        Для знаковой интерпретации результат некорректен вследствие возникающего переполнения.
        $$\begin{array}{ccc|cccccccccccccc}
            \SPACE & \INT                                                                                       \\
                   &              & \ncar &   &   &   &   &   &   &   &  & \SIGN  &     & \USIGN                \\
            \MINUS & A_{\MM{пр.}} & 0     & 1 & 0 & 0 & 1 & 0 & 0 & 1 &  & \MINUS & 73  &        & \MINUS & 73  \\
                   & B_{\MM{пр.}} & 1     & 1 & 0 & 0 & 0 & 0 & 0 & 0 &  &        & -64 &        &        & 192 \\ \cline{2-2} \cline{5-10} \cline{12-13} \cline{15-16}
                   & C_{\MM{пр.}} & 1     & 0 & 0 & 0 & 1 & 0 & 0 & 1 &  &        &     &        &        & 137 \\ \cline{2-2} \cline{5-10} \cline{12-13} \cline{15-16}
                   & C_{\MM{об.}} & 0     & 1 & 1 & 1 & 0 & 1 & 1 & 1 &  &        & 137 &        &        &     \\
          \end{array}
        $$
        $$ CF=1,\ ZF=0,\ PF=0,\ AF=0,\ SF=1,\ OF=1,\ $$
        Для знаковой интерпретации результат некорректен вследствие возникающего переполнения, для беззнаковой интерпретации результат некорректен из-за возникающего заёма из старшего разряда.
  \item Сохранив операнд B, подобрать такое значение операнда A, чтобы при вычитании отрицательного B из положительного A имело место переполнение формата, а при вычитании положительного B из отрицательного A результат был бы корректен. Выполнить два примера, иллюстрирующие этот случай. Для каждого из них проделать пункты a, b, c, d.
        $$ A = 80,\ B = 48 $$

        $$\begin{array}{ccc|cccccccccccccc}
            \SPACE & \INT                                                                                        \\
                   &              & \ncar &   &   &   &   &   &   &   &  & \SIGN  &      & \USIGN                \\
            \MINUS & A_{\MM{пр.}} & 0     & 1 & 0 & 1 & 0 & 0 & 0 & 0 &  & \MINUS & 80   &        & \MINUS & 80  \\
                   & B_{\MM{пр.}} & 1     & 1 & 0 & 1 & 0 & 0 & 0 & 0 &  &        & -48  &        &        & 208 \\ \cline{2-2} \cline{5-10} \cline{12-13} \cline{15-16}
                   & C_{\MM{пр.}} & 1     & 0 & 0 & 0 & 0 & 0 & 0 & 0 &  &        &      &        &        & 128 \\ \cline{2-2} \cline{5-10} \cline{12-13} \cline{15-16}
                   & C_{\MM{об.}} & 1     & 0 & 0 & 0 & 0 & 0 & 0 & 0 &  &        & -128 &        &        &     \\
          \end{array}
        $$
        $$ CF=1,\ ZF=0,\ PF=0,\ AF=0,\ SF=1,\ OF=1,\ $$
        Результат беззнаковой интерпретации некорректен вследствие возникающего заёма из разряда за пределами формата, для знаковой интерпретации результат некорректен из-за переполнения.

        $$\begin{array}{ccc|cccccccccccccc}
            \SPACE & \INT                                                                                        \\
                   &              & \rcar &   &   &   &   &   &   &   &  & \SIGN  &      & \USIGN                \\
            \MINUS & A_{\MM{пр.}} & 1     & 0 & 1 & 1 & 0 & 0 & 0 & 0 &  & \MINUS & -80  &        & \MINUS & 176 \\
                   & B_{\MM{пр.}} & 0     & 0 & 1 & 1 & 0 & 0 & 0 & 0 &  &        & 48   &        &        & 48  \\ \cline{2-2} \cline{5-10} \cline{12-13} \cline{15-16}
                   & C_{\MM{пр.}} & 1     & 0 & 0 & 0 & 0 & 0 & 0 & 0 &  &        &      &        &        & 128 \\ \cline{2-2} \cline{5-10} \cline{12-13} \cline{15-16}
                   & C_{\MM{об.}} & 1     & 0 & 0 & 0 & 0 & 0 & 0 & 0 &  &        & -128 &        &        &     \\
          \end{array}
        $$
        $$ CF=0,\ ZF=0,\ PF=0,\ AF=0,\ SF=1,\ OF=0,\ $$

        Результаты знаковой и беззнаковой интерпретаций корректны.

\end{enumerate}
\end{document}
