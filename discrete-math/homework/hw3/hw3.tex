\documentclass{article}
\usepackage{import}
\import{../../../lib/latex/}{wgmlgz}

\begin{document}

\itmo[
  variant=111,
  labn=3,
  worktype=Домашняя работа,
  discipline=Дискретная математика,
  group=P3115,
  student=Владимир Мацюк,
  teacher=Поляков Владимир Иванович,
  logo=../../../lib/img/itmo.png
]

\section{Числа}
$$
  \begin{array}{|c|c|}
    \hline
    A & 73 \nl
    B & 48 \nl
  \end{array}
$$
\section{Задание}
\begin{enumerate}
  \item Для заданных чисел А и В выполнить операцию знакового вычитания со всеми комбинациями знаков операндов. Для каждого примера:
        \begin{enumerate}
          \item
          \item проставить межразрядные заёмы, возникающие при вычитании;
          \item дать знаковую интерпретацию операндов и результатов. При получении отрицательного результата предварительно преобразовать его из дополнительного кода в прямой;
          \item дать беззнаковую интерпретацию операндов и результатов, при получении неверного результата пояснить причину его возникновения;
          \item показать значения арифметических флагов.
        \end{enumerate}
  \item Cохранив значение первого операнда А, выбрать такое значение В, чтобы в операции вычитания с разными знаками имел место особый случай переполнения формата. Выполнить два примера, иллюстрирующие эти случаи, для каждого из них проделать пункты a, b, c, d.
  \item Сохранив операнд B, подобрать такое значение операнда A, чтобы при вычитании отрицательного B из положительного A имело место переполнение формата, а при вычитании положительного B из отрицательного A результат был бы корректен. Выполнить два примера, иллюстрирующие этот случай. Для каждого из них проделать пункты a, b, c, d.
\end{enumerate}
\end{document}
