\documentclass{article}
\usepackage{import}
\import{../../../lib/latex/}{wgmlgz}

\begin{document}

\itmo[
      variant=112,
      labn=1,
      worktype=Домашняя работа,
      discipline=Дискретная математика,
      group=P3115,
      student=Владимир Мацюк,
      teacher=Поляков Владимир Иванович,
      logo=../../../lib/img/itmo.png
]

\section{Числа}
$$
      \begin{tabular}{|c|c|}
            \hline
            A & 1975            \\
            \hline
            B & 0,84            \\
            \hline
            R & $\MM{C3598400}$ \\
            \hline
            S & $\MM{3FCB0000}$ \\
            \hline
      \end{tabular}
$$
\section{Задание}
\begin{enumerate}
      \item Заданное число А представить в виде двоично-кодированного десятичного числа:
            \begin{enumerate}
                  \item в упакованном формате (BCD)
                  \item в неупакованном формате (ASCII).
            \end{enumerate}
            
            Ответ: \begin{enumerate}
                  \item \begin{tabular}{|c|c|c|c|}
                              \hline
                              0001 & 1001 & 0111 & 0101 \\
                              \hline
                              1    & 9    & 7    & 5    \\ 
                              \hline
                        \end{tabular}
                  \item \begin{tabular}{|c|c|c|c|}
                              \hline
                              0011.0001 & 0011.1001 & 0011.0111 & 0011.0101 \\
                              \hline
                              1         & 9         & 7         & 5         \\ 
                              \hline
                        \end{tabular}
            \end{enumerate}
            
            
      \item Заданное число А и –A представить в форме с фиксированной запятой.
            $$ 1975_{10} = 11110110111_2 $$
            $$ \begin{tabular}{c r}
                        $[-A]_{\MM{пр}}$  & 0000.0111.1011.0111 \\
                        $[-A]_{\MM{об}}$  & 1111.1000.0100.1000 \\
                                          & +1                  \\
                        \hline
                        $[-A]_{\MM{доп}}$ & 1111.1000.0100.1001 \\
                  \end{tabular}$$
            Oтвет: $ A = 0000.0111.1011.0111, -A = 1111.1000.0100.1001$
      \item Заданные числа A и B представить в форме с плавающей запятой в формате Ф1.
            $$ 
                  A =
                  1975_{10} =
                  \texttt{7B7}_{16} =
                  11110110111_2 = 
                  \texttt{0,7B7}_{16} \cdot 16^3 $$
            $$ X_A = P_A + d = 3 + 64 = 67_{10} = 1000011_2 $$
            Ответ: \begin{tabular}{|c|l r|l c c c c r|}
                  \hline
                  0 & 100 & 0011 & 1111 & 0110 & 1110 & 0000 & 0000 & 0000 \\
                  \hline
                  0 & 1   & 7    & 8    &      &      &      &      & 31   \\ 
                  \hline
            \end{tabular}
            $$ 
                  B =
                  0,84_{10} =
                  \texttt{0.D70A3D70A3D71}_{16} =
                  \texttt{0.110101110000101000111101}_{2} =
                  \texttt{0.D70A3D70A3D71}_{16} \cdot 16^0 $$
            $$ X_B = P_B + d = 0 + 64 = 64{10} = 1000000_2 $$
            Ответ: \begin{tabular}{|c|l r|l c c c c r|}
                  \hline
                  0 & 100 & 0000 & 1101 & 0111 & 0000 & 1010 & 0011 & 1101 \\
                  \hline
                  0 & 1   & 7    & 8    &      &      &      &      & 31   \\ 
                  \hline
            \end{tabular}
      \item Заданные числа A и B представить в форме с плавающей запятой в формате Ф2.
            $$ 
                  A =
                  1975_{10} =
                  11110110111_2 = 
                  11110110111_2 \cdot 2^{11} $$
            $$ X_A = P_A + d = 11 + 128 = 139_{10} = 10001011_2 $$
            Ответ: \begin{tabular}{|c|c|c|}
                  \hline
                  0  & 10001011 & 11110110111000000000000 \\
                  \hline
                  31 & $30-23$  & $22-0$                  \\ 
                  \hline
            \end{tabular}
            $$ 
                  B =
                  0,84_{10} =
                  0.110101110000101000111101_{2} =
                  0.110101110000101000111101_{2} \cdot 2^0 $$
            $$ X_B = P_B + d = 0 + 128 = 128_{10} = 10000000_2 $$
            Ответ: \begin{tabular}{|c|c|c|}
                  \hline
                  0  & 10000000 & 11010111000010100011110 \\
                  \hline
                  31 & $30-23$  & $22-0$                  \\ 
                  \hline
            \end{tabular}
      \item Заданные числа A и B представить в форме с плавающей запятой в формате Ф3.
            $$ 
                  A =
                  1975_{10} =
                  11110110111_2 = 
                  1,1110110111_2 \cdot 2^{10} $$
            $$ X_A = P_A + d = 10 + 127 = 137_{10} = 10001001_2 $$
            Ответ: \begin{tabular}{|c|c|c|}
                  \hline
                  0  & 10001001 & 11101101110000000000000 \\
                  \hline
                  31 & $30-23$  & $22-0$                  \\ 
                  \hline
            \end{tabular}
            $$ 
                  B =
                  0,84_{10} =
                  0.110101110000101000111101_{2} =
                  1.10101110000101000111101_{2} \cdot 2^{-1} $$
            $$ X_B = P_B + d = -1 + 127 = 126_{10} = 01111110_2 $$
            Ответ: \begin{tabular}{|c|c|c|}
                  \hline
                  0  & 01111110 & 10101110000101000111101 \\
                  \hline
                  31 & $30-23$  & $22-0$                  \\ 
                  \hline
            \end{tabular}
      \item Найти значения чисел Y и Z по их заданным шестнадцатеричным представлениям R и S в форме с плавающей запятой в формате Ф1.
            $$ 
                  \texttt{C3598400}_{16} =
                  1.1000011.010110011000010000000000_2
            $$
            $$ X_Y = P_Y + d = 67 = 64 + 3 $$
            $$ Y = -1 * \texttt{0.598400}_{16} * 16^3 = -1432.25 $$
            Ответ: -1432.25
            
            $$ 
                  \texttt{3FCB0000}_{16} =
                  0.0111111.110010110000000000000000_2
            $$
            $$ X_Z = P_Z + d = 63 = 64 + (-1) $$
            $$ Z = \texttt{0.CB0000}_{16} * 16^{(-1)} = 0.049560546875 $$
            Ответ: 0.049560546875
      \item Найти значения чисел V и W по их заданным шестнадцатеричным представлениям R и S в форме с плавающей запятой в формате Ф2.
            $$ 
                  \texttt{C3598400}_{16} =
                  1.10000110.10110011000010000000000_2
            $$
            $$ X_V = P_V + d = 134 = 134 + 6 $$
            $$ V = -1 * 0.1011001100001_{2} * 2^6 = -44.7578125 $$
            Ответ: -44.7578125
            
            $$ 
                  \texttt{3FCB0000}_{16} =
                  0.01111111.10010110000000000000000_2
            $$
            $$ X_W = P_W + d = 127 = 128 + (-1) $$
            $$ W = 0.1001011_{2} * 2^{-1} = 0.29296875 $$
            Ответ: 0.29296875
      \item Найти значения чисел T и Q по их заданным шестнадцатеричным представлениям R и S в форме с плавающей запятой в формате Ф3.
            $$ 
                  \texttt{C3598400}_{16} =
                  1.10000110.10110011000010000000000_2
            $$
            $$ X_T = P_T - 127 = 134 - 127 = 7 $$
            $$ T = -1 * 1.011001100001_{2} * 2^7 = -179.03125 $$
            Ответ: -179.03125
            
            $$ 
                  \texttt{3FCB0000}_{16} =
                  0.01111111.10010110000000000000000_2
            $$
            $$ X_Q = P_Q - 127 = 127 - 127 = 0 $$
            $$ Q = 1.001011_{2} * 2^0 = 1.171875 $$
            Ответ: 1.171875
\end{enumerate}
\end{document}
