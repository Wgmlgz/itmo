\documentclass{article}
\usepackage{import}
\import{../../../lib/latex/}{wgmlgz}

\begin{document}

\itmo[
       variant=112,
       labn=1,
       worktype=Курсовая работа часть,
       discipline=Дискретная математика,
       group=P3115,
       student=Владимир Мацюк,
       teacher=Поляков Владимир Иванович,
       logo=../../../lib/img/itmo.png
]

\newcommand{\car}{\multicolumn{1}{c@{\hspace*{\tabcolsep}\makebox[0pt]{\curvearrowleft}}}{}}
\newcommand{\rcar}{\multicolumn{1}{c@{\hspace*{\tabcolsep}\makebox[0pt]{\curvearrowright}}}{}}
\newcommand{\ncar}{\multicolumn{1}{c@{\hspace*{\tabcolsep}\makebox[0pt]{}}}{}}
\newcommand{\SPACE}{\multicolumn{12}{c}{}}
\newcommand{\INT}{\multicolumn{5}{c}{\MM{Интерпретации}}}
\newcommand{\PLUS}{\multirow{2}{*}{+}}
\newcommand{\MINUS}{\multirow{2}{*}{-}}
\newcommand{\SIGN}{\multicolumn{2}{c}{\MM{Знаковая}}}
\newcommand{\USIGN}{\multicolumn{2}{c}{\MM{Беззнаковая}}}

\section{Вариант}
$$
       \begin{tabular}{|c|c|}
              \hline
              Условия, при которых $f=1$ & $3 \le |x_4 1 x_5 - x_1x_2x_3| < 6$ \nl
              Условия, при которых $f=d$ & $|x_4 1 x_5 - x_1x_2x_3| = 0$ \nl
       \end{tabular}
$$
\section{Задание}

\begin{enumerate}
       \item Составить таблицу истинности заданной булевой функции.
       \item Представить булеву функцию в аналитическом виде с помощью КДНФ и ККНФ.
       \item Найти МДНФ и/или МКНФ методом Квайна – Мак-Класки.
       \item Найти МДНФ и МКНФ на картах Карно.
       \item Преобразовать МДНФ и МКНФ к форме, обеспечивающей минимум цены схемы.
       \item По полученной форме построить комбинационную схему в булевом базисе. Определить задержку схемы.
       \item Построить схемы с минимальной ценой в универсальных базисах и сокращенных булевых базисах. Определить задержку каждой из схем.
       \item Построить схему в базисе Жегалкина. Определить цену и задержку.
       \item Построить схему в универсальном базисе с учетом заданного коэффициента объединения по входам. Определить цену и задержку схемы.
       \item Выполнить анализ построенных схем, определив их реакцию на заданные комбинации входных сигналов.
\end{enumerate}
\section{Решение}
\end{document}
