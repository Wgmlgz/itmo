% \usepackage{draculatheme}

% \usepackage{colortbl}
% \usepackage[usenames]{color}


\usepackage[paper=letterpaper,margin=2cm]{geometry}
\usepackage[utf8]{inputenc}
\usepackage[russian]{babel}
\usepackage{graphicx}
\usepackage{geometry}
\usepackage{xcolor}
\usepackage{hyperref}
\usepackage{fontspec}
% \usepackage{listings-rust}
\usepackage{listings}
\usepackage{keycommand}
\usepackage{caption}
\usepackage{dingbat}
\usepackage{array,multirow}
\usepackage{amssymb}
\usepackage{spreadtab}
\usepackage{unicode-math}

% \setmonofont{JetBrains Mono}[Contextuals=Alternate,Ligatures = TeX,]
\setmainfont[
  Ligatures=TeX,
  Extension=.otf,
  BoldFont=cmunbx,
  ItalicFont=cmunti,
  BoldItalicFont=cmunbi,
]{cmunrm}
\setsansfont[
  Ligatures=TeX,
  Extension=.otf,
  BoldFont=cmunsx,
  ItalicFont=cmunsi,
]{cmunss}

% \geometry{
%   a4paper,
%   top=25mm,
%   right=0mm,
%   bottom=25mm,
%   left=0mm
% }

\hypersetup{
  colorlinks=true,
  linkcolor=blue!50!red,
  urlcolor=blue!70!black
}

\captionsetup[lstlisting]{
  font={tt},
}

% based on Atom One Light
\lstset{
  frame=single,
  basicstyle=\ttfamily\color[HTML]{383a42},
  columns=fullflexible,
  breaklines=true,
  numbers=left,
  frame=tab,
  postbreak=\mbox{\textcolor{red}{$\hookrightarrow$}\space},
  extendedchars=false,
  showspaces=false,
  showstringspaces=false,
  identifierstyle=\ttfamily\color[HTML]{4078f2},
  commentstyle=\color[HTML]{a0a1a7},
  stringstyle=\color[HTML]{50a14f},
  keywordstyle=\color[HTML]{a626a4},
  numberstyle=\ttfamily\color[HTML]{2c91af},
  rulecolor=\color[HTML]{383a42}
}

\lstdefinelanguage{XML}
{
  morestring=[b]",
  morestring=[s]{>}{<},
  morecomment=[s]{<?}{?>},
}

\newcommand{\code}[1]{
  \lstset{title=#1}
  \lstinputlisting{#1}
}

\newcommand{\MM}[1]{\mathchoice{\hbox{#1}}{\hbox{#1}}{\hbox{\scriptsize{#1}}}{\hbox{\tiny{#1}}}}
\newcommand*\BitAnd{\mathbin{\&}}
\newcommand*\BitOr{\mathbin{|}}
\newcommand*\ShiftLeft{\ll}
\newcommand*\ShiftRight{\gg}
\newcommand*\BitNeg{\ensuremath{\mathord{\sim}}}
\def\nl{\\\hline}

\newkeycommand{\itmo}[
  variant=aboba,
  labn=aboba,
  worktype=Лабораторная работа,
  discipline=aboba,
  group=aboba,
  student=aboba,
  teacher=aboba,
  year=2023,
  logo=../../lib/img/itmo.png
]{
  \begin{titlepage}
    \begin{center}
      \section*{
        Федеральное государственное автономное образовательное учреждение\\ высшего образования\\
        «Национальный исследовательский университет ИТМО»\\
        Факультет Программной Инженерии и Компьютерной Техники \\
       }
      \includegraphics[scale=0.2]{\commandkey{logo}}
    \end{center}
    
    \vspace{4cm}
    
    \begin{center}
      \large \textbf{Вариант \textnumero \commandkey{variant}}\\
      \textbf{\commandkey{worktype} \textnumero \commandkey{labn}}\\
      по дисциплине\\
      \textbf{\commandkey{discipline}}
    \end{center}
    
    \vspace*{\fill}
    
    \begin{flushright}
      Выполнил Студент группы \commandkey{group}\\
      \textbf{\commandkey{student}}\\
      Преподаватель: \\
      \textbf{\commandkey{teacher}}\\
    \end{flushright}
    
    \vspace{1cm}
    
    \begin{center}
      Санкт-Петербург\\
      \commandkey{year}г.
    \end{center}
    
    \thispagestyle{empty}
  \end{titlepage}
}


\begin{document}

\itmo[
      variant=18,
      labn=4,
      discipline=Информатика,
      group=P3115,
      student=Владимир Мацюк,
      teacher=Малышева Татьяна Алексеевна
]

\tableofcontents

\section{Задание}
\begin{enumerate}
      \item Определить номер варианта как остаток деления на 36 порядкового
            номера в списке группы в ISU. В случае, если в данный день недели
            нет занятий, то увеличить номер варианта на восемь.
      \item  Изучить форму Бэкуса-Наура.
      \item  Изучить особенности языков разметки/форматов JSON, YAML, XML.
      \item  Понять устройство страницы с расписанием для своей группы:
            \url{http://itmo.ru/ru/schedule/0/P3110/schedule.htm}
      \item  Исходя из структуры расписания конкретного дня, сформировать
            файл с расписанием в формате, указанном в задании в качестве
            исходного. При этом необходимо, чтобы в выбранном дне было не
            менее двух занятий (можно использовать своё персональное). В
            случае, если в данный день недели нет таких занятий, то увеличить
            номер варианта ещё на восемь.
      \item  Обязательное задание (позволяет набрать до 65 процентов от
            максимального числа баллов БаРС за данную лабораторную):
            написать программу на языке Python 3.x, которая бы осуществляла
            парсинг и конвертацию исходного файла в новый.
      \item  Нельзя использовать готовые библиотеки, в том числе регулярные
            выражения в Python и библиотеки для загрузки XML-файлов.
      \item  Дополнительное задание задание No1 (позволяет набрать +10
            процентов от максимального числа баллов БаРС за данную
            лабораторную).
            \begin{enumerate}
                  \item Найти готовые библиотеки, осуществляющие аналогичный
                        парсинг и конвертацию файлов.
                  \item Переписать исходный код, применив найденные
                        библиотеки. Регулярные выражения также нельзя
                        использовать.
                  \item Сравнить полученные результаты и объяснить их
                        сходство/различие.
            \end{enumerate}
      \item Дополнительное задание задание No2 (позволяет набрать +10
            процентов от максимального числа баллов БаРС за данную
            лабораторную).
            \begin{enumerate}
                  \item Переписать исходный код, добавив в него использование
                        2
                        регулярных выражений.
                  \item Сравнить полученные результаты и объяснить их
                        сходство/различие.
                  \item Дополнительное задание задание No3 (позволяет набрать +10
                        процентов от максимального числа баллов БаРС за данную
                        лабораторную).
                  \item Используя свою исходную программу из обязательного
                        задания, программу из дополнительного задания No1 и
                        программу из дополнительного задания No2, сравнить
                        стократное время выполнения парсинга + конвертации в
                        цикле.
                  \item Проанализировать полученные результаты и объяснить их
                        сходство/различие.
            \end{enumerate}
      \item  Дополнительное задание задание No4 (позволяет набрать +5
            процентов от максимального числа баллов БаРС за данную
            лабораторную).
            \begin{enumerate}
                  \item  Переписать исходную программу, чтобы она осуществляла
                        парсинг и конвертацию исходного файла в любой другой
                        формат (кроме JSON, YAML, XML, HTML): PROTOBUF,
                        TSV, CSV, WML и т.п.
                  \item  Проанализировать полученные результаты, объяснить
                        особенности использования формата.
            \end{enumerate}
      \item  Проверить, что все пункты задания выполнены и выполнены верно.
      \item  Написать отчёт о проделанной работе.
      \item  Подготовиться к устным вопросам на защите
\end{enumerate}

\section{Вариант}
$$
      \begin{tabular}{|c|c|c|c|c|c|c|c|c|}
            \hline
            18 & JSON & XML & Четверг \\
            \hline
      \end{tabular}
$$
\section{Основные этапы выполнения}
\lstinputlisting{input.json}
\section{Вывод}
sdfsdf

\section*{Вывод}
Во время выполнения работы я ознакомился с ООП на языке java. Научился разрабатывать архитектуру проекта и подключать jar архивы к проекту.
\end{document}
