\documentclass{article}
\usepackage{import}
% \usepackage{draculatheme}

% \usepackage{colortbl}
% \usepackage[usenames]{color}


\usepackage[paper=letterpaper,margin=2cm]{geometry}
\usepackage[utf8]{inputenc}
\usepackage[russian]{babel}
\usepackage{graphicx}
\usepackage{geometry}
\usepackage{xcolor}
\usepackage{hyperref}
\usepackage{fontspec}
% \usepackage{listings-rust}
\usepackage{listings}
\usepackage{keycommand}
\usepackage{caption}
\usepackage{dingbat}
\usepackage{array,multirow}
\usepackage{amssymb}
\usepackage{spreadtab}
\usepackage{unicode-math}

% \setmonofont{JetBrains Mono}[Contextuals=Alternate,Ligatures = TeX,]
\setmainfont[
  Ligatures=TeX,
  Extension=.otf,
  BoldFont=cmunbx,
  ItalicFont=cmunti,
  BoldItalicFont=cmunbi,
]{cmunrm}
\setsansfont[
  Ligatures=TeX,
  Extension=.otf,
  BoldFont=cmunsx,
  ItalicFont=cmunsi,
]{cmunss}

% \geometry{
%   a4paper,
%   top=25mm,
%   right=0mm,
%   bottom=25mm,
%   left=0mm
% }

\hypersetup{
  colorlinks=true,
  linkcolor=blue!50!red,
  urlcolor=blue!70!black
}

\captionsetup[lstlisting]{
  font={tt},
}

% based on Atom One Light
\lstset{
  frame=single,
  basicstyle=\ttfamily\color[HTML]{383a42},
  columns=fullflexible,
  breaklines=true,
  numbers=left,
  frame=tab,
  postbreak=\mbox{\textcolor{red}{$\hookrightarrow$}\space},
  extendedchars=false,
  showspaces=false,
  showstringspaces=false,
  identifierstyle=\ttfamily\color[HTML]{4078f2},
  commentstyle=\color[HTML]{a0a1a7},
  stringstyle=\color[HTML]{50a14f},
  keywordstyle=\color[HTML]{a626a4},
  numberstyle=\ttfamily\color[HTML]{2c91af},
  rulecolor=\color[HTML]{383a42}
}

\lstdefinelanguage{XML}
{
  morestring=[b]",
  morestring=[s]{>}{<},
  morecomment=[s]{<?}{?>},
}

\newcommand{\code}[1]{
  \lstset{title=#1}
  \lstinputlisting{#1}
}

\newcommand{\MM}[1]{\mathchoice{\hbox{#1}}{\hbox{#1}}{\hbox{\scriptsize{#1}}}{\hbox{\tiny{#1}}}}
\newcommand*\BitAnd{\mathbin{\&}}
\newcommand*\BitOr{\mathbin{|}}
\newcommand*\ShiftLeft{\ll}
\newcommand*\ShiftRight{\gg}
\newcommand*\BitNeg{\ensuremath{\mathord{\sim}}}
\def\nl{\\\hline}

\newkeycommand{\itmo}[
  variant=aboba,
  labn=aboba,
  worktype=Лабораторная работа,
  discipline=aboba,
  group=aboba,
  student=aboba,
  teacher=aboba,
  year=2023,
  logo=../../lib/img/itmo.png
]{
  \begin{titlepage}
    \begin{center}
      \section*{
        Федеральное государственное автономное образовательное учреждение\\ высшего образования\\
        «Национальный исследовательский университет ИТМО»\\
        Факультет Программной Инженерии и Компьютерной Техники \\
       }
      \includegraphics[scale=0.2]{\commandkey{logo}}
    \end{center}
    
    \vspace{4cm}
    
    \begin{center}
      \large \textbf{Вариант \textnumero \commandkey{variant}}\\
      \textbf{\commandkey{worktype} \textnumero \commandkey{labn}}\\
      по дисциплине\\
      \textbf{\commandkey{discipline}}
    \end{center}
    
    \vspace*{\fill}
    
    \begin{flushright}
      Выполнил Студент группы \commandkey{group}\\
      \textbf{\commandkey{student}}\\
      Преподаватель: \\
      \textbf{\commandkey{teacher}}\\
    \end{flushright}
    
    \vspace{1cm}
    
    \begin{center}
      Санкт-Петербург\\
      \commandkey{year}г.
    \end{center}
    
    \thispagestyle{empty}
  \end{titlepage}
}



\begin{document}

\itmo[
  variant=371371,
  labn=1,
  discipline=Веб-программирование,
  group=P3215,
  student=Владимир Мацюк,
  teacher=Кустарев Иван Павлович,
  year=2023,
  logo=../../lib/img/itmo.png
]

\section{Текст задания}
Разработать PHP-скрипт, определяющий попадание точки на координатной плоскости в заданную область, и создать HTML-страницу, которая формирует данные для отправки их на обработку этому скрипту.

Параметр R и координаты точки должны передаваться скрипту посредством HTTP-запроса. Скрипт должен выполнять валидацию данных и возвращать HTML-страницу с таблицей, содержащей полученные параметры и результат вычислений - факт попадания или непопадания точки в область. Предыдущие результаты должны сохраняться между запросами и отображаться в таблице.

Кроме того, ответ должен содержать данные о текущем времени и времени работы скрипта.

Разработанная HTML-страница должна удовлетворять следующим требованиям:

\begin{enumerate}
  \item Для расположения текстовых и графических элементов необходимо использовать блочную верстку.
  \item Данные формы должны передаваться на обработку посредством POST-запроса.
  \item Таблицы стилей должны располагаться в отдельных файлах.
  \item При работе с CSS должно быть продемонстрировано использование селекторов элементов, селекторов идентификаторов, селекторов потомств, селекторов классов а также такие свойства стилей CSS, как наследование и каскадирование.
  \item HTML-страница должна иметь "шапку", содержащую ФИО студента, номер группы и новер варианта. При оформлении шапки необходимо явным образом задать шрифт (sans-serif), его цвет и размер в каскадной таблице стилей.
  \item Отступы элементов ввода должны задаваться в процентах.
  \item Страница должна содержать сценарий на языке JavaScript, осуществляющий валидацию значений, вводимых пользователем в поля формы. Любые некорректные значения (например, буквы в координатах точки или отрицательный радиус) должны блокироваться.
\end{enumerate}

\begin{center}
  \includegraphics{task.png}
\end{center}

\section{Код программы}


\url{https://github.com/Wgmlgz/itmo/tree/main/web/l1}
\section{Вывод}
Во время выполнения лабораторной работы я познал php.
\end{document}
