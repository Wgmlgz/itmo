\documentclass{article}
% \usepackage{../../lib/latex/draculatheme}
\usepackage[utf8]{inputenc}
\usepackage[russian]{babel}
\usepackage{fontspec}
\usepackage{tikz}
\usepackage{xfrac}
\usepackage{amsmath}
\usepackage{wrapfig}
\usepackage{geometry}
\makeatletter
\newcommand*{\rom}[1]{\expandafter\@slowromancap\romannumeral #1@}
\makeatother

\geometry{
  a4paper,
 }
\setmainfont[
  Ligatures=TeX,
  Extension=.otf,
  BoldFont=cmunbx,
  ItalicFont=cmunti,
  BoldItalicFont=cmunbi,
]{cmunrm}
\setsansfont[
  Ligatures=TeX,
  Extension=.otf,
  BoldFont=cmunsx,
  ItalicFont=cmunsi,
]{cmunss}

\begin{document}
\section*{Задание 2. Координаты вектора в базисе}
\subsection*{L – пространство матриц второго порядка}
\subsubsection*{Докажите, что система $\mathcal{A}$ является базисом в соответствующем линейном пространстве $\mathcal{L}$.}
\begin{center}
  $$ \mathcal{A} = \{A_1, A_2, A_3, A_4\},\ x$$
  $$
    A_1=\left(\begin{array}{ll}
        2 & 2 \\
        3 & 1
      \end{array}\right), A_2=\left(\begin{array}{cc}
        -2 & 3 \\
        4  & 3
      \end{array}\right), A_3=\left(\begin{array}{cc}
        0  & 1 \\
        -1 & 1
      \end{array}\right), A_4=\left(\begin{array}{ll}
        1 & -3 \\
        2 & -1
      \end{array}\right) . \quad x=\left(\begin{array}{cc}
        -3 & -6 \\
        0  & 2
      \end{array}\right)
  $$
  Т.к. матрицы системы принадлежат пространству матриц второго порядка, поэтому для доказательства того, что эти матрицы образуют базис проверим, что они линейно независимы.
  $$
    \lambda_1,\ \lambda_2,\ \lambda_3,\ \lambda_4 \in R
  $$$$
    \lambda_1 A_1+\lambda_2 A_2+\lambda_3 A_3+\lambda_4 A_4=0
  $$$$
    \left(\begin{array}{cc}
        2 \lambda_1 & 2 \lambda_1 \\
        3 \lambda_1 & \lambda_1
      \end{array}\right)+\left(\begin{array}{cc}
        -2 \lambda_2 & 3 \lambda_2 \\
        4 \lambda_2  & 3 \lambda_2
      \end{array}\right)+\left(\begin{array}{cc}
        0          & \lambda_3 \\
        -\lambda_3 & \lambda_3
      \end{array}\right)+\left(\begin{array}{cc}
        \lambda_4   & -3 \lambda_4 \\
        2 \lambda_4 & -\lambda_4
      \end{array}\right)=0
  $$$$
    \left(\begin{array}{cc}
        2 \lambda_1-2 \lambda_2+0+\lambda_4           & 2 \lambda_1+3 \lambda_2+\lambda_3-3 \lambda_4 \\
        3 \lambda_1+4 \lambda_2-\lambda_3+2 \lambda_4 & \lambda_1+3 \lambda_2+\lambda_3-\lambda_4
      \end{array}\right)=0
  $$$$
    \left\{\begin{array}{ccccccc}
      2 \lambda_1 & - 2\lambda_2 &             & + \lambda_4    & =0 \\
      3 \lambda_1 & + 4\lambda_2 & - \lambda_3 & +  2 \lambda_4 & =0 \\
      2 \lambda_1 & + 3\lambda_2 & + \lambda_3 & -  3 \lambda_4 & =0 \\
      \lambda_1   & + 3\lambda_2 & + \lambda_3 & -  \lambda_4   & =0
    \end{array}\right.
  $$

  $$
    \begin{array}{|cccc|c}
      2 & -2 & 0  & 1  &                \\
      3 & 4  & -1 & 2  & -1.5\mathrm{I} \\
      2 & 3  & 1  & -3 & -\mathrm{I}    \\
      1 & 3  & 1  & -1 & -0.5\mathrm{I} \\
    \end{array}=
    \begin{array}{|cccc|c}
      2 & -2 & 0  & 1    &                         \\
      0 & 7  & -1 & 0.5  &                         \\
      0 & 5  & 1  & -4   & -\frac{5}{7}\mathrm{II} \\
      0 & 4  & 1  & -1.5 & +\frac{4}{7}\mathrm{II}
    \end{array}=
    \begin{array}{|cccc|c}
      2 & -2 & 0               & 1                &                            \\
      0 & 7  & -1              & 0.5              &                            \\
      0 & 0  & 1 \frac{5}{7}   & -4 \frac{5}{14}  &                            \\
      0 & 0  & 1_{\frac{4}{7}} & -1 \frac{11}{14} & -\frac{11}{12}\mathrm{III}
    \end{array}=
  $$$$
    =\begin{array}{|cccc|c}
      2 & -2 & 0             & 1               & \\
      0 & 7  & -1            & 0.5             & \\
      0 & 0  & 1 \frac{5}{7} & -4 \frac{5}{14} & \\
      0 & 0  & 0             & 2 \frac{5}{24}  &
    \end{array}=2 \cdot 7 \cdot 1 \frac{5}{7} \cdot 2 \frac{5}{24}=53 \Rightarrow $$
  $\Rightarrow$ система образует базис пространства $\mathcal{L}$
\end{center}
\subsubsection*{Найдите в этом базисе координаты элемента $x$.
}
\begin{center}
  $x=(\lambda_1,\lambda_2,\lambda_3,\lambda_4)$
  $$
    \left(\begin{array}{cc}
        2 \lambda_1 & 2 \lambda_1 \\
        3 \lambda_1 & \lambda_1
      \end{array}\right)+\left(\begin{array}{cc}
        -2 \lambda_2 & 3 \lambda_2 \\
        4 \lambda_2  & 3 \lambda_2
      \end{array}\right)+\left(\begin{array}{cc}
        0          & \lambda_3 \\
        -\lambda_3 & \lambda_3
      \end{array}\right)+\left(\begin{array}{cc}
        \lambda_4   & -3 \lambda_4 \\
        2 \lambda_4 & -\lambda_4
      \end{array}\right)=\left(\begin{array}{cc}
        -3 & -6 \\
        0  & 2
      \end{array}\right)
  $$$$
    \left(\begin{array}{cc}
        2 \lambda_1-2 \lambda_2+0+\lambda_4           & 2 \lambda_1+3 \lambda_2+\lambda_3-3 \lambda_4 \\
        3 \lambda_1+4 \lambda_2-\lambda_3+2 \lambda_4 & \lambda_1+3 \lambda_2+\lambda_3-\lambda_4
      \end{array}\right)=\left(\begin{array}{cc}
        -3 & -6 \\
        0  & 2
      \end{array}\right)
  $$$$
    \left\{\begin{array}{ccccccc}
      2 \lambda_1 & - 2\lambda_2 &             & + \lambda_4    & =-3 \\
      3 \lambda_1 & + 4\lambda_2 & - \lambda_3 & +  2 \lambda_4 & =0  \\
      2 \lambda_1 & + 3\lambda_2 & + \lambda_3 & -  3 \lambda_4 & =-6 \\
      \lambda_1   & + 3\lambda_2 & + \lambda_3 & -  \lambda_4   & =2
    \end{array}\right.
  $$ $$\left(\begin{array}{cccc|c}
        2 & -2 & 0  & 1  & -3 \\
        3 & 4  & -1 & 2  & 0  \\
        2 & 3  & 1  & -3 & -6 \\
        1 & 3  & 1  & -1 & 2  \\
      \end{array}\right)= \dotsb =\left(\begin{array}{cccc|c}
        1 & 0 & 0 & 0 & -2 \\
        0 & 1 & 0 & 0 & 1  \\
        0 & 0 & 1 & 0 & 4  \\
        0 & 0 & 0 & 1 & 3  \\
      \end{array}\right)
  $$$$\left\{\begin{array}{l}
      \lambda_1 =-3 \\
      \lambda_2 =0  \\
      \lambda_3 =-6 \\
      \lambda_4 =2
    \end{array}\right.$$
  Ответ: $x=(-3,\ 0,\ -6,\ 2)$
\end{center}

\subsection*{L - пространство многочленов степени не больше четырёх}
$$
  e_1=1+t\ \ e_2=t+t^2\ \ e_3=t^2+t^3\ \ e_4=t+t^3\ e_5=t^2+t^4
$$$$
  x=t^4-t^3+t^2-t+1$$$$
  e_1=(0,0,0,1,1)\ e_2=(0,1,0,1,0){\ e}_3=(0,1,1,0,0)\ e_4=(0,1,0,1,0)\ e_5=(1,0,1,0,0) $$
$$
  \lambda_1e_1+\lambda_2e_2+\lambda_3e_3+\lambda_4e_4+\lambda_5e_5= $$$$
  =(0,0,0,\lambda_1,\lambda_1)+(0,\lambda_2,0,\lambda_2,0)+\ (0,\lambda_3,\lambda_3,0,0)\ +\ (0,\lambda_4,0,\lambda_4,0)\ +\ (\lambda_5,0,\lambda_5,0,0) =\ 0\ $$$$
  (\lambda_2+\lambda_3),(\lambda_3+\lambda_4),(\lambda_3+\lambda_5),(\lambda_1+\lambda_2),(\lambda_2+\lambda_4) =0
$$
Следовательно $$
  \left\{\begin{array}{c}
    \lambda_2+\lambda_3=0 \\
    \lambda_3+\lambda_4=0 \\
    \lambda_3+\lambda_5=0 \\
    \lambda_1+\lambda_2=0 \\
    \lambda_2+\lambda_4=0 \\
  \end{array}\right. \Rightarrow \left(\begin{array}{ccccc|c}
      1 & 1 & 0 & 0 & 0 & 0 \\
      0 & 1 & 1 & 0 & 0 & 0 \\
      0 & 1 & 0 & 1 & 0 & 0 \\
      0 & 0 & 1 & 1 & 0 & 0 \\
      0 & 0 & 1 & 0 & 1 & 0 \\
    \end{array}\right)$$
От 1 строки отнимаем 2 строку, умноженную на 1, от 3 строки отнимаем 2 строку, умноженную на 1$$
  \left(\begin{matrix}1&0&\begin{matrix}-1&\begin{matrix}0&0\\\end{matrix}\\\end{matrix}\\0&1&\begin{matrix}1&\begin{matrix}0&0\\\end{matrix}\\\end{matrix}\\\begin{matrix}0\\\begin{matrix}0\\0\\\end{matrix}\\\end{matrix}&\begin{matrix}0\\\begin{matrix}0\\0\\\end{matrix}\\\end{matrix}&\begin{matrix}\begin{matrix}-1&\begin{matrix}1&0\\\end{matrix}\\\end{matrix}\\\begin{matrix}\begin{matrix}1&\begin{matrix}1&0\\\end{matrix}\\\end{matrix}\\\begin{matrix}1&\begin{matrix}0&1\\\end{matrix}\\\end{matrix}\\\end{matrix}\\\end{matrix}\\\end{matrix}\middle|\begin{matrix}0\\0\\\begin{matrix}0\\\begin{matrix}0\\0\\\end{matrix}\\\end{matrix}\\\end{matrix}\right)$$
3-ую строку делим на -1$$
  \left(\begin{matrix}1&0&\begin{matrix}-1&\begin{matrix}0&0\\\end{matrix}\\\end{matrix}\\0&1&\begin{matrix}1&\begin{matrix}0&0\\\end{matrix}\\\end{matrix}\\\begin{matrix}0\\\begin{matrix}0\\0\\\end{matrix}\\\end{matrix}&\begin{matrix}0\\\begin{matrix}0\\0\\\end{matrix}\\\end{matrix}&\begin{matrix}\begin{matrix}1&-\begin{matrix}1&0\\\end{matrix}\\\end{matrix}\\\begin{matrix}\begin{matrix}1&\begin{matrix}1&0\\\end{matrix}\\\end{matrix}\\\begin{matrix}1&\begin{matrix}0&1\\\end{matrix}\\\end{matrix}\\\end{matrix}\\\end{matrix}\\\end{matrix}\middle|\begin{matrix}0\\0\\\begin{matrix}0\\\begin{matrix}0\\0\\\end{matrix}\\\end{matrix}\\\end{matrix}\right)$$
К 1 строке добавляем 3 строку, умноженную на 1, от 2 строки отнимаем 3 строку, умноженную на 1, от 4 строки отнимаем 3 строку, умноженную на 1, от 5 строки отнимаем 3 строку, умноженную на 1$$
  \left(\begin{matrix}1&0&\begin{matrix}0&\begin{matrix}-1&0\\\end{matrix}\\\end{matrix}\\0&1&\begin{matrix}0&\begin{matrix}1&0\\\end{matrix}\\\end{matrix}\\\begin{matrix}0\\\begin{matrix}0\\0\\\end{matrix}\\\end{matrix}&\begin{matrix}0\\\begin{matrix}0\\0\\\end{matrix}\\\end{matrix}&\begin{matrix}\begin{matrix}1&-\begin{matrix}1&0\\\end{matrix}\\\end{matrix}\\\begin{matrix}\begin{matrix}0&\begin{matrix}2&0\\\end{matrix}\\\end{matrix}\\\begin{matrix}0&\begin{matrix}1&1\\\end{matrix}\\\end{matrix}\\\end{matrix}\\\end{matrix}\\\end{matrix}\middle|\begin{matrix}0\\0\\\begin{matrix}0\\\begin{matrix}0\\0\\\end{matrix}\\\end{matrix}\\\end{matrix}\right)$$
4-ую строку делим на 2$$
  \left(\begin{matrix}1&0&\begin{matrix}0&\begin{matrix}-1&0\\\end{matrix}\\\end{matrix}\\0&1&\begin{matrix}0&\begin{matrix}1&0\\\end{matrix}\\\end{matrix}\\\begin{matrix}0\\\begin{matrix}0\\0\\\end{matrix}\\\end{matrix}&\begin{matrix}0\\\begin{matrix}0\\0\\\end{matrix}\\\end{matrix}&\begin{matrix}\begin{matrix}1&-\begin{matrix}1&0\\\end{matrix}\\\end{matrix}\\\begin{matrix}\begin{matrix}0&\begin{matrix}1&0\\\end{matrix}\\\end{matrix}\\\begin{matrix}0&\begin{matrix}1&1\\\end{matrix}\\\end{matrix}\\\end{matrix}\\\end{matrix}\\\end{matrix}\middle|\begin{matrix}0\\0\\\begin{matrix}0\\\begin{matrix}0\\0\\\end{matrix}\\\end{matrix}\\\end{matrix}\right)$$


К 1 строке добавляем 4 строку, умноженную на 1, от 2 строки отнимаем 4 строку, умноженную на 1, к 3 строке добавляем 4 строку, умноженную на 1, от 5 строки отнимаем 4 строку, умноженную на 1$$
  \left(\begin{matrix}1&0&\begin{matrix}0&\begin{matrix}0&0\\\end{matrix}\\\end{matrix}\\0&1&\begin{matrix}0&\begin{matrix}0&0\\\end{matrix}\\\end{matrix}\\\begin{matrix}0\\\begin{matrix}0\\0\\\end{matrix}\\\end{matrix}&\begin{matrix}0\\\begin{matrix}0\\0\\\end{matrix}\\\end{matrix}&\begin{matrix}\begin{matrix}1&\begin{matrix}0&0\\\end{matrix}\\\end{matrix}\\\begin{matrix}\begin{matrix}0&\begin{matrix}1&0\\\end{matrix}\\\end{matrix}\\\begin{matrix}0&\begin{matrix}0&1\\\end{matrix}\\\end{matrix}\\\end{matrix}\\\end{matrix}\\\end{matrix}\middle|\begin{matrix}0\\0\\\begin{matrix}0\\\begin{matrix}0\\0\\\end{matrix}\\\end{matrix}\\\end{matrix}\right)\rightarrow $$ $\lambda_1=0\lambda_2=0\lambda_3=0\lambda_4=0\lambda_5=0$
Найдём координаты x в пространстве многочленов.
$$x=(1,-1,1,-1,1)$$
Координатами вектора x является 5 чисел $(\lambda_1, \lambda_2, \lambda_3, \lambda_4, \lambda_5)$, удовлетворяющие уравнению$$
  \lambda_1e_1+\lambda_2e_2+\lambda_3e_3+\lambda_4e_4+\lambda_5e_5=(1,-1,\ 1,-1,\ 1)$$$$
  (0,0,0,\lambda_1,\lambda_1)+(0,\lambda_2,0,\lambda_2,0)+\ (0,\lambda_3,\lambda_3,0,0)\ +\ (0,\lambda_4,0,\lambda_4,0)\ +\ (\lambda_5,0,\lambda_5,0,0)=(1,-1,\ 1,-1,\ 1)$$
Перепишем систему уравнений в матричном виде и решим его методом Гаусса$$
  \left(\begin{matrix}1&1&\begin{matrix}0&\begin{matrix}0&0\\\end{matrix}\\\end{matrix}\\0&1&\begin{matrix}1&\begin{matrix}0&0\\\end{matrix}\\\end{matrix}\\\begin{matrix}0\\\begin{matrix}0\\0\\\end{matrix}\\\end{matrix}&\begin{matrix}1\\\begin{matrix}0\\0\\\end{matrix}\\\end{matrix}&\begin{matrix}\begin{matrix}0&\begin{matrix}1&0\\\end{matrix}\\\end{matrix}\\\begin{matrix}\begin{matrix}1&\begin{matrix}1&0\\\end{matrix}\\\end{matrix}\\\begin{matrix}1&\begin{matrix}0&1\\\end{matrix}\\\end{matrix}\\\end{matrix}\\\end{matrix}\\\end{matrix}\middle|\begin{matrix}1\\-1\\\begin{matrix}1\\\begin{matrix}-1\\1\\\end{matrix}\\\end{matrix}\\\end{matrix}\right)$$
От 1 строки отнимаем 2 строку, умноженную на 1, от 3 строки отнимаем 2 строку, умноженную на 1$$
  \left(\begin{matrix}1&0&\begin{matrix}-1&\begin{matrix}0&0\\\end{matrix}\\\end{matrix}\\0&1&\begin{matrix}1&\begin{matrix}0&0\\\end{matrix}\\\end{matrix}\\\begin{matrix}0\\\begin{matrix}0\\0\\\end{matrix}\\\end{matrix}&\begin{matrix}0\\\begin{matrix}0\\0\\\end{matrix}\\\end{matrix}&\begin{matrix}\begin{matrix}-1&\begin{matrix}1&0\\\end{matrix}\\\end{matrix}\\\begin{matrix}\begin{matrix}1&\begin{matrix}1&0\\\end{matrix}\\\end{matrix}\\\begin{matrix}1&\begin{matrix}0&1\\\end{matrix}\\\end{matrix}\\\end{matrix}\\\end{matrix}\\\end{matrix}\middle|\begin{matrix}2\\-1\\\begin{matrix}2\\\begin{matrix}-1\\1\\\end{matrix}\\\end{matrix}\\\end{matrix}\right)$$
3-ую строку делим на -1$$
  \left(\begin{matrix}1&0&\begin{matrix}-1&\begin{matrix}0&0\\\end{matrix}\\\end{matrix}\\0&1&\begin{matrix}1&\begin{matrix}0&0\\\end{matrix}\\\end{matrix}\\\begin{matrix}0\\\begin{matrix}0\\0\\\end{matrix}\\\end{matrix}&\begin{matrix}0\\\begin{matrix}0\\0\\\end{matrix}\\\end{matrix}&\begin{matrix}\begin{matrix}1&-\begin{matrix}1&0\\\end{matrix}\\\end{matrix}\\\begin{matrix}\begin{matrix}1&\begin{matrix}1&0\\\end{matrix}\\\end{matrix}\\\begin{matrix}1&\begin{matrix}0&1\\\end{matrix}\\\end{matrix}\\\end{matrix}\\\end{matrix}\\\end{matrix}\middle|\begin{matrix}2\\-1\\\begin{matrix}-2\\\begin{matrix}-1\\1\\\end{matrix}\\\end{matrix}\\\end{matrix}\right)$$
К 1 строке добавляем 3 строку, умноженную на 1, от 2 строки отнимаем 3 строку, умноженную на 1, от 4 строки отнимаем 3 строку, умноженную на 1, от 5 строки отнимаем 3 строку, умноженную на 1$$
  \left(\begin{matrix}1&0&\begin{matrix}0&\begin{matrix}-1&0\\\end{matrix}\\\end{matrix}\\0&1&\begin{matrix}0&\begin{matrix}1&0\\\end{matrix}\\\end{matrix}\\\begin{matrix}0\\\begin{matrix}0\\0\\\end{matrix}\\\end{matrix}&\begin{matrix}0\\\begin{matrix}0\\0\\\end{matrix}\\\end{matrix}&\begin{matrix}\begin{matrix}1&-\begin{matrix}1&0\\\end{matrix}\\\end{matrix}\\\begin{matrix}\begin{matrix}0&\begin{matrix}2&0\\\end{matrix}\\\end{matrix}\\\begin{matrix}0&\begin{matrix}1&1\\\end{matrix}\\\end{matrix}\\\end{matrix}\\\end{matrix}\\\end{matrix}\middle|\begin{matrix}0\\1\\\begin{matrix}-2\\\begin{matrix}1\\3\\\end{matrix}\\\end{matrix}\\\end{matrix}\right)$$
4-ую строку делим на 2$$
  \left(\begin{matrix}1&0&\begin{matrix}0&\begin{matrix}-1&0\\\end{matrix}\\\end{matrix}\\0&1&\begin{matrix}0&\begin{matrix}1&0\\\end{matrix}\\\end{matrix}\\\begin{matrix}0\\\begin{matrix}0\\0\\\end{matrix}\\\end{matrix}&\begin{matrix}0\\\begin{matrix}0\\0\\\end{matrix}\\\end{matrix}&\begin{matrix}\begin{matrix}1&-\begin{matrix}1&0\\\end{matrix}\\\end{matrix}\\\begin{matrix}\begin{matrix}0&\begin{matrix}1&0\\\end{matrix}\\\end{matrix}\\\begin{matrix}0&\begin{matrix}1&1\\\end{matrix}\\\end{matrix}\\\end{matrix}\\\end{matrix}\\\end{matrix}\middle|\begin{matrix}0\\1\\\begin{matrix}-2\\\begin{matrix}0.5\\3\\\end{matrix}\\\end{matrix}\\\end{matrix}\right)$$
к 1 строке добавляем 4 строку, умноженную на 1, от 2 строки отнимаем 4 строку, умноженную на 1, к 3 строке добавляем 4 строку, умноженную на 1, от 5 строки отнимаем 4 строку, умноженную на 1$$
  \left(\begin{matrix}1&0&\begin{matrix}0&\begin{matrix}0&0\\\end{matrix}\\\end{matrix}\\0&1&\begin{matrix}0&\begin{matrix}0&0\\\end{matrix}\\\end{matrix}\\\begin{matrix}0\\\begin{matrix}0\\0\\\end{matrix}\\\end{matrix}&\begin{matrix}0\\\begin{matrix}0\\0\\\end{matrix}\\\end{matrix}&\begin{matrix}\begin{matrix}1&\begin{matrix}0&0\\\end{matrix}\\\end{matrix}\\\begin{matrix}\begin{matrix}0&\begin{matrix}1&0\\\end{matrix}\\\end{matrix}\\\begin{matrix}0&\begin{matrix}0&1\\\end{matrix}\\\end{matrix}\\\end{matrix}\\\end{matrix}\\\end{matrix}\middle|\begin{matrix}0.5\\0.5\\\begin{matrix}-1.5\\\begin{matrix}0.5\\2.5\\\end{matrix}\\\end{matrix}\\\end{matrix}\right)\Rightarrow x
  \left\{\begin{array}{l}
    x_1=0.5  \\
    x_2=0.5  \\
    x_3=-1.5 \\
    x_4=0.5  \\
    x_5=2.5
  \end{array}\right.
$$
$$
$$
\end{document}