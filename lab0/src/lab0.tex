\documentclass{article}
\usepackage[paper=letterpaper,margin=2cm]{geometry}
\usepackage[utf8]{inputenc}
\usepackage[russian]{babel}
\usepackage[]{graphicx}
\usepackage[usenames]{color}
\usepackage{colortbl}
\usepackage{listings}
\usepackage{geometry}
\usepackage{xcolor}

\geometry{
  a4paper,
  top=25mm, 
  right=30mm, 
  bottom=25mm, 
  left=30mm
}

\begin{document}

\begin{center}
  \section*{
    Федеральное государственное автономное образовательное учреждение\\ высшего образования\\
    «Национальный исследовательский университет ИТМО»\\
    Факультет Программной Инженерии и Компьютерной Техники \\
   }
  \includegraphics[scale=0.2]{../../img/itmo.png}
\end{center}
\vspace{4cm}


\begin{center}
  \large \textbf{Вариант \textnumero 311517}\\
  \textbf{Лабораторная работа \textnumero 1}\\
  по дисциплине\\
  \textbf{Основы профессиональной деятельности}
\end{center}

\vspace*{\fill}

\begin{flushright}
  Выполнил Студент группы P3115\\
  \textbf{Владимир Мацюк}\\
  Преподаватель: \\
  \textbf{ФИО препода}\\
\end{flushright}

\vspace{1cm}

\begin{center}
  г. Санкт-Петербург\\
  2022г.
\end{center}

\newpage
\section{Текст задания}

\section{Исходный код программы}

\lstset{
  language=bash,
  breaklines=true,
  extendedchars=false,
  showspaces=false,
  showstringspaces=false
}
\lstinputlisting{script.sh}

\section{Результат выполнения}

\section{Вывод}

\end{document}
