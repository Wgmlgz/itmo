\documentclass{article}
\usepackage[paper=letterpaper,margin=2cm]{geometry}
\usepackage[russian]{babel}
\usepackage[utf8]{inputenc}
\usepackage[]{graphicx}
\usepackage[usenames]{color}
\usepackage{colortbl}
\usepackage{geometry}
\usepackage{xcolor}
\usepackage{listings}

\geometry{
  a4paper,
  top=25mm, 
  right=30mm, 
  bottom=25mm, 
  left=30mm
}

\begin{document}

\begin{center}
  \section*{
    Федеральное государственное автономное образовательное учреждение\\ высшего образования\\
    «Национальный исследовательский университет ИТМО»\\
    Факультет Программной Инженерии и Компьютерной Техники \\
   }
  \includegraphics[scale=0.2]{../../img/itmo.png}
\end{center}
\vspace{4cm}


\begin{center}
  \large \textbf{Вариант \textnumero 666}\\
  \textbf{Лабораторная работа \textnumero 1}\\
  по дисциплине\\
  \textbf{Программирование}
\end{center}

\vspace*{\fill}

\begin{flushright}
  Выполнил Студент группы P3115\\
  \textbf{Владимир Мацюк}\\
  Преподаватель: \\
  \textbf{ФИО препода}\\
\end{flushright}

\vspace{1cm}

\begin{center}
  г. Санкт-Петербург\\
  2022г.
\end{center}

\newpage

\definecolor{od_black}{HTML}{282C34}
\definecolor{od_red}{HTML}{E06C75}
\definecolor{od_green}{HTML}{98C379}
\definecolor{od_blue}{HTML}{61AFEF}
\definecolor{od_purple}{HTML}{C678DD}

\lstset{
  inputencoding=utf8,
  frame=single,
  language=Java,
  breaklines=true,
  extendedchars=false,
  showspaces=false,
  showstringspaces=false,
  basicstyle=\footnotesize\ttfamily,
  keywordstyle=\bfseries\color{od_purple},
  commentstyle=\itshape\color{od_purple},
  identifierstyle=\color{od_blue},
  stringstyle=\color{od_green},
}

\section*{Текст задания}

\begin{enumerate}
  \item Создать одномерный массив a типа short. Заполнить его чётными числами от 4 до 18 включительно в порядке возрастания.
  \item Создать одномерный массив x типа double. Заполнить его 19-ю случайными числами в диапазоне от -12.0 до 13.0.
  \item Создать двумерный массив a размером 8x19. Вычислить его элементы по следующей формуле (где $x = x[j]$) \begin{itemize}
          \item если $a[i] = 16$, то $a[i][j] = ln(|ln(tan^2(x))|)$
          \item если $a[i] \in \{10, 12, 14, 18\}$, то $
                  a[i][j] = \left(
                  \frac{
                    sin\left(arctan\left(\frac{x+0.5}{25}\right)\right)
                  }{
                    \sqrt[3]{x^x}-1
                  }
                  \right)^{
                    sin\left(x^{x\cdot\left(x+\frac{2}{3}\right)}\right)
                  }
                $
          \item для остальных значений $a[i]: a[i][j]=sin(arcsin(cos(tan(cos(x)))))$
        \end{itemize}
  \item Напечатать полученный в результате массив в формате с пятью знаками после запятой.
\end{enumerate}

\section*{Исходный код программы}
\lstinputlisting{Lab1.java}

\section*{Результат работы программы}
\lstinputlisting{out.txt}

\section*{Вывод}
Я осознал что обречён писать 4 года лабы на java $( $
\end{document}
