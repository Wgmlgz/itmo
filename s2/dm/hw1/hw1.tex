\documentclass{article}
\usepackage{import}
\import{../../../lib/latex/}{wgmlgz}


\begin{document}

\itmo[
  variant=161,
  labn=1,
  worktype=Домашняя работа,
  discipline=Дискретная математика,
  group=P3115,
  student=Владимир Мацюк,
  teacher=Поляков Владимир Иванович,
  logo=../../../lib/img/itmo.png
]

Исходная таблица соединений R:

$$\begin{tabular}{|c|c|c|c|c|c|c|c|c|c|c|c|c|c|c|c|c|c|c|c|c|c|c|c|} \hline
    v/v & e1 & e2 & e3 & e4 & e5 & e6 & e7 & e8 & e9 & e10 & e11 & e12 & ri  \nl
    e1  & 0  & 3  &    &    & 4  & 4  & 4  & 4  &    & 3   & 4   &     & 7  \nl
    e2  & 3  & 0  & 1  &    &    &    &    & 4  &    & 2   &     &     & 4  \nl
    e3  &    & 1  & 0  & 5  &    &    &    &    & 3  & 1   &     &     & 4  \nl
    e4  &    &    & 5  & 0  & 1  & 4  & 1  &    & 4  & 5   & 4   &     & 7  \nl
    e5  & 4  &    &    & 1  & 0  & 1  &    &    &    & 3   &     &     & 4  \nl
    e6  & 4  &    &    & 4  & 1  & 0  & 2  &    &    &     & 4   &     & 5  \nl
    e7  & 4  &    &    & 1  &    & 2  & 0  &    &    & 4   &     & 1   & 5  \nl
    e8  & 4  & 4  &    &    &    &    &    & 0  & 3  & 3   &     & 5   & 5  \nl
    e9  &    &    & 3  & 4  &    &    &    & 3  & 0  &     & 5   &     & 4  \nl
    e10 & 3  & 2  & 1  & 5  & 3  &    & 4  & 3  &    & 0   & 2   &     & 8  \nl
    e11 & 4  &    &    & 4  &    & 4  &    &    & 8  & 2   & 0   & 4   & 6  \nl
    e12 &    &    &    &    &    &    & 1  & 5  &    &     & 4   & 0   & 3  \nl
  \end{tabular}$$

\begin{enumerate}

  \item Положим j = 1;
  \item Упорядочим вершины графа в порядке не возрастания ri:
        e10, e1, e4, e11, e6, e7, e8, e2, e3, e5, e9, e12

        % [('e10', 8, [1, 1, 1, 1, 1, 0, 1, 1, 0, 0, 1, 0]),
        %   ('e1', 7, [0, 1, 0, 0, 1, 1, 1, 1, 0, 1, 1, 0]),
        %   ('e4', 7, [0, 0, 1, 0, 1, 1, 1, 0, 1, 1, 1, 0]),
        %   ('e11', 6, [1, 0, 0, 1, 0, 1, 0, 0, 1, 1, 0, 1]),
        %   ('e6', 5, [1, 0, 0, 1, 1, 0, 1, 0, 0, 0, 1, 0]),
        %   ('e7', 5, [1, 0, 0, 1, 0, 1, 0, 0, 0, 1, 0, 1]),
        %   ('e8', 5, [1, 1, 0, 0, 0, 0, 0, 0, 1, 1, 0, 1]),
        %   ('e2', 4, [1, 0, 1, 0, 0, 0, 0, 1, 0, 1, 0, 0]),
        %   ('e3', 4, [0, 1, 0, 1, 0, 0, 0, 0, 1, 1, 0, 0]),
        %   ('e5', 4, [1, 0, 0, 1, 0, 1, 0, 0, 0, 1, 0, 0]),
        %   ('e9', 4, [0, 0, 1, 1, 0, 0, 0, 1, 0, 0, 1, 0]),
        %   ('e12', 3, [0, 0, 0, 0, 0, 0, 1, 1, 0, 0, 1, 0])]
  \item Красим в первый цвет вершины e10, e6, e9, e12. Остальные вершины смежны вершине e10.
  \item Так как остались неокрашенные вершины, удалим из матрицы R строки и столбцы, соответствующие вершинам e10, e6, e9, e12.
        $$\begin{tabular}{|c|c|c|c|c|c|c|c|c|c|c|c|c|c|c|c|c|c|c|c|c|c|c|c|} \hline
            v/v & e1 & e2 & e3 & e4 & e5 & e7 & e8 & e11 & $r_i$ \nl
            e1  & 0  & 3  &    &    & 4  & 4  & 4  & 4   & 5 \nl
            e2  & 3  & 0  & 1  &    &    &    & 4  &     & 3 \nl
            e3  &    & 1  & 0  & 5  &    &    &    &     & 2 \nl
            e4  &    &    & 5  & 0  & 1  & 1  &    & 4   & 4 \nl
            e5  & 4  &    &    & 1  & 0  &    &    &     & 2 \nl
            e7  & 4  &    &    & 1  &    & 0  &    &     & 2 \nl
            e8  & 4  & 4  &    &    &    &    & 0  &     & 2 \nl
            e11 & 4  &    &    & 4  &    &    &    & 0   & 2 \nl
          \end{tabular}$$
  \item Положим j = j + 1 = 1 + 1 = 2
  \item Упорядочим вершины графа в порядке не возрастания ri:
        e1, e4, e2, e3, e5, e7, e8, e11
  \item Красим во 2 цвет вершины e1, e4.
  \item Так как остались неокрашенные вершины, удалим из матрицы R строки и столбцы, соответствующие вершинам e1, e4.
        $$\begin{tabular}{|c|c|c|c|c|c|c|c|c|c|c|c|c|c|c|c|c|c|c|c|c|c|c|c|} \hline
            v/v & e2 & e3 & e5 & e7 & e8 & e11 & $r_i$ \nl
            e2  & 0  & 1  &    &    & 4  &     & 2 \nl
            e3  & 1  & 0  &    &    &    &     & 1 \nl
            e5  &    &    & 0  &    &    &     & 0 \nl
            e7  &    &    &    & 0  &    &     & 0 \nl
            e8  & 4  &    &    &    & 0  &     & 1 \nl
            e11 &    &    &    &    &    & 0   & 0 \nl
          \end{tabular}$$
  \item Положим  j = j + 1 = 3
  \item Упорядочим вершины графа в порядке не возрастания ri: 2,3,8,5,7,11,
  \item  Красим в 3 цвет вершины e2, e5, e7, e11.
  \item Так как остались неокрашенные вершины, удалим из матрицы R строки и столбцы, соответствующие вершинам e2, e5, e7, e11.
        $$\begin{tabular}{|c|c|c|c|c|c|c|c|c|c|c|c|c|c|c|c|c|c|c|c|c|c|c|c|} \hline
            v/v & e3 & e8 & ri\nl
            e3  & 0  &    & 0 \nl
            e8  &    & 0  & 0 \nl
          \end{tabular}$$
  \item  Положим  j = j + 1 = 4
  \item Упорядочим вершины графа в порядке не возрастания ri: 3,8,
  \item  Красим во 4 цвет вершины e3, e8
  \item Так как остались неокрашенные вершины, удалим из матрицы R строки и столбцы, соответствующие вершинам e3, e8

        В результате, все вершины окрашены, хроматическое число равно 4.
\end{enumerate}

\end{document}
