\documentclass{article}
\usepackage{import}
\import{../../../lib/latex/}{wgmlgz}


\begin{document}

\itmo[
  variant=31158,
  labn=2,
  discipline=Основы профессиональной деятельности,
  group=P3115,
  student=Владимир Мацюк,
  teacher=Пашнин Александр Денисович,
  logo=../../../lib/img/itmo.png
]

\section{Задание}
По выданному преподавателем варианту восстановить текст заданного варианта программы, определить предназначение и составить описание программы, определить область представления и область допустимых значений исходных данных и результата, выполнить трассировку программы.
\begin{center}
  \includegraphics[scale=0.8]{task.png}
\end{center}
\section{Таблица комманд}
\begin{tabular}{|c|r|l|l|} \hline
  Адрес & Код команды & Мнемоника          & Комментарии \nl
  4D0   & 04E0        & A                  & \nl
  4D1   & A000        & B                  & Загрузка (Прямая абсолютная адресация)\nl
  4D2   & E000        & C                  & Сохранение (Прямая абсолютная адресация)\nl
  4D3   & E000        & D                  & Сохранение (Прямая абсолютная адресация)\nl
  4D4   & +0200       & CLA                & Очистка аккумулятора\nl
  4D5   & EEFD        & ST IP-3 (D)        & Сохранение (Прямая относительная адресация)\nl
  4D6   & AF04        & LD 0x04            & Загрузка (Прямая загрузка операнда)\nl
  4D7   & EEFA        & ST IP-6 (C)        & Сохранение (Прямая относительная адресация)\nl
  4D8   & AEF7        & LD IP-9 (A)        & Загрузка (Прямая относительная адресация)\nl
  4D9   & EEF7        & ST IP-9 (B)        & Сохранение (Прямая относительная адресация)\nl

  4DA   & AAF6        & ma: LD (IP-A)+ (B) & Загрузка (Косвенная относительная автоинкрементная адресация)\nl
  4DB   & F301        & BPL IP+1           & Переход, если плюс\nl
  4DC   & 3AF6        & OR (IP-A)+ (D)     & Логическое или (Косвенная относительная автоинкрементная адресация)\nl
  4DD   & 84D2        & LOOP 0x4D2 (C)     & Декремент и пропуск (Прямая абсолютная адресация)\nl
  4DE   & CEFB        & BR IP-5 (ma)       & Безусловный переход (эквивалент JUMP c прямой относительной адресацией)\nl
  4DF   & 0100        & HLT                & Остановка\nl
  4E0   & 44D4        & ADD 0x4D4          & Сложение (Прямая абсолютная адресация)\nl
  4E1   & CE00        & Константа/ошибка   & \nl
  4E2   & 0900        & POPF               & Чтение флагов из стэка\nl
  4E3   & 0900        & POPF               & Чтение флагов из стэка \nl
\end{tabular}

\section{Функция}

\begin{lstlisting}
  
d = 0
c = 0x04
b = a

do {
  ac = b++
  if ac <= 0 {
    ac = ac | d++
  }
  --c
} while (c > 0);

\end{lstlisting}

\section{Область допустимых значений}
Пусть: $ X = -A -B $, тогда:
$$ -2^{15} \le X \le 2^{15} - 1 $$
$$ -2^{15} \le X,\ C \le 2^{15} - 1 $$
$$ -2^{15} \le X\ \&\ C \le 2^{15} - 1 $$
$$ -2^{15} \le ((-A -B)\ \&\ C) \le 2^{15} - 1 $$




\section{Область определения}
$$
  \left[{ \begin{array}{l}
        \begin{array}{l} -2^{14}+1 \le A,B \le 2^{14} \\
        \end{array}                \\
        \\
        \left\{ \begin{array}{l}
          -2^{15}+1 \le A   \le 0 \\
          0 \le B   \le 2^{15}    \\
        \end{array}\right. \\
        \\
        \left\{ \begin{array}{l}
          0 \le A   \le 2^{15} \\
          -2^{15}+1 \le B   \le 0
        \end{array}\right.
      \end{array}}\right.
$$

\section{Расположение данных в памяти}
Исходные данные: 0x178, 0x179, 0x17A. \\
Программа: 0x17B-0x182. \\
Промежуточное значение: 0x183. \\
Результат: 0x284. \\

\section{Таблица трассировки}

\begin{tabular}{|c|c|c|c|c|c|c|c|c|c|c|c|c|c|c|} \hline
  Адр & Код  & IP  & CR   & AR  & DR   & SP  & BR   & AC   & PS  & NZVC & Адр & Код \nl
  17B & 0200 & 17B & 0000 & 000 & 0000 & 000 & 0000 & 0000 & 004 & 0100 &     & \nl
  17B & 0200 & 17C & 0200 & 17B & 0200 & 000 & 017B & 0000 & 004 & 0100 &     & \nl
  17C & 6178 & 17D & 6178 & 178 & E184 & 000 & 017C & 1E7C & 000 & 0000 &     & \nl
  17D & 6179 & 17E & 6179 & 179 & 0100 & 000 & 017D & 1D7C & 001 & 0001 &     & \nl
  17E & E183 & 17F & E183 & 183 & 1D7C & 000 & 017E & 1D7C & 001 & 0001 & 183 & 1D7C \nl
  17F & A17A & 180 & A17A & 17A & 0200 & 000 & 017F & 0200 & 001 & 0001 &     & \nl
  180 & 2183 & 181 & 2183 & 183 & 1D7C & 000 & 0180 & 0000 & 005 & 0101 &     & \nl
  181 & E184 & 182 & E184 & 184 & 0000 & 000 & 0181 & 0000 & 005 & 0101 & 184 & 0000 \nl
  182 & 0100 & 183 & 0100 & 182 & 0100 & 000 & 0182 & 0000 & 005 & 0101 &     & \nl
\end{tabular}
\section{Уменьшенная программа}
\begin{tabular}{|c|r|l|l|} \hline
  Адрес & Код команды & Мнемоника & Комментарии \nl
  178   & E184        &           & A \nl
  179   & 0100        &           & B \nl
  17A   & 0200        &           & C \nl
  17B   & + 0200      & CLA       & Очистка аккумулятора \nl
  17C   & 6178        & SUB 0x178 & Вычитание A $(-A)$\nl
  17D   & 6179        & SUB 0x179 & Вычитание B $(-A -B)$ \nl
  17E   & 2183        & AND 0x17A & Логическое умножение C $((-A -B)\ \&\ C)$ \nl
  17F   & E184        & ST 0x181  & Сохранение \nl
  180   & 0100        & HLT       & Остановка \nl
  181   & 2183        &           & Результат $((-A -B)\ \&\ C)$ \nl
\end{tabular}
\section{Вывод}

В ходе данной лабораторной работы я познакомился с БЭВМ и научился манипулировать памятью и исполнять базовые программы.
\end{document}
