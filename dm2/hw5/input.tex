Для определения соответствия вершин с ρ(x) = ρ(y) = 5 попробуем связать с установленными вершинами из ρ(x) = ρ(y) = 8 и ρ(x) = ρ(y) = 6 и ρ(x) = ρ(y) = 3.
    
    $$\begin{tabular}{|c|c|c|c|c|c|c|c|} \hline X & Y \nl\begin{tikzpicture}[ -, node distance = 1cm, 
                    auto, main/.style = {}]\begin{scope}[node distance=0mm and 20mm]
    \node[main] (10)  {x10};
\node[main, below= of 10] (11)  {x11};
\node[main, below= of 11] (12)  {x12};
\node[main, left= of 10] (6)  {x6};
\node[main, below= of 6] (7)  {x7};
\node[main, below= of 7] (8)  {x8};
    \path[every node/.style={sloped,anchor=south,auto=false}]
(10) edge (7)
(10) edge (8)
(11) edge (6)
(12) edge (7)
(12) edge (8);\end{scope}  \end{tikzpicture} &\begin{tikzpicture}[ -, node distance = 1cm, 
                    auto, main/.style = {}]\begin{scope}[node distance=0mm and 20mm]
    \node[main] (9)  {y9};
\node[main, below= of 9] (5)  {y5};
\node[main, below= of 5] (8)  {y8};
\node[main, left= of 9] (2)  {y2};
\node[main, below= of 2] (3)  {y3};
\node[main, below= of 3] (10)  {y10};
    \path[every node/.style={sloped,anchor=south,auto=false}]
(5) edge (2)
(8) edge (3)
(8) edge (10)
(9) edge (3)
(9) edge (10);\end{scope}  \end{tikzpicture}   \nl \end{tabular}$$Анализ связей показывает следующее соответствие:
    $$\begin{tabular}{|c|c|c|c|c|c|c|c|} \hline X & Y \nl\rowcolor{Green} x6 & y5\nl 
\rowcolor{Green} x7 & y1\nl 
\rowcolor{Green} x8 & y12\nl 
 x10 & y7\nl 
 x11 & y2\nl 
 x12 & y3\nl   \end{tabular}$$
    
    Для определения соответствия вершин с ρ(x) = ρ(y) = 7 попробуем связать с установленными вершинами из ρ(x) = ρ(y) = 8 и ρ(x) = ρ(y) = 6 и ρ(x) = ρ(y) = 3 и ρ(x) = ρ(y) = 5.
    
    $$\begin{tabular}{|c|c|c|c|c|c|c|c|} \hline X & Y \nl\begin{tikzpicture}[ -, node distance = 1cm, 
                    auto, main/.style = {}]\begin{scope}[node distance=0mm and 20mm]
    \node[main] (10)  {x10};
\node[main, below= of 10] (11)  {x11};
\node[main, below= of 11] (12)  {x12};
\node[main, below= of 12] (6)  {x6};
\node[main, below= of 6] (7)  {x7};
\node[main, below= of 7] (8)  {x8};
\node[main, left= of 10] (1)  {x1};
\node[main, below= of 1] (4)  {x4};
    \path[every node/.style={sloped,anchor=south,auto=false}]
(6) edge (1)
(6) edge (4)
(7) edge (1)
(7) edge (4)
(8) edge (1)
(10) edge (1)
(10) edge (4)
(11) edge (1)
(11) edge (4);\end{scope}  \end{tikzpicture} &\begin{tikzpicture}[ -, node distance = 1cm, 
                    auto, main/.style = {}]\begin{scope}[node distance=0mm and 20mm]
    \node[main] (9)  {y9};
\node[main, below= of 9] (5)  {y5};
\node[main, below= of 5] (8)  {y8};
\node[main, below= of 8] (2)  {y2};
\node[main, below= of 2] (3)  {y3};
\node[main, below= of 3] (10)  {y10};
\node[main, left= of 9] (4)  {y4};
\node[main, below= of 4] (7)  {y7};
    \path[every node/.style={sloped,anchor=south,auto=false}]
(2) edge (4)
(2) edge (7)
(3) edge (7)
(5) edge (4)
(5) edge (7)
(9) edge (4)
(9) edge (7)
(10) edge (4)
(10) edge (7);\end{scope}  \end{tikzpicture}   \nl \end{tabular}$$Анализ связей показывает следующее соответствие:
    $$\begin{tabular}{|c|c|c|c|c|c|c|c|} \hline X & Y \nl\rowcolor{Green} x1 & y9\nl 
\rowcolor{Green} x4 & y4\nl 
 x6 & y5\nl 
 x7 & y1\nl 
 x8 & y12\nl 
 x10 & y7\nl 
 x11 & y2\nl 
 x12 & y3\nl   \end{tabular}$$
    
    Для определения соответствия вершин с ρ(x) = ρ(y) = 4 попробуем связать с установленными вершинами из ρ(x) = ρ(y) = 8 и ρ(x) = ρ(y) = 6 и ρ(x) = ρ(y) = 3 и ρ(x) = ρ(y) = 5 и ρ(x) = ρ(y) = 7.
    
    $$\begin{tabular}{|c|c|c|c|c|c|c|c|} \hline X & Y \nl\begin{tikzpicture}[ -, node distance = 1cm, 
                    auto, main/.style = {}]\begin{scope}[node distance=0mm and 20mm]
    \node[main] (10)  {x10};
\node[main, below= of 10] (11)  {x11};
\node[main, below= of 11] (12)  {x12};
\node[main, below= of 12] (6)  {x6};
\node[main, below= of 6] (7)  {x7};
\node[main, below= of 7] (8)  {x8};
\node[main, below= of 8] (1)  {x1};
\node[main, below= of 1] (4)  {x4};
\node[main, left= of 10] (2)  {x2};
\node[main, below= of 2] (3)  {x3};
\node[main, below= of 3] (5)  {x5};
\node[main, below= of 5] (9)  {x9};
    \path[every node/.style={sloped,anchor=south,auto=false}]
(1) edge (2)
(1) edge (5)
(4) edge (3)
(4) edge (5)
(4) edge (9)
(6) edge (5)
(8) edge (2)
(8) edge (9)
(10) edge (2)
(10) edge (3)
(10) edge (5)
(11) edge (9);\end{scope}  \end{tikzpicture} &\begin{tikzpicture}[ -, node distance = 1cm, 
                    auto, main/.style = {}]\begin{scope}[node distance=0mm and 20mm]
    \node[main] (9)  {y9};
\node[main, below= of 9] (5)  {y5};
\node[main, below= of 5] (8)  {y8};
\node[main, below= of 8] (2)  {y2};
\node[main, below= of 2] (3)  {y3};
\node[main, below= of 3] (10)  {y10};
\node[main, below= of 10] (4)  {y4};
\node[main, below= of 4] (7)  {y7};
\node[main, left= of 9] (1)  {y1};
\node[main, below= of 1] (6)  {y6};
\node[main, below= of 6] (11)  {y11};
\node[main, below= of 11] (12)  {y12};
    \path[every node/.style={sloped,anchor=south,auto=false}]
(2) edge (6)
(3) edge (1)
(3) edge (11)
(4) edge (1)
(4) edge (6)
(4) edge (12)
(5) edge (1)
(7) edge (6)
(7) edge (11)
(9) edge (6)
(9) edge (11)
(9) edge (12);\end{scope}  \end{tikzpicture}   \nl \end{tabular}$$Анализ связей показывает следующее соответствие:
    $$\begin{tabular}{|c|c|c|c|c|c|c|c|} \hline X & Y \nl x1 & y9\nl 
\rowcolor{Green} x2 & y6\nl 
\rowcolor{Green} x3 & y8\nl 
 x4 & y4\nl 
\rowcolor{Green} x5 & y11\nl 
 x6 & y5\nl 
 x7 & y1\nl 
 x8 & y12\nl 
\rowcolor{Green} x9 & y10\nl 
 x10 & y7\nl 
 x11 & y2\nl 
 x12 & y3\nl   \end{tabular}$$
    
    