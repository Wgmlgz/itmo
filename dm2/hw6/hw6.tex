\documentclass{article}
\usepackage{import}
\usepackage{tikz} 
\import{../../lib/latex/}{wgmlgz}
\begin{document}

\itmo[
  variant=-,
  labn=6,
  worktype=Домашняя работа,
  discipline=Дискретная математика,
  group=P3115,
  student=Владимир Мацюк,
  teacher=Поляков Владимир Иванович,
  logo=../../lib/img/itmo.png
]





Исходная таблица соединений R:




Построить знаковый граф.

Построим знаковый ориентированный граф, который будет показывать отношение различных процессов во время организации оплачиваемой стажировки в IT компании. Вершины графа – это сущности, связанные с метрополитеном, и процессы, проходящие в нем. Есть ребро $(u, v) \Rightarrow u $ влияет на v («+» если положительно, «–» если отрицательно).


\begin{center}
  \begin{tikzpicture}[ ->, thick, node distance={40mm}, main/.style = {draw, circle}]
    
\node[main,text width=2cm, align=center] (1) {Организация стажировки};
\node[main,text width=2cm, align=center] (2) [above of=1] {Затраты на проведение};
\node[main,text width=2.5cm, align=center] (9) [right of=1] {Количество проектов для стажеров};
\node[main,text width=2cm, align=center] (8) [right of=9] {Мотивированные стажеры };
\node[main,text width=2cm, align=center] (7) [below of=8] {Гранты на развитие IT индустрии};
\node[main,text width=2cm, align=center] (10) [right of=7] {Выделение бюджета на оплату стажировки};
\node[main,text width=2cm, align=center] (6) [above of=10] {Траты на подготовку};
\node[main,text width=2cm, align=center] (12) [below left of=10] {Удобные офисы для обучения};
\node[main,text width=2cm, align=center] (3) [above of=9] {Отставания от графика сотрудников};
\node[main,text width=2cm, align=center] (5) [above of=3] {Политическая обстановка в регионе};
\node[main,text width=2cm, align=center] (11) [right of=5] {Рекламная компания стажировки};
\node[main,text width=2cm, align=center] (4) [below of=11] {Возможность получить стажера в штат };


\path[every node/.style={sloped,anchor=south,auto=false}]
        (1) edge node {+} (2)            
        (1) edge node {-} (3)            
        (1) edge node {+} (9)             
        (3) edge node {+} (5)            
        (3) edge node {-} (4)            
        (5) edge node {-} (11)            
        (6) edge node {+} (10)            
        (6) edge node {+} (11)            
        (7) edge node {+} (1)            
        (7) edge node {+} (6)            
        (7) edge node {+} (10)            
        (8) edge node {+} (4)            
        (8) edge node {+} (7)            
        (9) edge node {+} (8)            
        (10) edge node {+} (12)            
        (11) edge node {+} (4)            
        (12) edge node {+} (1);
  \end{tikzpicture}
\end{center}

\end{document}