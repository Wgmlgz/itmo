\documentclass{article}
\usepackage{import}
\import{../../../lib/latex/}{wgmlgz}

\begin{document}

\itmo[
       variant=109,
       labn=7,
       worktype=Домашняя работа,
       discipline=Дискретная математика,
       group=P3115,
       student=Владимир Мацюк,
       teacher=Поляков Владимир Иванович,
       logo=../../../lib/img/itmo.png
]

\newcommand{\car}{\multicolumn{1}{c@{\hspace*{\tabcolsep}\makebox[0pt]{\curvearrowleft}}}{}}
\newcommand{\rcar}{\multicolumn{1}{c@{\hspace*{\tabcolsep}\makebox[0pt]{\curvearrowright}}}{}}
\newcommand{\ncar}{\multicolumn{1}{c@{\hspace*{\tabcolsep}\makebox[0pt]{}}}{}}
\newcommand{\SPACE}{\multicolumn{12}{c}{}}
\newcommand{\INT}{\multicolumn{5}{c}{\MM{Интерпретации}}}
\newcommand{\PLUS}{\multirow{2}{*}{+}}
\newcommand{\MINUS}{\multirow{2}{*}{-}}
\newcommand{\SIGN}{\multicolumn{2}{c}{\MM{Знаковая}}}
\newcommand{\USIGN}{\multicolumn{2}{c}{\MM{Беззнаковая}}}

\section{Числа}
$$
       \begin{array}{|c|c|}
              \hline
              A & 8,9 \nl
              B & 0,071 \nl
       \end{array}
$$
\section{Задание}

\begin{enumerate}
       \item Заданные числа А (множимое) и В (множитель) представить в фор-матах Ф1 и Ф2 с укороченной мантиссой (12 двоичных разрядов). Метод округления выбирается произвольно.
             Примечание: общее число разрядов в формате – 20.
       \item Выполнить операцию умножения операндов в формате Ф1, исполь-зуя метод ускоренного умножения мантисс на два разряда множите-ля.
       \item Выполнить операцию умножения операндов в формате Ф2, исполь-зуя метод ускоренного умножения мантисс на четыре разряда мно-жителя.
       \item Результаты представить в форматах операндов, перевести в десятич-ную систему счисления и проверить их правильность.
       \item Определить абсолютную и относительную погрешности результатов и обосновать их причину.
       \item Сравнить погрешности результатов аналогичных операций для форматов Ф1 и Ф2 и объяснить причины их сходства или различия.
             Варианты задания приведены в табл. 7 Приложения 1.
             
\end{enumerate}
\section{Решение}
\begin{enumerate}
       \item Формат Ф1 (число разрядов мантисы = 12):
             $$ A = 8,9_{10} = \textmd{8.E6}_{16} = \textmd{0.8E6}_{16} * 16^1 $$
             $$
                    \begin{array}{|c|c|c|}
                           \hline        
                           0 & 1000001 & 100011100110 \nl
                    \end{array}
             $$
             $$ A = 0,071_{10} = \textmd{0.123}_{16} = \textmd{0.123}_{16} * 16^0 $$
             $$
                    \begin{array}{|c|c|c|}
                           \hline        
                           0 & 1000000 & 000100100011 \nl
                    \end{array}
             $$
             $$ SignC = SignA \oplus SignB = 0 $$
             $$
                    \begin{array}{rcr}
                           X_A         = & \PLUS  & 1000001         \\
                           X_B         = &        & 1000000 \nl
                           X_A + X_B   = & \MINUS & 10000001        \\
                           d           = &        & 1000000     \nl
                           X_C         = &        & 1000001
                    \end{array}
             $$
             $$ P_C = 1 $$
             
  $$\begin{array}{c}(A>0,\ B>0) \\ 
\left.[A]_{пр.}=0111000;\ [B]_{пр.}=1011011\right. \\ 
 \\ \begin{array}{|c|c|c|c|p{5cm}|} \hline N\ \textup{шага} & \textup{Операнды\ и\ действия} & \begin{array}{c}\textup{СЧП}\ (\textup{старшие}\\\textup{разряды})\end{array} & \begin{array}{c}\textup{Множитель}\ и\\\textup{СЧП}\ (\textup{младшие}\\\textup{разряды})\end{array} & \textup{Пояснения} \\ \hline 
0 & \textup{СЧП} & 0\ 0\ 0\ 0\ 0\ 0\ 0 & 1\ 0\ 1\ 1\ 0\ 1\ \underline{1} & \textup{Обнуление\ старших\ разрядов\ СЧП} \\ \hline 
1 & \begin{array}{c} [A]_{\textup{доп.}}\\ \textup{СЧП}\\\textup{СЧП}\rightarrow\end{array} & \begin{array}{c} 0\ 1\ 1\ 1\ 0\ 0\ 0 \\ \hline 1\ 0\ 0\ 1\ 0\ 0\ 0 \\ 1\ 1\ 0\ 0\ 1\ 0\ 0 \end{array} & \begin{array}{c}  \\ |\ 1\ 0\ 1\ 1\ 0\ 1\ 1 \\ 0\ |\ 1\ 0\ 1\ 1\ 0\ \underline{1} \end{array} & \textup{Вычитание\ множимого\ из\ СЧП.;\ Сдвиг \ СЧП\ и\ множителя\ вправо} \\ \hline 
2 & \textup{СЧП} \rightarrow & 1\ 1\ 1\ 0\ 0\ 1\ 0 & 0\ 0\ |\ 1\ 0\ 1\ 1\ \underline{0} & \textup{При\ сдвиге\ младший\ разряд\ не\ изменился;\ Сдвиг\ СЧП\ и\ множителя\ вправо} \\ \hline 
3 & \begin{array}{c} [A]_{\textup{пр.}}\\ \textup{СЧП}\\\textup{СЧП}\rightarrow\end{array} & \begin{array}{c} 1\ 0\ 0\ 1\ 0\ 0\ 0 \\ \hline 0\ 1\ 0\ 1\ 0\ 1\ 0 \\ 0\ 0\ 1\ 0\ 1\ 0\ 1 \end{array} & \begin{array}{c}  \\ 0\ 0\ |\ 1\ 0\ 1\ 1\ 0 \\ 0\ 0\ 0\ |\ 1\ 0\ 1\ \underline{1} \end{array} & \textup{Cложение\ СЧП\ с\ множимым.;\ Сдвиг \ СЧП\ и\ множителя\ вправо} \\ \hline 
4 & \begin{array}{c} [A]_{\textup{доп.}}\\ \textup{СЧП}\\\textup{СЧП}\rightarrow\end{array} & \begin{array}{c} 0\ 1\ 1\ 1\ 0\ 0\ 0 \\ \hline 1\ 0\ 1\ 1\ 1\ 0\ 1 \\ 1\ 1\ 0\ 1\ 1\ 1\ 0 \end{array} & \begin{array}{c}  \\ 0\ 0\ 0\ |\ 1\ 0\ 1\ 1 \\ 1\ 0\ 0\ 0\ |\ 1\ 0\ \underline{1} \end{array} & \textup{Вычитание\ множимого\ из\ СЧП.;\ Сдвиг \ СЧП\ и\ множителя\ вправо} \\ \hline 
5 & \textup{СЧП} \rightarrow & 1\ 1\ 1\ 0\ 1\ 1\ 1 & 0\ 1\ 0\ 0\ 0\ |\ 1\ \underline{0} & \textup{При\ сдвиге\ младший\ разряд\ не\ изменился;\ Сдвиг\ СЧП\ и\ множителя\ вправо} \\ \hline 
6 & \begin{array}{c} [A]_{\textup{пр.}}\\ \textup{СЧП}\\\textup{СЧП}\rightarrow\end{array} & \begin{array}{c} 1\ 0\ 0\ 1\ 0\ 0\ 0 \\ \hline 0\ 1\ 0\ 1\ 1\ 1\ 1 \\ 0\ 0\ 1\ 0\ 1\ 1\ 1 \end{array} & \begin{array}{c}  \\ 0\ 1\ 0\ 0\ 0\ |\ 1\ 0 \\ 1\ 0\ 1\ 0\ 0\ 0\ |\ \underline{1} \end{array} & \textup{Cложение\ СЧП\ с\ множимым.;\ Сдвиг \ СЧП\ и\ множителя\ вправо} \\ \hline 
7 & \begin{array}{c} [A]_{\textup{доп.}}\\ \textup{СЧП}\\\textup{СЧП}\rightarrow\end{array} & \begin{array}{c} 0\ 1\ 1\ 1\ 0\ 0\ 0 \\ \hline 1\ 0\ 0\ 1\ 1\ 1\ 1 \\ 0\ 1\ 0\ 0\ 1\ 1\ 1 \end{array} & \begin{array}{c}  \\ 1\ 0\ 1\ 0\ 0\ 0\ |\ 1 \\ 1\ 1\ 0\ 1\ 0\ 0\ \underline{0}\ | \end{array} & \textup{Вычитание\ множимого\ из\ СЧП.;\ Сдвиг \ СЧП\ и\ множителя\ вправо} \\ \hline 
 \end{array} \\
 \\ 
 \\  \left[C\right]_{\textup{пр.}}=01001111101000_2 = 5096_{10}\end{array}$$
  $$\begin{array}{c}(A<0,\ B>0) \\ 
\left.[-A]_{доп.}=1001000;\ [B]_{пр.}=1011011\right. \\ 
 \\ \begin{array}{|c|c|c|c|p{5cm}|} \hline N\ \textup{шага} & \textup{Операнды\ и\ действия} & \begin{array}{c}\textup{СЧП}\ (\textup{старшие}\\\textup{разряды})\end{array} & \begin{array}{c}\textup{Множитель}\ и\\\textup{СЧП}\ (\textup{младшие}\\\textup{разряды})\end{array} & \textup{Пояснения} \\ \hline 
0 & \textup{СЧП} & 0\ 0\ 0\ 0\ 0\ 0\ 0 & 1\ 0\ 1\ 1\ 0\ 1\ \underline{1} & \textup{Обнуление\ старших\ разрядов\ СЧП} \\ \hline 
1 & \begin{array}{c} [-A]_{\textup{пр.}}\\ \textup{СЧП}\\\textup{СЧП}\rightarrow\end{array} & \begin{array}{c} 0\ 1\ 1\ 1\ 0\ 0\ 0 \\ \hline 0\ 1\ 1\ 1\ 0\ 0\ 0 \\ 0\ 0\ 1\ 1\ 1\ 0\ 0 \end{array} & \begin{array}{c}  \\ |\ 1\ 0\ 1\ 1\ 0\ 1\ 1 \\ 0\ |\ 1\ 0\ 1\ 1\ 0\ \underline{1} \end{array} & \textup{Cложение\ СЧП\ с\ множимым.;\ Сдвиг \ СЧП\ и\ множителя\ вправо} \\ \hline 
2 & \textup{СЧП} \rightarrow & 0\ 0\ 0\ 1\ 1\ 1\ 0 & 0\ 0\ |\ 1\ 0\ 1\ 1\ \underline{0} & \textup{При\ сдвиге\ младший\ разряд\ не\ изменился;\ Сдвиг\ СЧП\ и\ множителя\ вправо} \\ \hline 
3 & \begin{array}{c} [-A]_{\textup{доп.}}\\ \textup{СЧП}\\\textup{СЧП}\rightarrow\end{array} & \begin{array}{c} 1\ 0\ 0\ 1\ 0\ 0\ 0 \\ \hline 1\ 0\ 1\ 0\ 1\ 1\ 0 \\ 1\ 1\ 0\ 1\ 0\ 1\ 1 \end{array} & \begin{array}{c}  \\ 0\ 0\ |\ 1\ 0\ 1\ 1\ 0 \\ 0\ 0\ 0\ |\ 1\ 0\ 1\ \underline{1} \end{array} & \textup{Вычитание\ множимого\ из\ СЧП.;\ Сдвиг \ СЧП\ и\ множителя\ вправо} \\ \hline 
4 & \begin{array}{c} [-A]_{\textup{пр.}}\\ \textup{СЧП}\\\textup{СЧП}\rightarrow\end{array} & \begin{array}{c} 0\ 1\ 1\ 1\ 0\ 0\ 0 \\ \hline 0\ 1\ 0\ 0\ 0\ 1\ 1 \\ 0\ 0\ 1\ 0\ 0\ 0\ 1 \end{array} & \begin{array}{c}  \\ 0\ 0\ 0\ |\ 1\ 0\ 1\ 1 \\ 1\ 0\ 0\ 0\ |\ 1\ 0\ \underline{1} \end{array} & \textup{Cложение\ СЧП\ с\ множимым.;\ Сдвиг \ СЧП\ и\ множителя\ вправо} \\ \hline 
5 & \textup{СЧП} \rightarrow & 0\ 0\ 0\ 1\ 0\ 0\ 0 & 1\ 1\ 0\ 0\ 0\ |\ 1\ \underline{0} & \textup{При\ сдвиге\ младший\ разряд\ не\ изменился;\ Сдвиг\ СЧП\ и\ множителя\ вправо} \\ \hline 
6 & \begin{array}{c} [-A]_{\textup{доп.}}\\ \textup{СЧП}\\\textup{СЧП}\rightarrow\end{array} & \begin{array}{c} 1\ 0\ 0\ 1\ 0\ 0\ 0 \\ \hline 1\ 0\ 1\ 0\ 0\ 0\ 0 \\ 1\ 1\ 0\ 1\ 0\ 0\ 0 \end{array} & \begin{array}{c}  \\ 1\ 1\ 0\ 0\ 0\ |\ 1\ 0 \\ 0\ 1\ 1\ 0\ 0\ 0\ |\ \underline{1} \end{array} & \textup{Вычитание\ множимого\ из\ СЧП.;\ Сдвиг \ СЧП\ и\ множителя\ вправо} \\ \hline 
7 & \begin{array}{c} [-A]_{\textup{пр.}}\\ \textup{СЧП}\\\textup{СЧП}\rightarrow\end{array} & \begin{array}{c} 0\ 1\ 1\ 1\ 0\ 0\ 0 \\ \hline 0\ 1\ 1\ 0\ 0\ 0\ 0 \\ 1\ 0\ 1\ 1\ 0\ 0\ 0 \end{array} & \begin{array}{c}  \\ 0\ 1\ 1\ 0\ 0\ 0\ |\ 1 \\ 0\ 0\ 1\ 1\ 0\ 0\ \underline{0}\ | \end{array} & \textup{Cложение\ СЧП\ с\ множимым.;\ Сдвиг \ СЧП\ и\ множителя\ вправо} \\ \hline 
 \end{array} \\
 \\ 
 \\  \left[C\right]_{\textup{доп}.}=10110000011000_2  \\
  \left[C\right]_{\textup{пр}.}= -01001111101000_2 = -5096_{10}\end{array}$$
  $$\begin{array}{c}(A>0,\ B<0) \\ 
\left.[A]_{пр.}=0111000;\ [-B]_{доп.}=0100101\right. \\ 
 \\ \begin{array}{|c|c|c|c|p{5cm}|} \hline N\ \textup{шага} & \textup{Операнды\ и\ действия} & \begin{array}{c}\textup{СЧП}\ (\textup{старшие}\\\textup{разряды})\end{array} & \begin{array}{c}\textup{Множитель}\ и\\\textup{СЧП}\ (\textup{младшие}\\\textup{разряды})\end{array} & \textup{Пояснения} \\ \hline 
0 & \textup{СЧП} & 0\ 0\ 0\ 0\ 0\ 0\ 0 & 0\ 1\ 0\ 0\ 1\ 0\ \underline{1} & \textup{Обнуление\ старших\ разрядов\ СЧП} \\ \hline 
1 & \begin{array}{c} [A]_{\textup{доп.}}\\ \textup{СЧП}\\\textup{СЧП}\rightarrow\end{array} & \begin{array}{c} 0\ 1\ 1\ 1\ 0\ 0\ 0 \\ \hline 1\ 0\ 0\ 1\ 0\ 0\ 0 \\ 1\ 1\ 0\ 0\ 1\ 0\ 0 \end{array} & \begin{array}{c}  \\ |\ 0\ 1\ 0\ 0\ 1\ 0\ 1 \\ 0\ |\ 0\ 1\ 0\ 0\ 1\ \underline{0} \end{array} & \textup{Вычитание\ множимого\ из\ СЧП.;\ Сдвиг \ СЧП\ и\ множителя\ вправо} \\ \hline 
2 & \begin{array}{c} [A]_{\textup{пр.}}\\ \textup{СЧП}\\\textup{СЧП}\rightarrow\end{array} & \begin{array}{c} 1\ 0\ 0\ 1\ 0\ 0\ 0 \\ \hline 0\ 0\ 1\ 1\ 1\ 0\ 0 \\ 0\ 0\ 0\ 1\ 1\ 1\ 0 \end{array} & \begin{array}{c}  \\ 0\ |\ 0\ 1\ 0\ 0\ 1\ 0 \\ 0\ 0\ |\ 0\ 1\ 0\ 0\ \underline{1} \end{array} & \textup{Cложение\ СЧП\ с\ множимым.;\ Сдвиг \ СЧП\ и\ множителя\ вправо} \\ \hline 
3 & \begin{array}{c} [A]_{\textup{доп.}}\\ \textup{СЧП}\\\textup{СЧП}\rightarrow\end{array} & \begin{array}{c} 0\ 1\ 1\ 1\ 0\ 0\ 0 \\ \hline 1\ 0\ 1\ 0\ 1\ 1\ 0 \\ 1\ 1\ 0\ 1\ 0\ 1\ 1 \end{array} & \begin{array}{c}  \\ 0\ 0\ |\ 0\ 1\ 0\ 0\ 1 \\ 0\ 0\ 0\ |\ 0\ 1\ 0\ \underline{0} \end{array} & \textup{Вычитание\ множимого\ из\ СЧП.;\ Сдвиг \ СЧП\ и\ множителя\ вправо} \\ \hline 
4 & \begin{array}{c} [A]_{\textup{пр.}}\\ \textup{СЧП}\\\textup{СЧП}\rightarrow\end{array} & \begin{array}{c} 1\ 0\ 0\ 1\ 0\ 0\ 0 \\ \hline 0\ 1\ 0\ 0\ 0\ 1\ 1 \\ 0\ 0\ 1\ 0\ 0\ 0\ 1 \end{array} & \begin{array}{c}  \\ 0\ 0\ 0\ |\ 0\ 1\ 0\ 0 \\ 1\ 0\ 0\ 0\ |\ 0\ 1\ \underline{0} \end{array} & \textup{Cложение\ СЧП\ с\ множимым.;\ Сдвиг \ СЧП\ и\ множителя\ вправо} \\ \hline 
5 & \textup{СЧП} \rightarrow & 0\ 0\ 0\ 1\ 0\ 0\ 0 & 1\ 1\ 0\ 0\ 0\ |\ 0\ \underline{1} & \textup{При\ сдвиге\ младший\ разряд\ не\ изменился;\ Сдвиг\ СЧП\ и\ множителя\ вправо} \\ \hline 
6 & \begin{array}{c} [A]_{\textup{доп.}}\\ \textup{СЧП}\\\textup{СЧП}\rightarrow\end{array} & \begin{array}{c} 0\ 1\ 1\ 1\ 0\ 0\ 0 \\ \hline 1\ 0\ 1\ 0\ 0\ 0\ 0 \\ 1\ 1\ 0\ 1\ 0\ 0\ 0 \end{array} & \begin{array}{c}  \\ 1\ 1\ 0\ 0\ 0\ |\ 0\ 1 \\ 0\ 1\ 1\ 0\ 0\ 0\ |\ \underline{0} \end{array} & \textup{Вычитание\ множимого\ из\ СЧП.;\ Сдвиг \ СЧП\ и\ множителя\ вправо} \\ \hline 
7 & \begin{array}{c} [A]_{\textup{пр.}}\\ \textup{СЧП}\\\textup{СЧП}\rightarrow\end{array} & \begin{array}{c} 1\ 0\ 0\ 1\ 0\ 0\ 0 \\ \hline 0\ 1\ 1\ 0\ 0\ 0\ 0 \\ 1\ 0\ 1\ 1\ 0\ 0\ 0 \end{array} & \begin{array}{c}  \\ 0\ 1\ 1\ 0\ 0\ 0\ |\ 0 \\ 0\ 0\ 1\ 1\ 0\ 0\ \underline{0}\ | \end{array} & \textup{Cложение\ СЧП\ с\ множимым.;\ Сдвиг \ СЧП\ и\ множителя\ вправо} \\ \hline 
 \end{array} \\
 \\ 
 \\  \left[C\right]_{\textup{доп}.}=10110000011000_2  \\
  \left[C\right]_{\textup{пр}.}= -01001111101000_2 = -5096_{10}\end{array}$$
  $$\begin{array}{c}(A<0,\ B<0) \\ 
\left.[-A]_{доп.}=1001000;\ [-B]_{доп.}=0100101\right. \\ 
 \\ \begin{array}{|c|c|c|c|p{5cm}|} \hline N\ \textup{шага} & \textup{Операнды\ и\ действия} & \begin{array}{c}\textup{СЧП}\ (\textup{старшие}\\\textup{разряды})\end{array} & \begin{array}{c}\textup{Множитель}\ и\\\textup{СЧП}\ (\textup{младшие}\\\textup{разряды})\end{array} & \textup{Пояснения} \\ \hline 
0 & \textup{СЧП} & 0\ 0\ 0\ 0\ 0\ 0\ 0 & 0\ 1\ 0\ 0\ 1\ 0\ \underline{1} & \textup{Обнуление\ старших\ разрядов\ СЧП} \\ \hline 
1 & \begin{array}{c} [-A]_{\textup{пр.}}\\ \textup{СЧП}\\\textup{СЧП}\rightarrow\end{array} & \begin{array}{c} 0\ 1\ 1\ 1\ 0\ 0\ 0 \\ \hline 0\ 1\ 1\ 1\ 0\ 0\ 0 \\ 0\ 0\ 1\ 1\ 1\ 0\ 0 \end{array} & \begin{array}{c}  \\ |\ 0\ 1\ 0\ 0\ 1\ 0\ 1 \\ 0\ |\ 0\ 1\ 0\ 0\ 1\ \underline{0} \end{array} & \textup{Cложение\ СЧП\ с\ множимым.;\ Сдвиг \ СЧП\ и\ множителя\ вправо} \\ \hline 
2 & \begin{array}{c} [-A]_{\textup{доп.}}\\ \textup{СЧП}\\\textup{СЧП}\rightarrow\end{array} & \begin{array}{c} 1\ 0\ 0\ 1\ 0\ 0\ 0 \\ \hline 1\ 1\ 0\ 0\ 1\ 0\ 0 \\ 1\ 1\ 1\ 0\ 0\ 1\ 0 \end{array} & \begin{array}{c}  \\ 0\ |\ 0\ 1\ 0\ 0\ 1\ 0 \\ 0\ 0\ |\ 0\ 1\ 0\ 0\ \underline{1} \end{array} & \textup{Вычитание\ множимого\ из\ СЧП.;\ Сдвиг \ СЧП\ и\ множителя\ вправо} \\ \hline 
3 & \begin{array}{c} [-A]_{\textup{пр.}}\\ \textup{СЧП}\\\textup{СЧП}\rightarrow\end{array} & \begin{array}{c} 0\ 1\ 1\ 1\ 0\ 0\ 0 \\ \hline 0\ 1\ 0\ 1\ 0\ 1\ 0 \\ 0\ 0\ 1\ 0\ 1\ 0\ 1 \end{array} & \begin{array}{c}  \\ 0\ 0\ |\ 0\ 1\ 0\ 0\ 1 \\ 0\ 0\ 0\ |\ 0\ 1\ 0\ \underline{0} \end{array} & \textup{Cложение\ СЧП\ с\ множимым.;\ Сдвиг \ СЧП\ и\ множителя\ вправо} \\ \hline 
4 & \begin{array}{c} [-A]_{\textup{доп.}}\\ \textup{СЧП}\\\textup{СЧП}\rightarrow\end{array} & \begin{array}{c} 1\ 0\ 0\ 1\ 0\ 0\ 0 \\ \hline 1\ 0\ 1\ 1\ 1\ 0\ 1 \\ 1\ 1\ 0\ 1\ 1\ 1\ 0 \end{array} & \begin{array}{c}  \\ 0\ 0\ 0\ |\ 0\ 1\ 0\ 0 \\ 1\ 0\ 0\ 0\ |\ 0\ 1\ \underline{0} \end{array} & \textup{Вычитание\ множимого\ из\ СЧП.;\ Сдвиг \ СЧП\ и\ множителя\ вправо} \\ \hline 
5 & \textup{СЧП} \rightarrow & 1\ 1\ 1\ 0\ 1\ 1\ 1 & 0\ 1\ 0\ 0\ 0\ |\ 0\ \underline{1} & \textup{При\ сдвиге\ младший\ разряд\ не\ изменился;\ Сдвиг\ СЧП\ и\ множителя\ вправо} \\ \hline 
6 & \begin{array}{c} [-A]_{\textup{пр.}}\\ \textup{СЧП}\\\textup{СЧП}\rightarrow\end{array} & \begin{array}{c} 0\ 1\ 1\ 1\ 0\ 0\ 0 \\ \hline 0\ 1\ 0\ 1\ 1\ 1\ 1 \\ 0\ 0\ 1\ 0\ 1\ 1\ 1 \end{array} & \begin{array}{c}  \\ 0\ 1\ 0\ 0\ 0\ |\ 0\ 1 \\ 1\ 0\ 1\ 0\ 0\ 0\ |\ \underline{0} \end{array} & \textup{Cложение\ СЧП\ с\ множимым.;\ Сдвиг \ СЧП\ и\ множителя\ вправо} \\ \hline 
7 & \begin{array}{c} [-A]_{\textup{доп.}}\\ \textup{СЧП}\\\textup{СЧП}\rightarrow\end{array} & \begin{array}{c} 1\ 0\ 0\ 1\ 0\ 0\ 0 \\ \hline 1\ 0\ 0\ 1\ 1\ 1\ 1 \\ 0\ 1\ 0\ 0\ 1\ 1\ 1 \end{array} & \begin{array}{c}  \\ 1\ 0\ 1\ 0\ 0\ 0\ |\ 0 \\ 1\ 1\ 0\ 1\ 0\ 0\ \underline{0}\ | \end{array} & \textup{Вычитание\ множимого\ из\ СЧП.;\ Сдвиг \ СЧП\ и\ множителя\ вправо} \\ \hline 
 \end{array} \\
 \\ 
 \\  \left[C\right]_{\textup{пр.}}=01001111101000_2 = 5096_{10}\end{array}$$

             $$
                    \begin{array}{c}
                           C^* = 0.0A1D_{16} * 16^1 = 0.A1D_{16}  = 0.632\\
                           \Delta C = C_T - C^* = 0.6319 - 0.632 = -0.0001 \\
                           \delta C = \left|\frac{\Delta C}{C_T}\right| \cdot 100\% = \left|\frac{0.0001}{0.6319}\right| \cdot 100\% = 0.0158 \\
                    \end{array}
             $$
       \item Формат Ф2
              $$ A = 8,9_{10} = 1000.1110011_{2} = 0.100011100110_{2} * 2^4 $$
              $$
                     \begin{array}{|c|c|c|}
                            \hline        
                            0 & 10000100 & 00011100110 \nl
                     \end{array}
              $$
              $$ A = 0,071_{10} = 0.0001001000101101_{2} = 0.100100010110_{2} * 2^{-3} $$
              $$
                     \begin{array}{|c|c|c|}
                            \hline        
                            0 & 10000000 & 00100010110 \nl
                     \end{array}
              $$
              $$ SignC = SignA \oplus SignB = 0 $$
              $$
                    \begin{array}{rcr}
                           X_A         = & \PLUS  & 1000100         \\
                           X_B         = &        & 1000000 \nl
                           X_A + X_B   = & \MINUS & 10000100        \\
                           d           = &        & 1000000     \nl
                           X_C         = &        & 1000100
                    \end{array}
             $$
             $$ P_C = 4 $$
              
  $$\begin{array}{c}\begin{array}{|c|c|c|c|} \hline \begin{array}{c}\textup{Оперaнды}\end{array} & \begin{array}{c}\textup{СЧП\ (стaршие\ рaзряды)}\end{array} & \begin{array}{c}\textup{\ В/СЧП\ (млaдшие\ рaзряды)}\end{array} & \begin{array}{c}\textup{Признaк\ коррекции}\end{array} \\ \hline 
\textup{СЧП} & 00000000000000000 & 100100010110 & 0 \\ \hline 
\begin{array}{c} \left[4A \right]_{\textup{}} \\  \left[+2A\right]_{\textup{}} \\ \textup{СЧП} \\ \textup{СЧП\ \rightarrow\ 4}\end{array} & \begin{array}{c}00010001110011000 \\ 00001000111001100 \\ 00011010101100100 \\ 00000001101010110\end{array} & \begin{array}{c} \\  \\ 100100010110 \\ 010010010001\end{array} & 0 \\ \hline 
\begin{array}{c} \left[0A \right]_{\textup{}} \\  \left[+1A\right]_{\textup{}} \\ \textup{СЧП} \\ \textup{СЧП\ \rightarrow\ 4}\end{array} & \begin{array}{c}00000000000000000 \\ 00000100011100110 \\ 00000110000111100 \\ 00000000011000011\end{array} & \begin{array}{c} \\  \\ 010010010001 \\ 110001001001\end{array} & 0 \\ \hline 
\begin{array}{c} \left[8A \right]_{\textup{}} \\  \left[+A \right]_{\textup{}} \\ \textup{СЧП} \\ \textup{СЧП\ \rightarrow\ 4}\end{array} & \begin{array}{c}00100011100110000 \\ 00000100011100110 \\ 00101000011011001 \\ 00000010100001101\end{array} & \begin{array}{c} \\  \\ 110001001001 \\ 100111000100\end{array} & 0 \\ \hline 
 \end{array} \\
 \\ 
 \\ \end{array}$$

              $$
                     \begin{array}{c}
                            C^* = 0.000010100001101_{2} * 2^4 = 0.10100001101_{2} = 0.6313\\
                            \Delta C = C_T - C^* = 0.6319 - 0.6313 = 0.00055 \\
                            \delta C = \left|\frac{\Delta C}{C_T}\right| \cdot 100\% = \left|\frac{0.00055}{0.6319}\right| \cdot 100\% = 0.087 \\
                     \end{array}
              $$
\end{enumerate}
\end{document}
