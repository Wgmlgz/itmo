\documentclass{article}
\usepackage{import}
\import{../../../lib/latex/}{wgmlgz}

\begin{document}

\itmo[
  variant=110,
  labn=2,
  worktype=Домашняя работа,
  discipline=Дискретная математика,
  group=P3115,
  student=Владимир Мацюк,
  teacher=Поляков Владимир Иванович,
  logo=../../../lib/img/itmo.png
]

\section{Числа}
$$
  \begin{array}{|c|c|}
    \hline
    A & 36 \nl
    B & 73 \nl
  \end{array}
$$
\section{Задание}
\begin{enumerate}
  \item Для заданных чисел А и В выполнить операцию знакового сложения со всеми комбинациями знаков операндов. Для каждого примера:
        \begin{enumerate}
          \item проставить межразрядные переносы, возникающие при сложении;
          \item дать знаковую интерпретацию (Зн) операндов и результатов. При получении отрицательного результата предварительно преобразовать его из дополнительного кода в прямой;
          \item дать беззнаковую интерпретацию (БзИ) операндов и результатов, при получении неверного результата пояснить причину его возникновения;
          \item показать значения арифметических флагов.
        \end{enumerate}
        \newcommand{\ncar}{\multicolumn{1}{c@{\hspace*{\tabcolsep}\makebox[0pt]{}}}{}}
        \newcommand{\car}{\multicolumn{1}{c@{\hspace*{\tabcolsep}\makebox[0pt]{\curvearrowleft}}}{}}
        \newcommand{\SPACE}{\multicolumn{12}{c}{}}
        \newcommand{\INT}{\multicolumn{5}{c}{\MM{Интерпретации}}}
        \newcommand{\PLUS}{\multirow{2}{*}{+}}
        \newcommand{\SIGN}{\multicolumn{2}{c}{\MM{Знаковая}}}
        \newcommand{\USIGN}{\multicolumn{2}{c}{\MM{Беззнаковая}}}
        $$ A = 36_{10} = 100100_2,\ B = 73_{10} = 1001001_2 $$
        $$ A > 0,\ B > 0 $$
        $$\begin{array}{ccc|cccccccccccccc}
            \SPACE & \INT                                                                                     \\
                   &              & \ncar &   &   &   &   &   &   &   &  & \SIGN &     & \USIGN               \\
            \PLUS  & A_{\MM{пр.}} & 0     & 0 & 1 & 0 & 0 & 1 & 0 & 0 &  & \PLUS & 36  &        & \PLUS & 36  \\
                   & B_{\MM{пр.}} & 0     & 1 & 0 & 0 & 1 & 0 & 0 & 1 &  &       & 73  &        &       & 73  \\  \cline{2-2} \cline{5-10} \cline{12-13} \cline{15-16}
                   & C_{\MM{пр.}} & 0     & 1 & 1 & 0 & 1 & 1 & 0 & 1 &  &       & 109 &        &       & 109 \\
          \end{array}
        $$
        $$ CF=0,\ ZF=0,\ PF=0,\ AF=0,\ SF=0,\	OF=0 $$

        $$ B_{\MM{доп.}} = 10110111_2 $$
        $$ A > 0,\ B < 0 $$
        $$\begin{array}{ccc|cccccccccccccc}
            \SPACE & \INT                                                                                            \\
                   &               & \ncar & \car &   &   & \car &   &   &   &  & \SIGN &     & \USIGN               \\
            \PLUS  & A_{\MM{пр.}}  & 0     & 0    & 1 & 0 & 0    & 1 & 0 & 0 &  & \PLUS & 36  &        & \PLUS & 36  \\
                   & B_{\MM{доп.}} & 1     & 0    & 1 & 1 & 0    & 1 & 1 & 1 &  &       & -73 &        &       & 183 \\  \cline{2-2} \cline{5-10} \cline{12-13} \cline{15-16}
                   & C_{\MM{доп.}} & 1     & 1    & 0 & 1 & 1    & 0 & 1 & 1 &  &       &     &        &       & 219 \\  \cline{2-2} \cline{5-10} \cline{12-13} \cline{15-16}
                   & C_{\MM{пр.}}  & 1     & 0    & 1 & 0 & 0    & 1 & 0 & 1 &  &       & -37 &        &       &     \\
          \end{array}
        $$
        $$ CF=0,\ ZF=0,\ PF=1,\ AF=0,\ SF=1,\	OF=0 $$

        $$ A_{\MM{доп.}} = 11011100_2 $$
        $$ A < 0,\ B > 0 $$
        $$\begin{array}{ccc|cccccccccccccc}
            \SPACE & \INT                                                                                           \\
                   & \car          & \car &   & \car & \car &   &   &   &   &  & \SIGN &     & \USIGN               \\
            \PLUS  & A_{\MM{доп.}} & 1    & 1 & 0    & 1    & 1 & 1 & 0 & 0 &  & \PLUS & -36 &        & \PLUS & 220 \\
                   & B_{\MM{пр.}}  & 0    & 1 & 0    & 0    & 1 & 0 & 0 & 1 &  &       & 73  &        &       & 73  \\  \cline{2-2} \cline{5-10} \cline{12-13} \cline{15-16}
                   & C_{\MM{пр.}}  & 0    & 0 & 1    & 0    & 0 & 1 & 0 & 1 &  &       & 37  &        &       & 293 \\
          \end{array}
        $$
        $$ CF=1,\ ZF=0,\ PF=0,\ AF=0,\ SF=0,\	OF=0 $$
        Для БзИ результат неверен вследствие возникающего переноса из старшего разряда. Вес этого переноса составляет 256.

        $$ A_{\MM{доп.}} = 11011100_2,\  B_{\MM{доп.}} = 10110111_2 $$
        $$ A < 0,\ B > 0 $$
        $$\begin{array}{ccc|cccccccccccccc}
            \SPACE & \INT                                                                                                  \\
                   & \car          & \car & \car & \car & \car & \car &   &   &   &  & \SIGN &      & \USIGN               \\
            \PLUS  & A_{\MM{доп.}} & 1    & 1    & 0    & 1    & 1    & 1 & 0 & 0 &  & \PLUS & -36  &        & \PLUS & 220 \\
                   & B_{\MM{доп.}} & 1    & 0    & 1    & 1    & 0    & 1 & 1 & 1 &  &       & -73  &        &       & 183 \\  \cline{2-2} \cline{5-10} \cline{12-13} \cline{15-16}
                   & C_{\MM{доп.}} & 1    & 0    & 0    & 1    & 0    & 0 & 1 & 1 &  &       &      &        &       & 403 \\  \cline{2-2} \cline{5-10} \cline{12-13} \cline{15-16}
                   & C_{\MM{пр.}}  & 1    & 1    & 1    & 0    & 1    & 1 & 0 & 1 &  &       & -109 &        &       &     \\
          \end{array}
        $$
        $$ CF=1,\ ZF=0,\ PF=1,\ AF=1,\ SF=1,\	OF=0 $$
        Для БзИ результат неверен вследствие возникающего переноса из старшего разряда. Вес этого переноса составляет 256.

  \item Cохранив значение первого операнда А, выбрать такое значение В, чтобы в операции сложения с одинаковыми знаками имел место особый случай переполнения формата. Выполнить два примера, иллюстрирующие эти случаи, для каждого из них проделать пункты a, b, c, d.
        $$ A + B > 128\ \Rightarrow \ 127>B>128 - A $$
        $$ B = 96_{10} = 1100000_2 $$
        $$\begin{array}{ccc|cccccccccccccc}
            \SPACE & \INT                                                                                        \\
                   &               & \car & \car &   &   &   &   &   &   &  & \SIGN &     & \USIGN               \\
            \PLUS  & A_{\MM{пр.}}  & 0    & 0    & 1 & 0 & 0 & 1 & 0 & 0 &  & \PLUS & 36  &        & \PLUS & 36  \\
                   & B_{\MM{пр.}}  & 0    & 1    & 1 & 0 & 0 & 0 & 0 & 0 &  &       & 96  &        &       & 96  \\  \cline{2-2} \cline{5-10} \cline{12-13} \cline{15-16}
                   & C_{\MM{доп.}} & 1    & 0    & 0 & 0 & 0 & 1 & 0 & 0 &  &       &     &        &       & 132 \\  \cline{2-2} \cline{5-10} \cline{12-13} \cline{15-16}
                   & C_{\MM{пр.}}  & 0    & 0    & 1 & 0 & 0 & 1 & 0 & 1 &  &       & -37 &        &       &     \\
          \end{array}
        $$
        $$ CF=0,\ ZF=0,\ PF=0,\ AF=1,\ SF=1,\	OF=0 $$
        Для знакового сложения результат я	вляется некорректным вследствие переполнения формата.

        $$ A_{\MM{доп.}} = 11011100_2,\  B_{\MM{доп.}} = 10100000_2  $$
        $$\begin{array}{ccc|cccccccccccccc}
            \SPACE & \INT                                                                                       \\
                   & \car          & \ncar &   &   &   &   &   &   &   &  & \SIGN &      & \USIGN               \\
            \PLUS  & A_{\MM{доп.}} & 1     & 1 & 0 & 1 & 1 & 1 & 0 & 0 &  & \PLUS & -36  &        & \PLUS & 220 \\
                   & B_{\MM{доп.}} & 1     & 0 & 1 & 0 & 0 & 0 & 0 & 0 &  &       & -96  &        &       & 160 \\  \cline{2-2} \cline{5-10} \cline{12-13} \cline{15-16}
                   & C_{\MM{пр.}}  & 0     & 1 & 1 & 1 & 1 & 1 & 0 & 0 &  &       & -132 &        &       & 380 \\
          \end{array}
        $$
        $$ CF=1 ,\ ZF=0,\ PF=0,\ AF=1,\ SF=1,\	OF=1 $$
        Для знаковой интерпретации результат неверен вследствие возникающего переполнения, для беззнаковой интерпретации результат неверен вследствие возникающего переноса из старшего разряда.
  \item Сохранив операнд В, подавать такое значение операнда А, чтобы при сложении положительных операндов имело место переполнение формата, а при сложении отрицательных операндов результат был бы корректен. Выполнить два примера, иллюстрирующие этот случай. Для каждого из них проделать пункты a, b, c, d.
        $$ B = 73 $$
        $$ A + B = 128\ \Rightarrow \ A = 128 - 73 = 55_{10} = 110111_2 $$
        $$\begin{array}{ccc|cccccccccccccc}
            \SPACE & \INT                                                                                                         \\
                   &               & \car & \car & \car & \car & \car & \car & \car &   &  & \SIGN &       & \USIGN               \\
            \PLUS  & A_{\MM{пр.}}  & 0    & 0    & 1    & 1    & 0    & 1    & 1    & 1 &  & \PLUS & 55    &        & \PLUS & 55  \\
                   & B_{\MM{пр.}}  & 0    & 1    & 0    & 0    & 1    & 0    & 0    & 1 &  &       & 73    &        &       & 73  \\  \cline{2-2} \cline{5-10} \cline{12-13} \cline{15-16}
                   & C_{\MM{доп.}} & 1    & 0    & 0    & 0    & 0    & 0    & 0    & 0 &  &       &       &        &       & 128 \\  \cline{2-2} \cline{5-10} \cline{12-13} \cline{15-16}
                   & C_{\MM{пр.}}  & 1    & 0    & 0    & 0    & 0    & 0    & 0    & 0 &  &       & -128? &        &       &     \\
          \end{array}
        $$
        $$ CF=0 ,\ ZF=0,\ PF=0,\ AF=1,\ SF=1,\	OF=1 $$

        $$\begin{array}{ccc|cccccccccccccc}
            \SPACE & \INT                                                                                                         \\
                   & \car          & \car & \car & \car & \car & \car & \car & \car &   &  & \SIGN &      & \USIGN                \\
            \PLUS  & A_{\MM{пр.}}  & 0    & 0    & 1    & 1    & 0    & 1    & 1    & 1 &  & \PLUS & -55  &        & \PLUS & 55   \\
                   & B_{\MM{пр.}}  & 0    & 1    & 0    & 0    & 1    & 0    & 0    & 1 &  &       & -73  &        &       & 73   \\  \cline{2-2} \cline{5-10} \cline{12-13} \cline{15-16}
                   & C_{\MM{доп.}} & 1    & 0    & 0    & 0    & 0    & 0    & 0    & 0 &  &       &      &        &       & 128? \\  \cline{2-2} \cline{5-10} \cline{12-13} \cline{15-16}
                   & C_{\MM{пр.}}  & 1    & 0    & 0    & 0    & 0    & 0    & 0    & 0 &  &       & -128 &        &       &      \\
          \end{array}
        $$
        $$ CF=1 ,\ ZF=0,\ PF=0,\ AF=1,\ SF=1,\	OF=0 $$
\end{enumerate}
\end{document}
