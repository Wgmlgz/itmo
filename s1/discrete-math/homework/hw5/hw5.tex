\documentclass{article}
\usepackage{import}
\import{../../../lib/latex/}{wgmlgz}

\begin{document}

\itmo[
       variant=111,
       labn=5,
       worktype=Домашняя работа,
       discipline=Дискретная математика,
       group=P3115,
       student=Владимир Мацюк,
       teacher=Поляков Владимир Иванович,
       logo=../../../lib/img/itmo.png
]

\newcommand{\car}{\multicolumn{1}{c@{\hspace*{\tabcolsep}\makebox[0pt]{\curvearrowleft}}}{}}
\newcommand{\rcar}{\multicolumn{1}{c@{\hspace*{\tabcolsep}\makebox[0pt]{\curvearrowright}}}{}}
\newcommand{\ncar}{\multicolumn{1}{c@{\hspace*{\tabcolsep}\makebox[0pt]{}}}{}}
\newcommand{\SPACE}{\multicolumn{12}{c}{}}
\newcommand{\INT}{\multicolumn{5}{c}{\MM{Интерпретации}}}
\newcommand{\PLUS}{\multirow{2}{*}{+}}
\newcommand{\MINUS}{\multirow{2}{*}{-}}
\newcommand{\SIGN}{\multicolumn{2}{c}{\MM{Знаковая}}}
\newcommand{\USIGN}{\multicolumn{2}{c}{\MM{Беззнаковая}}}

\section{Числа}
$$
       \begin{array}{|c|c|}
              \hline
              A & 1150 \nl
              B & 23 \nl
       \end{array}
$$
\section{Задание}

\begin{center}

       \begin{enumerate}
              \begin{enumerate}
                     \item Выполнить операцию деления заданных целых чисел A и B со всеми комбинациями знаков, используя метод деления в дополнительных кодах. Для представления делимого (A) использовать 16 двоичных разрядов (один –знаковый и 15 – цифровых), для представления делителя (B) – 8 разрядов (один – знаковый и 7 – цифровых). Остаток от деления и частное представляются в той же разрядной сетке, что и делитель.
                     \item Результаты операции представить в десятичной системе счисления и проверить их правильность.
              \end{enumerate}
                    
$$\begin{array}{c}(A>0,\ B>0)                                     \\
    \left.[A]_{пр.}=0111000;\ [B]_{пр.}=1011011\right. \\
    \\ \begin{array}{|c|c|c|c|p{5cm}|} \hline N шага & Операнды     & \begin{array}{c}\textup{Делимое/остаток} \\ \textup{старш.\ разряды}\end{array} & \begin{array}{c}\textup{Делимое/остаток} \\ \textup{младш.\ разряды}\end{array} & \begin{array}{c}\textup{Знак} \\ \textup{остатка}\end{array} & \begin{array}{c}\textup{Знак} \\ \textup{частного}\end{array} \\ \hline
             0                                & \textup{СЧП} & 0\ 0\ 0\ 0\ 0\ 0\ 0                                                             & 1\ 0\ 0\ 0\ 0\ 1\ \underline{0}                                                 & \textup{Обнуление\ старших\ разрядов\ СЧП}                                                                                                             \\ \hline
    \end{array} \\
    \\
    \\  \left[C\right]_{\textup{пр.}}=01001111101000_2 = 5096_{10}\end{array}$$
$$\begin{array}{c}(A<0,\ B>0)                                       \\
    \left.[-A]_{доп.}=1001000;\ [B]_{пр.}=1011011\right. \\
    \\ \begin{array}{|c|c|c|c|p{5cm}|} \hline N шага & Операнды     & \begin{array}{c}\textup{Делимое/остаток} \\ \textup{старш.\ разряды}\end{array} & \begin{array}{c}\textup{Делимое/остаток} \\ \textup{младш.\ разряды}\end{array} & \begin{array}{c}\textup{Знак} \\ \textup{остатка}\end{array} & \begin{array}{c}\textup{Знак} \\ \textup{частного}\end{array} \\ \hline
             0                                & \textup{СЧП} & 0\ 0\ 0\ 0\ 0\ 0\ 0                                                             & 1\ 0\ 0\ 0\ 0\ 1\ \underline{0}                                                 & \textup{Обнуление\ старших\ разрядов\ СЧП}                                                                                                             \\ \hline
    \end{array} \\
    \\
    \\  \left[C\right]_{\textup{доп}.}=10110000011000_2 = -5096_{10}\end{array}$$
$$\begin{array}{c}(A>0,\ B<0)                                       \\
    \left.[A]_{пр.}=0111000;\ [-B]_{доп.}=0100101\right. \\
    \\ \begin{array}{|c|c|c|c|p{5cm}|} \hline N шага & Операнды     & \begin{array}{c}\textup{Делимое/остаток} \\ \textup{старш.\ разряды}\end{array} & \begin{array}{c}\textup{Делимое/остаток} \\ \textup{младш.\ разряды}\end{array} & \begin{array}{c}\textup{Знак} \\ \textup{остатка}\end{array} & \begin{array}{c}\textup{Знак} \\ \textup{частного}\end{array} \\ \hline
             0                                & \textup{СЧП} & 0\ 0\ 0\ 0\ 0\ 0\ 0                                                             & 1\ 0\ 0\ 0\ 0\ 1\ \underline{0}                                                 & \textup{Обнуление\ старших\ разрядов\ СЧП}                                                                                                             \\ \hline
    \end{array} \\
    \\
    \\  \left[C\right]_{\textup{пр.}}=10110000011000_2 = -5096_{10}\end{array}$$
$$\begin{array}{c}(A<0,\ B<0)                                         \\
    \left.[-A]_{доп.}=1001000;\ [-B]_{доп.}=0100101\right. \\
    \\ \begin{array}{|c|c|c|c|p{5cm}|} \hline N шага & Операнды     & \begin{array}{c}\textup{Делимое/остаток} \\ \textup{старш.\ разряды}\end{array} & \begin{array}{c}\textup{Делимое/остаток} \\ \textup{младш.\ разряды}\end{array} & \begin{array}{c}\textup{Знак} \\ \textup{остатка}\end{array} & \begin{array}{c}\textup{Знак} \\ \textup{частного}\end{array} \\ \hline
             0                                & \textup{СЧП} & 0\ 0\ 0\ 0\ 0\ 0\ 0                                                             & 1\ 0\ 0\ 0\ 0\ 1\ \underline{0}                                                 & \textup{Обнуление\ старших\ разрядов\ СЧП}                                                                                                             \\ \hline
    \end{array} \\
    \\
    \\  \left[C\right]_{\textup{доп}.}=01001111101000_2 = 5096_{10}\end{array}$$

       \end{enumerate}
\end{center}
\end{document}
