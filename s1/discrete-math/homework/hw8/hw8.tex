\documentclass{article}
\usepackage{import}
\import{../../../lib/latex/}{wgmlgz}

\begin{document}

\itmo[
       variant=111,
       labn=8,
       worktype=Домашняя работа,
       discipline=Дискретная математика,
       group=P3115,
       student=Владимир Мацюк,
       teacher=Поляков Владимир Иванович,
       logo=../../../lib/img/itmo.png
]

\newcommand{\car}{\multicolumn{1}{c@{\hspace*{\tabcolsep}\makebox[0pt]{\curvearrowleft}}}{}}
\newcommand{\rcar}{\multicolumn{1}{c@{\hspace*{\tabcolsep}\makebox[0pt]{\curvearrowright}}}{}}
\newcommand{\ncar}{\multicolumn{1}{c@{\hspace*{\tabcolsep}\makebox[0pt]{}}}{}}
\newcommand{\SPACE}{\multicolumn{12}{c}{}}
\newcommand{\INT}{\multicolumn{5}{c}{\MM{Интерпретации}}}
\newcommand{\PLUS}{\multirow{2}{*}{+}}
\newcommand{\MINUS}{\multirow{2}{*}{-}}
\newcommand{\SIGN}{\multicolumn{2}{c}{\MM{Знаковая}}}
\newcommand{\USIGN}{\multicolumn{2}{c}{\MM{Беззнаковая}}}

\section{Числа}
$$
       \begin{array}{|c|c|}
              \hline
              A & 4,7 \nl
              B & 0,053 \nl
       \end{array}
$$
\section{Задание}

\begin{enumerate}
       \item Заданные числа А (делимое) и В (делитель) представить в форматах Ф1 и Ф2 с укороченной мантиссой (8 двоичных разрядов). Метод округления выбирается произвольно.
       \\ Примечание: общее число разрядов в формате – 16.
       \item Выполнить операцию деления операндов в формате Ф1.
       \item В случае положительного результата «пробного» вычитания сохранить младшую тетраду.
       \item Выполнить операцию деления операндов в формате Ф2.
       \item Результаты представить в форматах операндов, перевести в десятичную систему счисления и проверить их правильность.
       \item Определить абсолютную и относительную погрешности результатов и обосновать их причину.
       Варианты задания приведены в табл. 8 Приложения 1.
       
             
\end{enumerate}
\section{Решение}
\begin{enumerate}
       \item Деление в формате Ф1:

$$ A = 4.7 = (4.B30)_{16} = (0.4B)_{16} * 16^1 $$
$$ B = 0.053 = (0.0D910)_{16} = (0.D9)_{16} * 16^{-1} $$
$$ Xc = Xa - Xb + d $$
$$ Xc = 65 - 63 + 64 = 66 $$
$$ Pc = 2 $$
$$\begin{array}{c} \\ 
 \\ \begin{array}{|c|c|c|c|} \hline \textup{N\ шага} & \textup{Действие} & \textup{Делимое} & \textup{Частное} \\ \hline 
\begin{array}{c}$$0$$ \\ $$$$ \\ $$$$\end{array} & \begin{array}{c}$$Ma$$ \\ $$[-M_b]доп$$ \\ $$R0$$\end{array} & \begin{array}{c}$$001001011$$ \\ $$100100111$$ \\ $$101110010$$\end{array} & \begin{array}{c}$$00000000$$ \\ $$$$ \\ $$00000000$$\end{array} \\ \hline 
\begin{array}{c}$$1$$ \\ $$$$ \\ $$$$\end{array} & \begin{array}{c}$$\leftarrow R0$$ \\ $$[M_b]пр$$ \\ $$R1$$\end{array} & \begin{array}{c}$$011100100$$ \\ $$011011001$$ \\ $$110111101$$\end{array} & \begin{array}{c}$$00000000$$ \\ $$$$ \\ $$00000000$$\end{array} \\ \hline 
\begin{array}{c}$$2$$ \\ $$$$ \\ $$$$\end{array} & \begin{array}{c}$$\leftarrow R1$$ \\ $$[M_b]пр$$ \\ $$R2$$\end{array} & \begin{array}{c}$$101111010$$ \\ $$011011001$$ \\ $$001010011$$\end{array} & \begin{array}{c}$$00000000$$ \\ $$$$ \\ $$00000001$$\end{array} \\ \hline 
\begin{array}{c}$$3$$ \\ $$$$ \\ $$$$\end{array} & \begin{array}{c}$$\leftarrow R2$$ \\ $$[-M_b]доп$$ \\ $$R3$$\end{array} & \begin{array}{c}$$010100110$$ \\ $$100100111$$ \\ $$111001101$$\end{array} & \begin{array}{c}$$00000010$$ \\ $$$$ \\ $$00000010$$\end{array} \\ \hline 
\begin{array}{c}$$4$$ \\ $$$$ \\ $$$$\end{array} & \begin{array}{c}$$\leftarrow R3$$ \\ $$[M_b]пр$$ \\ $$R4$$\end{array} & \begin{array}{c}$$110011010$$ \\ $$011011001$$ \\ $$001110011$$\end{array} & \begin{array}{c}$$00000100$$ \\ $$$$ \\ $$00000101$$\end{array} \\ \hline 
\begin{array}{c}$$5$$ \\ $$$$ \\ $$$$\end{array} & \begin{array}{c}$$\leftarrow R4$$ \\ $$[-M_b]доп$$ \\ $$R5$$\end{array} & \begin{array}{c}$$011100110$$ \\ $$100100111$$ \\ $$000001101$$\end{array} & \begin{array}{c}$$00001010$$ \\ $$$$ \\ $$00001011$$\end{array} \\ \hline 
\begin{array}{c}$$6$$ \\ $$$$ \\ $$$$\end{array} & \begin{array}{c}$$\leftarrow R5$$ \\ $$[-M_b]доп$$ \\ $$R6$$\end{array} & \begin{array}{c}$$000011010$$ \\ $$100100111$$ \\ $$101000001$$\end{array} & \begin{array}{c}$$00010110$$ \\ $$$$ \\ $$00010110$$\end{array} \\ \hline 
\begin{array}{c}$$7$$ \\ $$$$ \\ $$$$\end{array} & \begin{array}{c}$$\leftarrow R6$$ \\ $$[M_b]пр$$ \\ $$R7$$\end{array} & \begin{array}{c}$$010000010$$ \\ $$011011001$$ \\ $$101011011$$\end{array} & \begin{array}{c}$$00101100$$ \\ $$$$ \\ $$00101100$$\end{array} \\ \hline 
\begin{array}{c}$$8$$ \\ $$$$ \\ $$$$\end{array} & \begin{array}{c}$$\leftarrow R7$$ \\ $$[M_b]пр$$ \\ $$R8$$\end{array} & \begin{array}{c}$$010110110$$ \\ $$011011001$$ \\ $$110001111$$\end{array} & \begin{array}{c}$$01011000$$ \\ $$$$ \\ $$01011000$$\end{array} \\ \hline 
 \end{array} \\
 \\ 
 \\ \end{array}$$

$$ C^* = (0.58)_{16} * 16^2 = (58)_{16} = 88 $$
$$ \Delta C = C^T - C^* = 88.679 - 88 = 0.679 $$
$$ δC = |\Delta C/C^T| * 100\% = |0.679/88.679| * 100\% = 0.77\% $$
Погрешность вызвана неточным представлением операндов

\item Деление в формате Ф2:

$$ A = 4.7 = (100.101100110)_2 = (0.10010110)_2 * 2^3 $$
$$ B = 0.053 = (0.0000110110010001)_2 = (0.11011001)_2 * 2^{-4} $$
$$ Xc = Xa - Xb + d $$
$$ Xc = 131 - 124 + 128 = 135 $$
$$ Pc = 7 $$
$$\begin{array}{c} \\ 
 \\ \begin{array}{|c|c|c|c|} \hline \textup{N\ шага} & \textup{Действие} & \textup{Делимое} & \textup{Частное} \\ \hline 
\begin{array}{c}$$0$$ \\ $$$$ \\ $$$$\end{array} & \begin{array}{c}$$Ma$$ \\ $$[-M_b]доп$$ \\ $$R0$$\end{array} & \begin{array}{c}$$010010110$$ \\ $$100100111$$ \\ $$110111101$$\end{array} & \begin{array}{c}$$00000000$$ \\ $$$$ \\ $$00000000$$\end{array} \\ \hline 
\begin{array}{c}$$1$$ \\ $$$$ \\ $$$$\end{array} & \begin{array}{c}$$\leftarrow R0$$ \\ $$[M_b]пр$$ \\ $$R1$$\end{array} & \begin{array}{c}$$101111010$$ \\ $$011011001$$ \\ $$001010011$$\end{array} & \begin{array}{c}$$00000000$$ \\ $$$$ \\ $$00000001$$\end{array} \\ \hline 
\begin{array}{c}$$2$$ \\ $$$$ \\ $$$$\end{array} & \begin{array}{c}$$\leftarrow R1$$ \\ $$[-M_b]доп$$ \\ $$R2$$\end{array} & \begin{array}{c}$$010100110$$ \\ $$100100111$$ \\ $$111001101$$\end{array} & \begin{array}{c}$$00000010$$ \\ $$$$ \\ $$00000010$$\end{array} \\ \hline 
\begin{array}{c}$$3$$ \\ $$$$ \\ $$$$\end{array} & \begin{array}{c}$$\leftarrow R2$$ \\ $$[M_b]пр$$ \\ $$R3$$\end{array} & \begin{array}{c}$$110011010$$ \\ $$011011001$$ \\ $$001110011$$\end{array} & \begin{array}{c}$$00000100$$ \\ $$$$ \\ $$00000101$$\end{array} \\ \hline 
\begin{array}{c}$$4$$ \\ $$$$ \\ $$$$\end{array} & \begin{array}{c}$$\leftarrow R3$$ \\ $$[-M_b]доп$$ \\ $$R4$$\end{array} & \begin{array}{c}$$011100110$$ \\ $$100100111$$ \\ $$000001101$$\end{array} & \begin{array}{c}$$00001010$$ \\ $$$$ \\ $$00001011$$\end{array} \\ \hline 
\begin{array}{c}$$5$$ \\ $$$$ \\ $$$$\end{array} & \begin{array}{c}$$\leftarrow R4$$ \\ $$[-M_b]доп$$ \\ $$R5$$\end{array} & \begin{array}{c}$$000011010$$ \\ $$100100111$$ \\ $$101000001$$\end{array} & \begin{array}{c}$$00010110$$ \\ $$$$ \\ $$00010110$$\end{array} \\ \hline 
\begin{array}{c}$$6$$ \\ $$$$ \\ $$$$\end{array} & \begin{array}{c}$$\leftarrow R5$$ \\ $$[M_b]пр$$ \\ $$R6$$\end{array} & \begin{array}{c}$$010000010$$ \\ $$011011001$$ \\ $$101011011$$\end{array} & \begin{array}{c}$$00101100$$ \\ $$$$ \\ $$00101100$$\end{array} \\ \hline 
\begin{array}{c}$$7$$ \\ $$$$ \\ $$$$\end{array} & \begin{array}{c}$$\leftarrow R6$$ \\ $$[M_b]пр$$ \\ $$R7$$\end{array} & \begin{array}{c}$$010110110$$ \\ $$011011001$$ \\ $$110001111$$\end{array} & \begin{array}{c}$$01011000$$ \\ $$$$ \\ $$01011000$$\end{array} \\ \hline 
\begin{array}{c}$$8$$ \\ $$$$ \\ $$$$\end{array} & \begin{array}{c}$$\leftarrow R7$$ \\ $$[M_b]пр$$ \\ $$R8$$\end{array} & \begin{array}{c}$$100011110$$ \\ $$011011001$$ \\ $$111110111$$\end{array} & \begin{array}{c}$$10110000$$ \\ $$$$ \\ $$10110000$$\end{array} \\ \hline 
 \end{array} \\
 \\ 
 \\ \end{array}$$

$$ C^* = (0.10110000)_2 * 2^7 = (1011000.0)_2 = 88 $$
$$ \Delta C = C^T - C^* = 88.679 - 88 = 0.679 $$
$$ δC = |\Delta C/C^T| * 100\% = |0.679/88.679| * 100\% = 0.77\% $$
Погрешность вызвана неточным представлением операндов

\end{enumerate}
\end{document}
