\documentclass{article}
\usepackage[utf8]{inputenc}
\usepackage[russian]{babel}
\usepackage{fontspec}
\usepackage{tikz}
\usepackage{xfrac}
\usepackage{amsmath}
\usepackage{wrapfig}

\setmainfont[
  Ligatures=TeX,
  Extension=.otf,
  BoldFont=cmunbx,
  ItalicFont=cmunti,
  BoldItalicFont=cmunbi,
]{cmunrm}
\setsansfont[
  Ligatures=TeX,
  Extension=.otf,
  BoldFont=cmunsx,
  ItalicFont=cmunsi,
]{cmunss}

\usetikzlibrary{positioning,calc}
\begin{document}
\begin{figure}
  \begin{minipage}[c]{6cm}
    \begin{tikzpicture}
      \def\dx{0.7}
      \def\x{4}
      \draw[gray, very thick] (0,0) rectangle (\x,\x);
      \draw[gray, very thick] (0,0) rectangle (\x + \dx,\x + \dx);
      \draw[gray, very thick] (0,0) -- (\x,\x);
      \draw[red!60, very thick] (\x,\x) -- node[anchor=north]{$dl$} (\x+\dx,\x+\dx) ;
      \draw[blue!60, very thick] (0,\x)  -- node[anchor=south]{$x$} (\x,\x) ;
      \draw[orange, very thick] (0,\x+\dx)  -- node[anchor=south]{$x + dx$} (\x+\dx,\x+\dx) ;
    \end{tikzpicture}
  \end{minipage}%
  \begin{minipage}[c]{7cm}
    $$ S = a^2, \hspace*{2mm} dx = \frac{dl}{\sqrt{2}} $$
    $$ S_a = x^2, \hspace*{2mm} S_b = (x +  dx)^2 $$
    \linebreak
    $$ dS = S_b - S_a $$
    $$ dS = (x + dx)^2 - x^2 = x^2 + 2xdx + dx^2 - x^2 = $$
    $$ = 2xdx + dx^2 = 2x\frac{dl}{\sqrt{2}} + \left(\frac{dl}{\sqrt{2}}\right)^2 = $$
    $$ = \sqrt{2}xdl + \frac{dl^2}{2} $$
    \linebreak
    $$ \frac{dS}{dl} = \frac{\sqrt{2}xdl + \frac{dl^2}{2}}{dl} = \sqrt{2}x + \frac{dl}{2} \sim \sqrt{2}x $$
  \end{minipage}
\end{figure}
\begin{figure}
  \begin{minipage}[c]{4cm}
    Ответ: $ \sqrt{2}xdl + \frac{dl^2}{2} $
  \end{minipage}
  \begin{minipage}[c]{6cm}
    \begin{tikzpicture}
      \def\dx{1.5}
      \def\x{4}
      \draw[blue!60, fill=blue!10, very thick]
      (0,0)
      rectangle node[anchor=center]{$\displaystyle x^2$}
      (\x,\x);
      
      \draw[orange!60, fill=orange!10, very thick]
      (0,\x)
      rectangle node[anchor=center]{$\displaystyle\frac{\sqrt{2}xdl}{2}$}
      (\x,\x+\dx);
      
      \draw[orange!60, fill=orange!10, very thick]
      (\x,0)
      rectangle node[anchor=center]{$\displaystyle\frac{\sqrt{2}xdl}{2}$}
      (\x+\dx,\x);
      
      \draw[red!60, fill=orange!10, very thick]
      (\x,\x)
      rectangle node[anchor=center]{$\displaystyle\frac{dl^2}{2}$}
      (\x+\dx,\x+\dx);
      
      \draw[blue!60, very thick] (0,\x) -- (\x,\x) ;
    \end{tikzpicture}
  \end{minipage}
\end{figure}

\begin{figure}
  В круг радиуса r вписан квадрат, в квадрат вписан круг и так n раз.
  Найдите предел суммы площадей всех кругов и предел суммы
  площадей всех квадратов при $n \rightarrow \inf$





  \begin{tikzpicture}
    \def\r{3}
    \def\cs{0.70710678118 }
    
    \draw[blue!60, very thick] (0,0) circle (\r);
    \draw[blue!60, very thick] (0,0) -- node[anchor=north]{$r_0$} (\cs*\r,\cs*\r) ;
    \draw[blue!60, very thick] (0,0) -- node[anchor=east]{$l_0$} (0,\cs*\r) ;
    \draw[blue!60, very thick] (-\cs*\r,-\cs*\r) rectangle (\cs*\r,\cs*\r);
    
    \def\r{3*\cs}
    
    \draw[red!60, very thick] (0,0) circle (\r);
    \draw[red!60, very thick] (0,0) -- node[anchor=south]{$r_1$} (-\cs*\r,-\cs*\r) ;
    \draw[red!60, very thick] (0,0) -- node[anchor=north]{$l_1$} (\cs*\r,0) ;
    \draw[red!60, very thick] (-\cs*\r,-\cs*\r) rectangle (\cs*\r,\cs*\r);
  \end{tikzpicture}
  
  $$ S(r_i) = \pi {r_i}^2,\ \frac{S(l_i)}{4} = {l_i}^2,\ c = \cos(\sfrac{\pi}{4}) = \frac{\sqrt{2}}{2}$$
  $$ r_0 = r,\ l_0 = r_0c,\ r_1 = l_0c,\ \ldots\ r_i = rc^{2i},\ l_i = rc^{2i + 1}$$
  \begin{equation}
    \begin{split}
      S_r & = \sum_{i=0}^{\infty} S(r_i) = \sum_{i=0}^{\infty} \pi {r_i}^2= \pi\sum_{i=0}^{\infty} {\left(rc^{2i}\right)}^2= \\
      & =\pi r^2\sum_{i=0}^{\infty} c^{4i} = \frac{\pi r^2}{1 - c^4} = \frac{\pi r^2}{1 - \sfrac{1}{4}} = \frac{\pi r^2}{\sfrac{3}{4}} = \frac{4\pi r^2}{3} \\
      \frac{S_l}{4} & = \sum_{i=0}^{\infty} S(l_i) = \sum_{i=0}^{\infty} {l_i}^2=\sum_{i=0}^{\infty} {\left(rc^{2i+1}\right)}^2= \\
      & =(cr)^2\sum_{i=0}^{\infty} c^{4i} = \frac{(cr)^2}{1 - c^4} = \frac{(cr)^2}{1 - \sfrac{1}{4}} = \frac{(cr)^2}{\sfrac{3}{4}} = \frac{4(cr)^2}{3}\\
      S_l &= 4\frac{4(cr)^2}{3} = \frac{(cr)^2}{3} = \frac{\frac{1}{2}r^2}{3} = \frac{r^2}{6}
    \end{split}
  \end{equation}
\end{figure}


\end{document}