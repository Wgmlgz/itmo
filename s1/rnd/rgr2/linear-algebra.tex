\documentclass{article}
\usepackage[utf8]{inputenc}
\usepackage[russian]{babel}
\usepackage{fontspec}
\usepackage{tikz}
\usepackage{xfrac}
\usepackage{amsmath}
\usepackage{wrapfig}
\usepackage{pgfplots}

\setmainfont[
  Ligatures=TeX,
  Extension=.otf,
  BoldFont=cmunbx,
  ItalicFont=cmunti,
  BoldItalicFont=cmunbi,
]{cmunrm}
\setsansfont[
  Ligatures=TeX,
  Extension=.otf,
  BoldFont=cmunsx,
  ItalicFont=cmunsi,
]{cmunss}

\usetikzlibrary{positioning,calc}
\begin{document}
Определите траекторию и её уравнение для точки, которая в своем движении остается вдвое
ближе к точке $A(1,0)$, чем к точке $B(4,0)$.

$$ 2(\sqrt{(x - 1) ^ 2 + y^2}) = \sqrt{(x - 4) ^ 2 + y^2} $$
$$ 4((x - 1) ^ 2 + y^2) = (x - 4) ^ 2 + y^2 $$
$$ 4(x^2 - 2x + 1) + 4y^2 = (x^2 - 8x + 16) + y^2 $$
$$ 4x^2 - 8x + 4 + 4y^2 = x^2 - 8x + 16 + y^2 $$
$$ 4x^2 - 8x + 4 + 4y^2 -x^2 + 8x - 16 - y^2 = 0 $$
$$ 3x^2  + 3y^2  - 12  = 0 $$
$$ x^2 + y^2 - 4= 0 $$
$$ x^2 + y^2 = 2^2 $$

\begin{center}
  Получается окружность с радиусом $2$ и центром в $(0, 0)$.

  \begin{tikzpicture}
    \draw[help lines, color=gray!30, dashed] (-4.9,-4.9) grid (4.9,4.9);
    \draw[->,ultra thick] (-5,0)--(5,0) node[right]{$x$};
    \draw[->,ultra thick] (0,-5)--(0,5) node[above]{$y$};
    \filldraw[red!60] (1,0) circle (2pt) node[anchor=north]{A};
    \filldraw[blue!60] (4,0) circle (2pt) node[anchor=north]{B};

    \draw[green!60, very thick] (0,0) circle (2);

    \filldraw[black] (3,2) circle (0) node[anchor=north]{$\displaystyle x^2 + y^2 = 2^2 $};

    \draw[red!60, ultra thick] (0,2)--(1,0) node[right]{};
    \draw[red!60, ultra thick] (0,2)--(2,1) node[right]{};
    \draw[blue!60, ultra thick] (4,0)--(2,1) node[right]{};
  \end{tikzpicture}
\end{center}

\end{document}