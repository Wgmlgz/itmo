\documentclass{article}
\usepackage{import}
\import{../../lib/latex/}{wgmlgz}


\begin{document}

\itmo[
  variant=8912,
  labn=6,
  discipline=Основы профессиональной деятельности,
  group=P3115,
  student=Владимир Мацюк,
  teacher=Абузов Ярослав Александрович,
  logo=../../lib/img/itmo.png
]


\section{Текст задания}
По выданному преподавателем варианту разработать и исследовать работу комплекса программ обмена данными в режиме прерывания программы. Основная программа должна изменять содержимое заданной ячейки памяти (Х), которое должно быть представлено как знаковое число. Область допустимых значений изменения Х должна быть ограничена заданной функцией F(X) и конструктивными особенностями регистра данных ВУ (8-ми битное знаковое представление). Программа обработки прерывания должна выводить на ВУ модифицированное значение Х в соответствии с вариантом задания, а также игнорировать все необрабатываемые прерывания.


\begin{enumerate}
  \item Основная программа должна декрементировать содержимое X (ячейки памяти с адресом $00A_{16}$) в цикле.
  \item Обработчик прерывания должен по нажатию кнопки готовности ВУ-3 осуществлять вывод результата вычисления функции F(X)=6X-8 на данное ВУ, a по нажатию кнопки готовности ВУ-2 записать содержимое РД данного ВУ в Х
  \item Если Х оказывается вне ОДЗ при выполнении любой операции по его изменению, то необходимо в Х записать максимальное по ОДЗ число.
\end{enumerate}

\section{Программа}

\lstinputlisting{code.bcomp}

\section{Область допустимых значений}

$$-128 \le 6x-8 \le 127$$
$$-120 \le 6x \le 135 $$
$$-20 \le\ x \le 22.5 $$
$$x \in [-20; 22]$$
$$-20=FFEC_{16},\ 22=0016_{16}$$


\section{Расположение данных в памяти}

\begin{enumerate}
  \item Вектор прерываний: 0x000 – 0x00F
  \item Переменные: 0x0A – 0x0C
  \item Программа: 0x0D – 0x03B
\end{enumerate}


\section{Область представления}

X, min, max - i16

\section{Вывод}

В ходе выполнения лабораторной работы я изучил обмен данными в режиме прерываний в БЭВМ.

\newpage

\section{Методика проверки программы}

Проверка обработки прерываний:
\begin{enumerate}
  \item  Загрузить текст программы в БЭВМ.
  \item  Заменить NOP по нужному адресу на HLT.
  \item  Запустить программу в режиме РАБОТА.
  \item  Установить «Готовность ВУ-3».
  \item  Дождаться останова.
  \item  Записать текущее значение X из памяти БЭВМ:
  \item  Запомнить текущее состояние счетчика команд.
  \item  Ввести в клавишный регистр значение 0xA
  \item  Нажать «Ввод адреса».
  \item  Нажать «Чтение».
  \item  Записать значение регистра данных.
  \item  Вернуть счетчик команд в исходное состояние.
  \item  Записать результат обработки прерывания – содержимое DR контроллера ВУ-3
  \item  Рассчитать ожидаемое значение обработки прерывания
  \item  Нажать «Продолжение».
  \item  Ввести в ВУ-2 произвольное число, записать его
  \item  Установить «Готовность ВУ-2».
  \item  Дождаться останова.
  \item  Записать текущее значение X из памяти БЭВМ, также, как и в пункте 6.
  \item  Нажать «Продолжение».
  \item  Записать текущее значение X из памяти БЭВМ, также, как и в пункте 6.
  \item  Рассчитать ожидаемое значение переменной X после обработки прерывания
\end{enumerate}

Проверка основной программы:
\begin{enumerate}

  \item  Загрузить текст программы в БЭВМ.
  \item  Записать в переменную X минимальное по ОДЗ значение (22)
  \item  Запустить программу в режиме останова.
  \item  Пройти нужное количество шагов программы, убедиться, что при уменьшении X на 1, до после момента, когда он равен -20, происходит сброс значения в максимальное по ОДЗ.
\end{enumerate}

% \begin{center}


%   Прерывание ВУ-3

%   \begin{tabular}[]{|c|c|c|} \hline
%     AC (0...7)        & Ожидание $6X-8$  & DR \nl
%     $10_{16} (16)$ & $52_{16} (-20) $ & $52_{16} (-20) $ \nl
%     $FF_{-1} (-20)$ & $16_{16} (22)  $ & $16_{16} (22) $ \nl
%     $FFEB_{16} (-21)$ & $16_{16} (22)  $ & $16_{16} (22) $ \nl
%   \end{tabular}
% \end{center}


% \begin{center}

%   Прерывание ВУ-2

%   \begin{tabular}[]{|c|c|c|c|} \hline
%     AC (0...7)        & DR КВУ-2         & AC (DR - X)      & Результат AC (0...7) \nl
%     $1_{16} (1)$ & $7f_{16} (127) $ & $52_{16} (-20) $ & $52_{16} (-20) $ \nl
%     $1_{16} (1)$ & $1_{16} (1)  $ & $16_{16} (22)  $ & $16_{16} (22) $ \nl
%     $1_{16} (1)$ & $e1_{16} (-31)  $ & $16_{16} (22)  $ & $16_{16} (22) $ \nl
%   \end{tabular}
% \end{center}

\begin{center}

  Основная программа

  \begin{tabular}[]{|c|c|c|} \hline
    AC                & Ожидание         & AC \nl
    $FFED_{16} (-19)$ & $52_{16} (-20) $ & $52_{16} (-20) $ \nl
    $FFEC_{16} (-20)$ & $16_{16} (22)  $ & $16_{16} (22) $ \nl
    $FFEB_{16} (-21)$ & $16_{16} (22)  $ & $16_{16} (22) $ \nl
  \end{tabular}
\end{center}




\end{document}
