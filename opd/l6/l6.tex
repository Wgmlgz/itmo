\documentclass{article}
\usepackage{import}
\import{../../lib/latex/}{wgmlgz}


\begin{document}

\itmo[
  variant=8912,
  labn=6,
  discipline=Основы профессиональной деятельности,
  group=P3115,
  student=Владимир Мацюк,
  teacher=Абузов Ярослав Александрович,
  logo=../../lib/img/itmo.png
]


\section{Текст задания}
По выданному преподавателем варианту разработать и исследовать работу комплекса программ обмена данными в режиме прерывания программы. Основная программа должна изменять содержимое заданной ячейки памяти (Х), которое должно быть представлено как знаковое число. Область допустимых значений изменения Х должна быть ограничена заданной функцией F(X) и конструктивными особенностями регистра данных ВУ (8-ми битное знаковое представление). Программа обработки прерывания должна выводить на ВУ модифицированное значение Х в соответствии с вариантом задания, а также игнорировать все необрабатываемые прерывания.


\begin{enumerate}
  \item Основная программа должна декрементировать содержимое X (ячейки памяти с адресом 00A16) в цикле.
  \item Обработчик прерывания должен по нажатию кнопки готовности ВУ-3 осуществлять вывод результата вычисления функции F(X)=6X-8 на данное ВУ, a по нажатию кнопки готовности ВУ-2 записать содержимое РД данного ВУ в Х
  \item Если Х оказывается вне ОДЗ при выполнении любой операции по его изменению, то необходимо в Х записать максимальное по ОДЗ число.
\end{enumerate}

\section{Программа}

\lstinputlisting{code.bcomp}

\section{Область допустимых значений}

$$-128 \le 6x-8 \le 127$$
$$-120 \le 6x \le 135 $$
$$-20 \le\ x \le 22.5 $$
$$x \in [-20; 22]$$
$$-20=FFEC_{16},\ 22=0016_{16}$$


\section{Расположение данных в памяти}

\begin{enumerate}
  \item Вектор прерываний: 0x000 – 0x00F
  \item Переменные: 0x0A – 0x0C
  \item Программа: 0x0D – 0x03B
\end{enumerate}


\section{Область представления}

X, min, max - i16

\section{Вывод}

В ходе выполнения лабораторной работы я изучил обмен данными в режиме прерываний в БЭВМ.

\end{document}
