\documentclass{article}
\usepackage{import}
\import{../../lib/latex/}{wgmlgz}


\begin{document}

\itmo[
  variant=1111,
  labn=5,
  discipline=Основы профессиональной деятельности,
  group=P3115,
  student=Владимир Мацюк,
  teacher=Абузов Ярослав Александрович,
  logo=../../lib/img/itmo.png
]


\section{Текст задания}
По выданному преподавателем варианту разработать программу асинхронного обмена данными с внешним устройством. При помощи программы осуществить ввод или вывод информации, используя в качестве подтверждения данных сигнал (кнопку) готовности ВУ.
\begin{enumerate}
  \item Программа осуществляет асинхронный ввод данных с ВУ-2
  \item Программа начинается с адреса $495_{16}$. Размещаемая строка находится по адресу $616_{16}$.
  \item Строка должна быть представлена в кодировке КОИ-8.
  \item Формат представления строки в памяти: АДР1: СИМВ2 СИМВ1 АДР2: СИМВ4 СИМВ3 ... СТОП\_СИМВ.
  \item Ввод или вывод строки должен быть завершен по символу c кодом 00 (NUL). Стоп символ является обычным символом строки и подчиняется тем же правилам расположения в памяти что и другие символы строки.
\end{enumerate}

\section{Описание программы}

Передаваемое сообщение: Да
В кодировке КОИ-8: c4 c1

Текст программы на ассемблере:

\begin{lstlisting}
ORG 0x495

res:	WORD 0x616
terminator: WORD 0x00
tmp:	WORD ?

START:
  CLA
s1:
  IN 5
  AND #0x40
  BEQ s1
  IN 4
  ST (res)
  ST tmp
  CMP terminator
  BEQ exit
  CLA
s2:
  IN 5
  AND #0x40
  BEQ s2
  IN 4
  SWAB
  OR tmp
  ST (res)
  SUB tmp
  SWAB
  CMP terminator
  BEQ exit
  LD (res)+
  CLA
  JUMP s1
exit:
  LD (res)+
  HLT
\end{lstlisting}

Текст исходной программы:

\begin{tabular}{|c|r|l|l|} \hline
  Адрес & Код команды & Мнемоника   & Комментарии \nl
  495   & 0562        & res         & \nl
  496   & 0000        & terminator  & \nl
  497   & 0000        & tmp         & Нет   \nl
  498   & +0200       & CLA         & Очистка аккумулятора  \nl
  499   & 1207        & IN 5        & Чтение регистра состояния ВУ-2  \nl
  49a   & 2F40        & AND \#0x40  & Логическое умножение (Прямая загрузка операнда)  \nl
  49b   & F0FD        & BEQ IP-3    & Переход, если равенство  \nl
  49c   & 1206        & IN 4        & Чтение регистра данных ВУ-2  \nl
  49d   & E8F7        & ST (IP-9)   & Сохранение (Косвенная относительная адресация)  \nl
  49e   & EEF8        & ST IP-8     & Сохранение (Прямая относительная адресация)  \nl
  49f   & 7EF6        & CMP IP-A    & Сравнение (Прямая относительная адресация)  \nl
  4a0   & F00F        & BEQ IP+F    & Переход, если равенство  \nl
  4a1   & 0200        & CLA         & Очистка аккумулятора  \nl
  4a2   & 1207        & IN 5        & Чтение регистра состояния ВУ-2  \nl
  4a3   & 2F40        & AND \#0x40  & Логическое умножение (Прямая загрузка операнда)  \nl
  4a4   & F0FD        & BEQ IP-3    & Переход, если равенство  \nl
  4a5   & 1206        & IN 4        & Чтение регистра данных ВУ-2 \nl
  4a6   & 0680        & SWAB        & Обмен ст. и мл. байтов  \nl
  4a7   & 3EEF        & OR IP-11    & Логическое или (Прямая относительная адресация)  \nl
  4a8   & E8EC        & ST (IP-14)  & Сохранение (Косвенная относительная адресация)  \nl
  4a9   & 6EED        & SUB IP-13   & Вычитание (Прямая относительная адресация)  \nl
  4aa   & 0680        & SWAB        & Обмен ст. и мл. байтов  \nl
  4ab   & 7EEA        & CMP IP-16   & Сравнение (Прямая относительная адресация)  \nl
  4ac   & F003        & BEQ IP+3    & Переход, если равенство  \nl
  4ad   & AAE7        & LD (IP-19)+ & Загрузка (Косвенная относительная автоинкрементная адресация)  \nl
  4ae   & 0200        & CLA         & Очистка аккумулятора  \nl
  4af   & CEE9        & BR IP-17    & Безусловный переход  \nl
  4b0   & AAE4        & LD (IP-1C)+ & Загрузка (Косвенная относительная автоинкрементная адресация)  \nl
  4b1   & 0100        & HLT         & Остановка  \nl
\end{tabular}

\section{Описание программы}

Программа осуществляет посимвольный асинхронный ввод данных с ВУ-2, посимвольно записывает их в память. Программа будет получать символы до тех пор, пока на ВУ-2 не будет введен стоп-символ, который она запишет в память и прекратит свое выполнение.

\section{Область представления}
\begin{itemize}
  \item	res – 11-разрядная ячейка со ссылкой на результат.
  \item terminator – 16-разрядная константа.
  \item tmp – 16-разрядный буфер для временного хранения введенных символов.
  \item 616 - ? – 16-разрядные ячейки, хранящие в себе строку.
\end{itemize}

\section{Расположение данных в памяти}

\begin{itemize}
  \item 0x498-0x4b1 – команды.
  \item 0x495 - 0x497 – переменные.
  \item 0x616 - ? – итоговый результат.
\end{itemize}


\section{Адреса первой и последней выполняемой команды}

\begin{itemize}
  \item Адрес первой команды: 498
  \item Адрес последней команды: 4b1
\end{itemize}

\section{Область допустимых значений}
\begin{itemize}
  \item res (указатель на ячейки массива, хранящий результат ввода) $\in$ [0x616;2047]
  \item temp (ячейка для записи нечетных символов)  $\in$ [0;255], т.к. в нее записывается только 1
        символ из 8 бит.
  \item Введенный символ: [00; FF]
\end{itemize}
Адрес первого элемента массива равен 562 по условию. Т.к. 2047 – 0x616 = 489 – кол-во ячеек,
которые могут использоваться для записи результата, 489*2 = 978 – максимально возможное
кол-во введенных символов (т.к. в данной кодировке символ занимает 1 байт), включая обязательный
стоп-символ. Кол-во введенных символов  $\in$ [1;978].


\section{Таблица трассировки}

\begin{tabular}{|c|c|c|c|c|c|c|c|c|c|c|c|c|c|c|} \hline
  Адр & Код  & IP  & CR   & AR  & DR   & SP  & BR   & AC   & PS  & NZVC     & Адр  Код \nl
  498 & 200  & 499 & 0200 & 498 & 0200 & 000 & 0498 & 0000 & 004 & 0100 \nl
  499 & 205  & 49A & 1205 & 499 & 1205 & 000 & 0499 & 0040 & 004 & 0100 \nl
  49A & F40  & 49B & 2F40 & 49A & 0040 & 000 & 0040 & 0040 & 000 & 0000 \nl
  49B & 0FD  & 49C & F0FD & 49B & F0FD & 000 & 049B & 0040 & 000 & 0000 \nl
  49C & 204  & 49D & 1204 & 49C & 1204 & 000 & 049C & 00C1 & 000 & 0000 \nl
  49D & 8F7  & 49E & E8F7 & 616 & 00C1 & 000 & FFF7 & 00C1 & 000 & 0000 616 & 00C1\nl
  49E & EF8  & 49F & EEF8 & 497 & 00C1 & 000 & FFF8 & 00C1 & 000 & 0000 497 & 00C1\nl
  49F & EF6  & 4A0 & 7EF6 & 496 & 0000 & 000 & FFF6 & 00C1 & 001 & 0001 \nl
  4A0 & 00F  & 4A1 & F00F & 4A0 & F00F & 000 & 04A0 & 00C1 & 001 & 0001 \nl
  4A1 & 0200 & 4A2 & 0200 & 4A1 & 0200 & 000 & 04A1 & 0000 & 005 & 0101 \nl
  4A2 & 1205 & 4A3 & 1205 & 4A2 & 1205 & 000 & 04A2 & 0000 & 005 & 0101 \nl
  4A3 & 2F40 & 4A4 & 2F40 & 4A3 & 0040 & 000 & 0040 & 0000 & 005 & 0101 \nl
  4A4 & F0FD & 4A2 & F0FD & 4A4 & F0FD & 000 & FFFD & 0000 & 005 & 0101 \nl
  4A2 & 1205 & 4A3 & 1205 & 4A2 & 1205 & 000 & 04A2 & 0000 & 005 & 0101 \nl
  4A3 & 2F40 & 4A4 & 2F40 & 4A3 & 0040 & 000 & 0040 & 0000 & 005 & 0101 \nl
  4A4 & F0FD & 4A4 & 0000 & 000 & 0000 & 000 & 0000 & 0000 & 004 & 0100 \nl
  4A4 & F0FD & 4A2 & F0FD & 4A4 & F0FD & 000 & FFFD & 0000 & 004 & 0100 \nl
  4A2 & 1205 & 4A3 & 1205 & 4A2 & 1205 & 000 & 04A2 & 0040 & 004 & 0100 \nl
  4A3 & 2F40 & 4A4 & 2F40 & 4A3 & 0040 & 000 & 0040 & 0040 & 000 & 0000 \nl
  4A4 & F0FD & 4A5 & F0FD & 4A4 & F0FD & 000 & 04A4 & 0040 & 000 & 0000 \nl
  4A5 & 1204 & 4A6 & 1204 & 4A5 & 1204 & 000 & 04A5 & 00C4 & 000 & 0000 \nl
  4A6 & 0680 & 4A7 & 0680 & 4A6 & 0680 & 000 & 04A6 & C400 & 008 & 1000 \nl
  4A7 & 3EEF & 4A8 & 3EEF & 497 & 00C1 & 000 & 3B3E & C4C1 & 008 & 1000 \nl
  4A8 & E8EC & 4A9 & E8EC & 616 & C4C1 & 000 & FFEC & C4C1 & 008 & 1000     & 616	C4C1 \nl
  4A9 & 6EED & 4AA & 6EED & 497 & 00C1 & 000 & FFED & C400 & 009 & 1001 \nl
  4AA & 0680 & 4AB & 0680 & 4AA & 0680 & 000 & 04AA & 00C4 & 001 & 0001 \nl
  4AB & 7EEA & 4AC & 7EEA & 496 & 0000 & 000 & FFEA & 00C4 & 001 & 0001 \nl
  4AC & F003 & 4AD & F003 & 4AC & F003 & 000 & 04AC & 00C4 & 001 & 0001 \nl
  4AD & AAE7 & 4AE & AAE7 & 616 & C4C1 & 000 & FFE7 & C4C1 & 009 & 1001     & 495	0617 \nl
  4AE & 0200 & 4AF & 0200 & 4AE & 0200 & 000 & 04AE & 0000 & 005 & 0101 \nl
  4AF & CEE9 & 499 & CEE9 & 4AF & 0499 & 000 & FFE9 & 0000 & 005 & 0101 \nl
  499 & 1205 & 49A & 1205 & 499 & 1205 & 000 & 0499 & 0040 & 005 & 0101 \nl
  49A & 2F40 & 49B & 2F40 & 49A & 0040 & 000 & 0040 & 0040 & 001 & 0001 \nl
  49B & F0FD & 49C & F0FD & 49B & F0FD & 000 & 049B & 0040 & 001 & 0001 \nl
  49C & 1204 & 49D & 1204 & 49C & 1204 & 000 & 049C & 00C1 & 001 & 0001 \nl
  49D & E8F7 & 49E & E8F7 & 617 & 00C1 & 000 & FFF7 & 00C1 & 001 & 0001     & 617	00C1 \nl
  49E & EEF8 & 49F & EEF8 & 497 & 00C1 & 000 & FFF8 & 00C1 & 001 & 0001     & 497	00C1 \nl
  49F & 7EF6 & 4A0 & 7EF6 & 496 & 0000 & 000 & FFF6 & 00C1 & 001 & 0001 \nl
  4A0 & F00F & 4A1 & F00F & 4A0 & F00F & 000 & 04A0 & 00C1 & 001 & 0001 \nl
  4A1 & 0200 & 4A2 & 0200 & 4A1 & 0200 & 000 & 04A1 & 0000 & 005 & 0101 \nl
  4A2 & 1205 & 4A3 & 1205 & 4A2 & 1205 & 000 & 04A2 & 0040 & 005 & 0101 \nl
  4A3 & 2F40 & 4A4 & 2F40 & 4A3 & 0040 & 000 & 0040 & 0040 & 001 & 0001 \nl
  4A4 & F0FD & 4A5 & F0FD & 4A4 & F0FD & 000 & 04A4 & 0040 & 001 & 0001 \nl
  4A5 & 1204 & 4A6 & 1204 & 4A5 & 1204 & 000 & 04A5 & 0000 & 001 & 0001 \nl
  4A6 & 0680 & 4A7 & 0680 & 4A6 & 0680 & 000 & 04A6 & 0000 & 005 & 0101 \nl
  4A7 & 3EEF & 4A8 & 3EEF & 497 & 00C1 & 000 & FF3E & 00C1 & 001 & 0001 \nl
  4A8 & E8EC & 4A9 & E8EC & 617 & 00C1 & 000 & FFEC & 00C1 & 001 & 0001     & 617	00C1 \nl
  4A9 & 6EED & 4AA & 6EED & 497 & 00C1 & 000 & FFED & 0000 & 005 & 0101 \nl
  4AA & 0680 & 4AB & 0680 & 4AA & 0680 & 000 & 04AA & 0000 & 005 & 0101 \nl
  4AB & 7EEA & 4AC & 7EEA & 496 & 0000 & 000 & FFEA & 0000 & 005 & 0101 \nl
  4AC & F003 & 4B0 & F003 & 4AC & F003 & 000 & 0003 & 0000 & 005 & 0101 \nl
  4B0 & AAE4 & 4B1 & AAE4 & 617 & 00C1 & 000 & FFE4 & 00C1 & 001 & 0001     & 495	0618 \nl
  4B1 & 0100 & 4B2 & 0100 & 4B1 & 0100 & 000 & 04B1 & 00C1 & 001 & 0001 \nl
\end{tabular}

\section{Вывод}

При выполнении данной лабораторной работы я познакомился с асинхронным вводом-выводом данных в БЭВМ, узнал о внешних устройствах, их регистрах и принципах работы.

\end{document}
