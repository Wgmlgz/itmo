\documentclass{article}
\usepackage{import}
\import{../../lib/latex/}{wgmlgz}


\begin{document}

\itmo[
  variant=129,
  labn=7,
  discipline=Основы профессиональной деятельности,
  group=P3115,
  student=Владимир Мацюк,
  teacher=Абузов Ярослав Александрович,
  logo=../../lib/img/itmo.png
]


\section{Текст задания}

Синтезировать цикл исполнения для выданных преподавателем команд. Разработать тестовые программы, которые проверяют каждую из синтезированных команд. Загрузить в микропрограммную память БЭВМ циклы исполнения синтезированных команд, загрузить в основную память БЭВМ тестовые программы. Проверить и отладить разработанные тестовые программы и микропрограммы.



\begin{enumerate}
  \item SHR X - сдвиг аккумулятора вправо на X разрядов, 15 разряд заполняется значением 0, количество сдвигов содержится в коде команды. Признаки N/Z/V/C не устанавливать
  \item Код операции - 0F8X
  \item Тестовая программа должна начинаться с адреса 022116
\end{enumerate}

\section{Программа}

\lstinputlisting{code.bcomp}

\section{Область допустимых значений}

$$-128 \le 6x-8 \le 127$$
$$-120 \le 6x \le 135 $$
$$-20 \le\ x \le 22.5 $$
$$x \in [-20; 22]$$
$$-20=FFEC_{16},\ 22=0016_{16}$$


\section{Расположение данных в памяти}

\begin{enumerate}
  \item Вектор прерываний: 0x000 – 0x00F
  \item Переменные: 0x0A – 0x0C
  \item Программа: 0x0D – 0x03B
\end{enumerate}


\section{Область представления}

X, min, max - i16

\section{Вывод}

В ходе выполнения лабораторной работы я изучил обмен данными в режиме прерываний в БЭВМ.

\end{document}
