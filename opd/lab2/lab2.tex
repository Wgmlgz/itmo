\documentclass{article}
\usepackage{import}
\import{../../lib/latex/}{wgmlgz}


\begin{document}

\itmo[
      variant=31158,
      labn=2,
      discipline=Основы профессиональной деятельности,
      group=P3115,
      student=Владимир Мацюк,
      teacher=Пашнин Александр Денисович
]
\lstset{language=Python}

\tableofcontents

\section{Задание}
По выданному преподавателем варианту определить функцию, вычисляемую программой, область представления и область допустимых значений исходных данных и результата, выполнить трассировку программы, предложить вариант с меньшим числом команд. При выполнении работы представлять результат и все операнды арифметических операций знаковыми числами, а логических операций набором из шестнадцати логических значений.
\begin{center}
      \includegraphics[scale=0.8]{task.png}
\end{center}
\section{Таблица комманд}
\begin{tabular}{|c|r|l|l|} \hline
      Адрес & Код команды & Мнемоника & Комментарии \nl
      178   & E184        &           & A \nl
      179   & 0100        &           & B \nl
      17A   & 0200        &           & C \nl
      17B   & + 0200      & CLA       & Очистка аккумулятора \nl
      17C   & 6178        & SUB 0x178 & Вычитание (Прямая абсолютная адресация) \nl
      17D   & 6179        & SUB 0x179 & Вычитание (Прямая абсолютная адресация) \nl
      17E   & E183        & ST 0x183  & Сохранение (Прямая абсолютная адресация) \nl
      17F   & A17A        & LD 0x17A  & Загрузка (Прямая абсолютная адресация) \nl
      180   & 2183        & AND 0x183 & Логическое умножение (Прямая абсолютная адресация) \nl
      181   & E184        & ST 0x184  & Сохранение (Прямая абсолютная адресация) \nl
      182   & 0100        & HLT       & Остановка \nl
      183   & 2183        &           & Временное заначение $(-A -B)$ \nl
      184   & 2183        &           & Результат $((-A -B)\ \&\ C)$ \nl        
\end{tabular}

\section{Формула}

$$ 
      (-A -B)\ \&\ C 
$$
\section{Область допустимых значений}
Пусть: $ X = -A -B $, тогда:
$$ -2^{15} \le R \le 2^{15} - 1 $$
$$ -2^{15} \le X\ \&\ C \le 2^{15} - 1 $$
$$ -2^{15} \le X,\ C \le 2^{15} - 1 $$
$$ -2^{15} \le X \le 2^{15} - 1 $$




\section{Область определения}
$$
      \left[{ \begin{array}{l}
                        \begin{array}{l} -2^{14}+1 \le A,B \le 2^{14} \\
                        \end{array} \\
                        \\
                        \left\{ \begin{array}{l}
                                      -2^{15}+1 \le A   \le 0 \\
                                      0 \le B   \le 2^{15}    \\
                                \end{array}\right.               \\
                        \\
                        \left\{ \begin{array}{l}
                                      0 \le A   \le 2^{15} \\
                                      -2^{15}+1 \le B   \le 0
                                \end{array}\right.
                  \end{array}}\right.
$$

\section{Расположение данных в памяти}
Исходные данные: 0x178, 0x179, 0x17A. \\
Программа: 0x17B-0x182. \\
Промежуточное значение: 0x183. \\
Результат: 0x284. \\

\section{Таблица трассировки}
\section{Уменьшенная программа}
\begin{tabular}{|c|r|l|l|} \hline
      Адрес & Код команды & Мнемоника & Комментарии \nl
      178   & E184        &           & A \nl
      179   & 0100        &           & B \nl
      17A   & 0200        &           & C \nl
      17B   & + 0200      & CLA       & Очистка аккумулятора \nl
      17C   & 6178        & SUB 0x178 & Вычитание (Прямая абсолютная адресация) \nl
      17D   & 6179        & SUB 0x179 & Вычитание (Прямая абсолютная адресация) \nl
      18E   & 2183        & AND 0x183 & Логическое умножение (Прямая абсолютная адресация) \nl
      18F   & E184        & ST 0x184  & Сохранение (Прямая абсолютная адресация) \nl
      182   & 0100        & HLT       & Остановка \nl
      184   & 2183        &           & Результат $((-A -B)\ \&\ C)$ \nl        
\end{tabular}
\section{Вывод}

В ходе данное лабораторной работы я познакомился с БЭВМ и командами. Я научился манипулировать памятью ЭВМ и исполнять базовые программы.
\end{document}
