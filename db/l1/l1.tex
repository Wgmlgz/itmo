\documentclass{article}
\usepackage{import}
% \usepackage{draculatheme}

% \usepackage{colortbl}
% \usepackage[usenames]{color}


\usepackage[paper=letterpaper,margin=2cm]{geometry}
\usepackage[utf8]{inputenc}
\usepackage[russian]{babel}
\usepackage{graphicx}
\usepackage{geometry}
\usepackage{xcolor}
\usepackage{hyperref}
\usepackage{fontspec}
% \usepackage{listings-rust}
\usepackage{listings}
\usepackage{keycommand}
\usepackage{caption}
\usepackage{dingbat}
\usepackage{array,multirow}
\usepackage{amssymb}
\usepackage{spreadtab}
\usepackage{unicode-math}

% \setmonofont{JetBrains Mono}[Contextuals=Alternate,Ligatures = TeX,]
\setmainfont[
  Ligatures=TeX,
  Extension=.otf,
  BoldFont=cmunbx,
  ItalicFont=cmunti,
  BoldItalicFont=cmunbi,
]{cmunrm}
\setsansfont[
  Ligatures=TeX,
  Extension=.otf,
  BoldFont=cmunsx,
  ItalicFont=cmunsi,
]{cmunss}

% \geometry{
%   a4paper,
%   top=25mm,
%   right=0mm,
%   bottom=25mm,
%   left=0mm
% }

\hypersetup{
  colorlinks=true,
  linkcolor=blue!50!red,
  urlcolor=blue!70!black
}

\captionsetup[lstlisting]{
  font={tt},
}

% based on Atom One Light
\lstset{
  frame=single,
  basicstyle=\ttfamily\color[HTML]{383a42},
  columns=fullflexible,
  breaklines=true,
  numbers=left,
  frame=tab,
  postbreak=\mbox{\textcolor{red}{$\hookrightarrow$}\space},
  extendedchars=false,
  showspaces=false,
  showstringspaces=false,
  identifierstyle=\ttfamily\color[HTML]{4078f2},
  commentstyle=\color[HTML]{a0a1a7},
  stringstyle=\color[HTML]{50a14f},
  keywordstyle=\color[HTML]{a626a4},
  numberstyle=\ttfamily\color[HTML]{2c91af},
  rulecolor=\color[HTML]{383a42}
}

\lstdefinelanguage{XML}
{
  morestring=[b]",
  morestring=[s]{>}{<},
  morecomment=[s]{<?}{?>},
}

\newcommand{\code}[1]{
  \lstset{title=#1}
  \lstinputlisting{#1}
}

\newcommand{\MM}[1]{\mathchoice{\hbox{#1}}{\hbox{#1}}{\hbox{\scriptsize{#1}}}{\hbox{\tiny{#1}}}}
\newcommand*\BitAnd{\mathbin{\&}}
\newcommand*\BitOr{\mathbin{|}}
\newcommand*\ShiftLeft{\ll}
\newcommand*\ShiftRight{\gg}
\newcommand*\BitNeg{\ensuremath{\mathord{\sim}}}
\def\nl{\\\hline}

\newkeycommand{\itmo}[
  variant=aboba,
  labn=aboba,
  worktype=Лабораторная работа,
  discipline=aboba,
  group=aboba,
  student=aboba,
  teacher=aboba,
  year=2023,
  logo=../../lib/img/itmo.png
]{
  \begin{titlepage}
    \begin{center}
      \section*{
        Федеральное государственное автономное образовательное учреждение\\ высшего образования\\
        «Национальный исследовательский университет ИТМО»\\
        Факультет Программной Инженерии и Компьютерной Техники \\
       }
      \includegraphics[scale=0.2]{\commandkey{logo}}
    \end{center}
    
    \vspace{4cm}
    
    \begin{center}
      \large \textbf{Вариант \textnumero \commandkey{variant}}\\
      \textbf{\commandkey{worktype} \textnumero \commandkey{labn}}\\
      по дисциплине\\
      \textbf{\commandkey{discipline}}
    \end{center}
    
    \vspace*{\fill}
    
    \begin{flushright}
      Выполнил Студент группы \commandkey{group}\\
      \textbf{\commandkey{student}}\\
      Преподаватель: \\
      \textbf{\commandkey{teacher}}\\
    \end{flushright}
    
    \vspace{1cm}
    
    \begin{center}
      Санкт-Петербург\\
      \commandkey{year}г.
    \end{center}
    
    \thispagestyle{empty}
  \end{titlepage}
}



\begin{document}

\itmo[
  variant=371364,
  labn=1,
  discipline=Информационные системы и базы данных,
  group=P3115,
  student=Владимир Мацюк,
  teacher=Горбунов Михаил Витальевич,
  year=2023,
  logo=../../lib/img/itmo.png
]

\section{Текст задания}
\begin{enumerate}
  \item На основе предложенной предметной области (текста) составить ее описание. Из полученного описания выделить сущности, их атрибуты и связи.
  \item Составить инфологическую модель.
  \item Составить даталогическую модель. При описании типов данных для атрибутов должны использоваться типы из СУБД PostgreSQL.
  \item Реализовать даталогическую модель в PostgreSQL. При описании и реализации даталогической модели должны учитываться ограничения целостности, которые характерны для полученной предметной области.
  \item Заполнить созданные таблицы тестовыми данными.
\end{enumerate}
\subsection{Описание предметной области, по которой должна быть построена доменная модель}

Первого ноября 1907 года в полицию Нового Орлеана поступили отчаянные заявления из южных районов, местностей болот и лагун, Тамошние поселенцы, в основном грубые, но дружелюбные потомки племени Лафитта, были охвачены ужасом в результате непонятного явления, происшедшего ночью, Это было несомненно колдовство, но колдовство столь кошмарное, что им такое не могло даже придти в голову; некоторые из женщин и детей исчезли с того момента, как зловещие звуки тамтама начали доноситься из глубин черного леса, в который не решался заходить ни дин из местных жителей. Оттуда слышались безумные крики и вопли истязаемых, леденящее душу пение, видны были дьявольские пляски огоньков; всего этого, как заключил напуганный посланник, люди уже не могли выносить.

\section{Список сущностей и их классификация}

Стержневвые:
\begin{itemize}
  \item Incedent: incedent type, message, datetime, location, description
  \item Entity: type fullname info
  \item Area: postition, info
  \item Endpoint: info, area
\end{itemize}
Характеристические:
\begin{itemize}
  \item Report: from endpoint, to endpoint, arrive date, content
  \item Incications: incedent, entity, info
\end{itemize}
Ассоциативные:
\begin{itemize}
  \item Victim: entity, incedent, victim type
\end{itemize}

\section{Инфологическая модель}
\begin{center}
  \includegraphics[scale=0.6]{infodiagram.png}
\end{center}

\section{Даталогическая модель}
\includegraphics[width=\textwidth]{dbdiagram.png}

\section{Реализация даталогической модели на SQL}

\lstset{language=SQL}
\code{create.psql}

\section{Вывод}
Во время выполнения работы я узнал что такое инфологическая и даталогическая модели.
\end{document}
